\section{Entrada y salida de la información}

  \paragraph{}Al tratarse de una aplicación web, el usuario podrá encontrar los
  elementos típicos que aparecen en ellas para interactuar con dicha aplicación,
  como son:

  \begin{itemize}
   \item \textbf{Botones}
   Elemento que permite la interacción del usuario con la página
   y que conlleva una acción inmediata, como por ejemplo, confirmación de los
   datos de un formulario. Los botones pueden tener diferente aspecto, pero
   todos cumplen un mismo papel, que es el de enlazar unas páginas con otras
   pasándole la información determinada que hayamos rellenado en cada página
   para que se evalúe en la siguiente, o bien cambia de página sin enviar
   información a la base de datos. En la figura \textit{INSERTAR} vemos un
   ejemplo de este tipo de botones. Indica \textit{INSERTAR}.
   \item \textbf{Formularios}
   Son un conjunto de elementos que se usan para que el
   usuario introduzca o elija información en el sistema, por ejemplo, los
   datos de una nueva titulación se introducen en el sistema mediante un
   formulario.

   La figura \textit{INSERTAR} muestra la captura de pantalla de la ventana de
   creación de una titulación para el usuario administrador principal. A
   continuación se describen los elementos de un formulario mediante la
   numeración que aparece en la figura \textit{INSERTAR}. Los elementos que
   pueden contener los formularios son: \textit{INSERTAR}

   \item \textbf{Hiperenlaces}
   Los hiperenlaces tienen una función similar a los botones
   y servirán para enlazar páginas en el seno de la aplicación. En este caso,
   se han utilizado para enlazar páginas con información relacionada
   introduciendo nuevas opciones de acción de una forma estructurada y
   organizada sin acumular gran cantidad de información en una misma página.

   Ejemplos: \textit{INSERTAR}
   \item \textbf{Menús} Permiten al usuario elegir una opción entre las presentes. Un
   menú lo forman un conjunto de hiperenlaces como los mostrados en la figura
   \textit{INSERTAR}.
   \end{itemize}