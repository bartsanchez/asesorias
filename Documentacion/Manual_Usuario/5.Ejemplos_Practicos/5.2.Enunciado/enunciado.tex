\section{Enunciado del ejemplo}\label{enunciado}

  \paragraph{}Los datos empleados en este ejemplo no pertenecen, en ningún caso,
  a la vida real. Son puramente ficticios y han sido incluidos a título
  didáctico.

  \paragraph{}Se trata de un alumno de la Universidad de Córdoba, llamado
  Julio Carlos Cobos Martínez, con DNI número 12345678A y cuyo correo
  electrónico es \textit{jccobos@kmail.com}. Este alumno pertenece a la
  titulación \textit{Grado de Bioquímica}, cuyo plan de estudios es del año 2000
  y que pertenece al centro \textit{Facultad de Ciencias}. Durante el curso
  académico 2010 el alumno se matricula de tres asignaturas su titulación, que
  son:

  \begin{itemize}
    \item \textbf{Biología molecular}.
    \begin{itemize}
      \item Curso: 1º.
      \item Tipo: Troncal.
      \item Número de créditos teóricos: 4.
      \item Número de créditos prácticos: 2.
    \end{itemize}
    \item \textbf{Biosíntesis de macromoléculas}.
    \begin{itemize}
      \item Curso: 2º.
      \item Tipo: Obligatoria.
      \item Número de créditos teóricos: 3.
      \item Número de créditos prácticos: 3.
    \end{itemize}
    \item \textbf{Genética humana}.
    \begin{itemize}
      \item Curso: 4º.
      \item Tipo: Optativa.
      \item Número de créditos teóricos: 2.
      \item Número de créditos prácticos: 4.
    \end{itemize}
  \end{itemize}

  \paragraph{}A este alumno se le asigna el asesor José Manuel, del departamento
  de \textit{Biología celular}, cuyo DNI es 99887766Z, durante el curso
  académico en que se ha matriculado; es decir, 2010.

  \paragraph{}Con objeto de aconsejar al alumno durante la superación de su
  carrera, dicho asesor accede a la aplicación para ver la matriculación
  realizada por su alumno durante ese año. Al observar que el alumno se ha
  matriculado de pocas asignaturas, tres en este caso, decide convocar una
  reunión individual para preguntarle el motivo de su escasa matriculación.

  \paragraph{}En esta reunión individual quiere proponerle cuestiones que no
  tiene en sus plantillas de asesor, pues este usuario no disponía de plantillas
  de asesor con anterioridad. Por lo tanto, el asesor crea para sí mismo una
  plantilla denominada \textit{Escasa matriculación}, que contendrá las
  preguntas:

  \begin{itemize}
   \item ¿Cuántas horas semanales dedica a sus estudios?
   \item ¿Compagina estudios y trabajo?
   \item Indique cuántos años cree que tardará en terminar la carrera.
  \end{itemize}

  \paragraph{}Por su parte, el alumno accede a la aplicación y descubre que se
  le ha propuesto la reunión anteriormente mencionada y decide responder a las
  preguntas, a las que responde: \textit{10 horas semanales},
  \textit{Sí}, y \textit{6 años}, por este orden.

  \paragraph{}De esta manera, al volver a acceder el usuario asesor al sistema
  y ver las respuestas a las cuestiones planteadas, alcanza a comprender a que
  se debe la escasa matriculación del alumno en la Universidad, al entender que
  compagina estudios y trabajo.

