\section{Enunciado del ejemplo}

  \paragraph{}Los datos empleados en este ejemplo no pertenecen, en ningún caso,
  a la vida real. Son puramente ficticios y han sido incluidos a título
  didáctico.

  \paragraph{}Se trata de un alumno de la Universidad de Córdoba, llamado
  Julio Carlos Cobos Martínez, con DNI número 12345678A y cuyo correo
  electrónico es \textit{jccobos@kmail.com}. Este alumno pertenece a la
  titulación \textit{Grado de Bioquímica}, cuyo plan de estudios es del año 2000
  y que pertenece al centro \textit{Facultad de Ciencias}. Durante el curso
  académico 2010 el alumno se matricula de tres asignaturas su titulación, que
  son:

  \begin{itemize}
    \item \textbf{Biología molecular}.
    \begin{itemize}
      \item Curso: 1º.
      \item Tipo: Troncal.
      \item Número de créditos teóricos: 4.
      \item Número de créditos prácticos: 2.
    \end{itemize}
    \item \textbf{Biosíntesis de macromoléculas}.
    \begin{itemize}
      \item Curso: 2º.
      \item Tipo: Obligatoria.
      \item Número de créditos teóricos: 3.
      \item Número de créditos prácticos: 3.
    \end{itemize}
    \item \textbf{Genética humana}.
    \begin{itemize}
      \item Curso: 4º.
      \item Tipo: Optativa.
      \item Número de créditos teóricos: 2.
      \item Número de créditos prácticos: 4.
    \end{itemize}
  \end{itemize}
