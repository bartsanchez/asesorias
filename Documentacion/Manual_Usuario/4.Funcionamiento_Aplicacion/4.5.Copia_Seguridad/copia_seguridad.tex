\section{Copia de seguridad}

  \paragraph{}La aplicación permite la realización de copias de seguridad con
  el objeto de evitar pérdidas de información existente en el sistema, y
  mantener una apropiada integridad de la misma.

  \paragraph{}Dado que el manejo de copias de seguridad no es una opción muy
  corriente para la mayoría de los usuarios que utilizarán la aplicación, esta
  opción queda restringida al usuario \textit{Administrador principal}. Además,
  es indispensable que dicho administrador tenga acceso al sistema operativo
  del servidor donde está funcionando la aplicación.

  \paragraph{}Aprovechando que el software utilizado,
  \textit{Django}\cite{django} en concreto, permite la creación y restauración
  de copias de seguridad, estas operaciones se realizarán a través del mismo.

\subsection{Creación de una copia de seguridad}

  \paragraph{}Para la creación de una copia de seguridad de la información
  existente en el sistema, deberá situar el \textit{prompt} en la ruta absoluta
  al directorio \textit{proyecto} que contiene el código fuente proporcionado
  por esta aplicación. Por ejemplo, si el código fuente de la aplicación ha sido
  descargado al directorio \textit{home} del usuario \textit{usuario}, la ruta
  debería ser: \textit{/home/usuario/}. Una vez situado en dicha ruta, se debe
  introducir en la línea de comandos la siguiente orden:

   \begin{verbatim}
   $ python manage.py dumpdata > copia_seguridad.json
   \end{verbatim}

  \paragraph{}Nótese que \textit{copia\_seguridad} es el nombre que hemos puesto
  a nuestro archivo que contiene la información del sistema, una vez guardada,
  y \textit{json} es el formato con el que se guarda dicho archivo. No obstante,
  en realidad no es obligatorio establecer una extensión a nuestro archivo de
  copia de seguridad, pero es altamente recomendable ya que sugiere con mayor
  precisión qué tipo de información contiene el archivo.

  \paragraph{}Una vez introducida la orden, se habrá generado un archivo, en el
  directorio donde ejecutamos el comando, con toda la información existente
  en la base de datos de la aplicación.

\subsection{Restauración de una copia de seguridad}

  \paragraph{}Para la restauración de una copia de seguridad evidentemente
  necesitamos haber realizado con anterioridad la creación de una copia de
  seguridad.

  \paragraph{}El archivo que se generó al realizar la copia de seguridad es
  necesario copiarlo al directorio \textit{proyecto} que contiene el código
  fuente proporcionado por esta aplicación. Por ejemplo, si el código fuente de
  la aplicación ha sido descargado al directorio \textit{home} del usuario
  \textit{usuario}, la ruta debería ser: \textit{/home/usuario/}. Seguidamente
  nos situaremos en dicho directorio y ejecutaremos la orden:

   \begin{verbatim}
   $ python manage.py loaddata copia_seguridad.json
   \end{verbatim}

  \paragraph{}Nótese que \textit{copia\_seguridad} es el nombre del archivo que
  contiene la información del sistema, una vez guardada, y \textit{json} es la
  extensión que indica el formato con el que se había guarda dicho archivo, si
  lo tuviera.

  \paragraph{}Una vez introducida la orden, se habrá cargado en el sistema toda
  la información que existía en la base de datos de la aplicación en el momento
  en que se realizó la copia de seguridad.
