\subsection{Introducción}

  \paragraph{}El usuario \textit{Administrador principal} (solo puede existir un
  único usuario de este tipo en el sistema) es el máximo responsable del
  correcto funcionamiento de la aplicación. Como tal, tiene todos los
  privilegios posibles a la hora de editar cualquier información del sistema,
  que es la que aparece a continuación:

  \begin{itemize}
    \item Organización institucional
    \begin{itemize}
      \item Centro
      \item Administrador de centro
      \item Centro- Administrador de centro
      \item Titulación
      \item Asignatura
      \item Asignatura curso académico
    \end{itemize}
   \item Organización docente
   \begin{itemize}
      \item Departamento
      \item Asesor
      \item Asesor curso académico
   \end{itemize}
   \item Alumnos
   \begin{itemize}
      \item Alumno
      \item Alumno curso académico
      \item Matrícula
      \item Calificación convocatoria
   \end{itemize}
   \item Reuniones
   \begin{itemize}
      \item Reunión
      \item Reunión - Pregunta de asesor
      \item Reunión - Pregunta oficial
   \end{itemize}
   \item Plantillas de entrevista
   \begin{itemize}
      \item Plantilla de entrevista oficial
      \item Pregunta oficial
      \item Plantilla de entrevista de asesor
      \item Pregunta de asesor
   \end{itemize}
  \end{itemize}

  \paragraph{}Para acceder a cada uno de estos elementos, se podrá realizar de
  dos formas distintas:

  \begin{itemize}
   \item Menú principal
   \item Menú lateral
  \end{itemize}


  \paragraph{}A continuación, se pasa a detallar todos y cada uno de los
  elementos que este usuario puede gestionar en el sistema.

