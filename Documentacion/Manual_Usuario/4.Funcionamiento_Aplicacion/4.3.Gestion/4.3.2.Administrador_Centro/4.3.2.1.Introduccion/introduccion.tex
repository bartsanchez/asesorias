\subsection{Introducción}

  \paragraph{}El usuario \textit{Administrador de centro} es responsable del
  correcto funcionamiento de la información relativa a un determinado centro
  existente en la aplicación. Como tal, tendrá ciertos privilegios a la hora de
  editar información relativa a dicho centro, que es la que aparece a
  continuación:

  \begin{itemize}
    \item Organización institucional
    \begin{itemize}
      \item Titulación
      \item Asignatura
      \item Asignatura curso académico
    \end{itemize}
   \item Organización docente
   \begin{itemize}
      \item Asesores curso académico
      \item Alumnos curso académico
      \item Matrícula
      \item Calificación convocatoria
   \end{itemize}
  \end{itemize}

  \paragraph{}Además, a este usuario le estará permitido modificar cierta
  información personal; en este caso, su contraseña.

  \paragraph{}Para acceder a cada uno de estos elementos, se podrá realizar de
  dos formas distintas:

  \begin{itemize}
   \item Mediante el menú principal. Este menú es el que refleja la figura
   \ref{capturaMenuPrincipalAdminCentro}.

  \begin{figure}[!ht]
    \begin{center}
      \includegraphics[scale=0.55]{4.Funcionamiento_Aplicacion/4.3.Gestion/4.3.2.Administrador_Centro/4.3.2.1.Introduccion/menu_principal.png}
      \caption{Captura del menú principal del usuario \textit{Administrador de centro}.}
      \label{capturaMenuPrincipalAdminCentro}
    \end{center}
  \end{figure}

   \item Mediante el menú lateral. Este menú es el que refleja la figura
   \ref{capturaMenuLateralAdminCentro}.

   \begin{figure}[!ht]
    \begin{center}
      \includegraphics[scale=0.55]{4.Funcionamiento_Aplicacion/4.3.Gestion/4.3.2.Administrador_Centro/4.3.2.1.Introduccion/menu_lateral.png}
      \caption{Captura del menú lateral del usuario \textit{Administrador de centro}.}
      \label{capturaMenuLateralAdminCentro}
    \end{center}
  \end{figure}

  \end{itemize}

  \paragraph{}A continuación, se pasa a detallar todos y cada uno de los
  elementos que este usuario puede gestionar en el sistema.
