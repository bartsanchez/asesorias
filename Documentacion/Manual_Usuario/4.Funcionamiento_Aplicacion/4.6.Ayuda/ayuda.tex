\section{Ayuda}

  \paragraph{}Con el objetivo de orientar y resolver las posibles dudas que
  puedan tener en cualquier momento los usuarios de la aplicación,
  permanentemente se proporciona un enlace que, al pulsarlo, nos lleva
  directamente a una página de ayuda. Este enlace es el que refleja la figura
  \ref{capturaEnlaceAyuda}, y está situado en el menú de sesión descrito en el
  capítulo \ref{gestionInformacion}, \textit{Gestión de la información}.

  \begin{figure}[!ht]
    \begin{center}
      \fbox{
      \includegraphics[scale=0.6]{4.Funcionamiento_Aplicacion/4.6.Ayuda/enlaceAyuda.png}
      }
      \caption{Captura de pantalla del enlace \textit{Ayuda}.}
      \label{capturaEnlaceAyuda}
    \end{center}
  \end{figure}

  \paragraph{}Esta página de ayuda es dependiente del tipo de usuario que esté
  utilizando la aplicación en un determinado momento. Es decir, el contenido de
  la ayuda variará en función del rol con el que se participe en la aplicación,
  que puede ser \textit{Administrador principal},
  \textit{Administrador de centro}, \textit{Asesor} o \textit{Alumno}. Esto es
  debido a que cada uno de estos tipos de usuarios le están permitidas ciertas
  funcionalidades específicas, en función del objetivo con el que participen en
  el sistema.
