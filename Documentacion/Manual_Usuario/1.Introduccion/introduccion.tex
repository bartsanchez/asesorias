\section{Introducción}

  \paragraph{}En este manual de usuario se realiza una descripción de la
  aplicación \textit{``Asesorías Académicas''}, centrándose únicamente en la
  entrada de información del usuario y en la salida resultante y describiendo,
  minuciosamente, tanto el proceso de instalación y desinstalación como la
  navegación a través de las pantallas que constituyen la misma.

  \paragraph{}El usuario final de la aplicación podrá encontrar todo lo
  necesario para conocer y explotar las diferentes posibilidades del sistema
  informático implementado, utilizando este manual de usuario como guía de
  iniciación para el aprendizaje y utilización del software desarrollado. Para
  facilitar este proceso, el documento se ha dividido en cuatro partes:

  \begin{itemize}
    \item \textbf{Instalación y desinstalación}. Este capítulo se encarga de
    detallar al máximo todos los pasos que se deben seguir en el proceso de
    instalación de la aplicación Asesorías Académicas para su correcto
    funcionamiento.
    \item \textbf{Características de la interfaz}. Esta parte servirá de
    diccionario a lo largo de todo el manual, ya que se encarga de la definición
    de todos los términos gráficos que se utilizarán posteriormente.
    \item \textbf{Funcionamiento de la aplicación}. En este capítulo, se
    realiza una navegación exhaustiva por todas las partes de la aplicación,
    detallando la funcionalidad de cada una de ellas.
    \item \textbf{Ejemplos prácticos}. En esta parte se proporcionan ejemplos
    útiles que sirven de guía para los usuarios de la aplicación.
  \end{itemize}

  \paragraph{}Como objetivo implícito, este documento pretende ser utilizado
  como base de posteriores estudios, futuros proyectos o posibles modificaciones
  y mejoras que se planteen en el marco de su desarrollo.
