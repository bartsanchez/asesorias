\item \textbf{Desinstalación del sistema gestor de bases de datos}
   Para proceder a la desinstalación del sistema gestor de bases de datos
   utilizado, \textit{PostgreSQL}, deberá introducirse en la línea de comandos:

   \begin{verbatim}
   # aptitude purge postgresql
   \end{verbatim}

   Aparecerá la siguiente información:

   \begin{verbatim}
   Se ELIMINARÁN los siguientes paquetes:
   postgresql{p} postgresql-8.4{u} postgresql-common{u}
   ssl-cert{u}
   0 paquetes actualizados, 0 nuevos instalados, 4 para eliminar
   y 0 sin actualizar.
   Necesito descargar 0 B de ficheros. Después de desempaquetar
   se liberarán 15,8 MB.
   ¿Quiere continuar? [Y/n/?]
   \end{verbatim}

   Introduciendo el carácter \textit{Y}, o pulsando \textit{Intro}, se procederá
   a la desinstalación completa. El sistema devuelve la siguiente información,
   devolviendo el \textit{prompt} a su estado inicial:

   \begin{verbatim}
   (Leyendo la base de datos ... 200858 files and directories
   currently installed.)
   Desinstalando postgresql ...
   (Leyendo la base de datos ... 200854 files and directories
   currently installed.)
   Desinstalando postgresql-8.4 ...
   Stopping PostgreSQL 8.4 database server: main.
   Desinstalando postgresql-common ...
   Desinstalando ssl-cert ...
   Procesando disparadores para man-db ...
   \end{verbatim}
