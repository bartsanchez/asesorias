\item \textbf{Desinstalación del servidor web Apache}.
   Se procede a desinstalar el servidor web junto con su módulo
   \textit{mod\_wsgi}. Para ello, escribiremos el siguiente comando:

   \begin{verbatim}
   # aptitude purge apache2 libapache2-mod-wsgi
   \end{verbatim}

   Seguidamente aparecerá el siguiente mensaje:

   \begin{verbatim}
   Se ELIMINARÁN los siguientes paquetes:
   apache2{p} apache2-mpm-worker{u} apache2-utils{u}
   apache2.2-bin{u} apache2.2-common{u}
   libapache2-mod-wsgi{p} libaprutil1-dbd-sqlite3{u}
   libaprutil1-ldap{u}
   0 paquetes actualizados, 0 nuevos instalados, 8 para eliminar
   y 0 sin actualizar.
   Necesito descargar 0 B de ficheros. Después de desempaquetar se
   liberarán 6693 kB.
   ¿Quiere continuar? [Y/n/?]
   \end{verbatim}

   Introduciendo el carácter \textit{Y}, o pulsando \textit{Intro}, se procederá
   a la desinstalación completa. El sistema devuelve la siguiente información,
   devolviendo el \textit{prompt} a su estado inicial:

   \begin{verbatim}
   (Leyendo la base de datos ... 201320 files and directories
   currently installed.)
   Desinstalando apache2 ...
   Desinstalando libapache2-mod-wsgi ...
   Module wsgi disabled.
   Run '/etc/init.d/apache2 restart' to activate new
   configuration!
   Purgando ficheros de configuración de libapache2-mod-wsgi ...
   (Leyendo la base de datos ... 201306 files and directories
   currently installed.)
   Desinstalando apache2-mpm-worker ...
   Stopping web server: apache2.
   Desinstalando apache2.2-common ...
   Desinstalando apache2-utils ...
   Desinstalando apache2.2-bin ...
   Desinstalando libaprutil1-dbd-sqlite3 ...
   Desinstalando libaprutil1-ldap ...
   Procesando disparadores para man-db ...
   \end{verbatim}
