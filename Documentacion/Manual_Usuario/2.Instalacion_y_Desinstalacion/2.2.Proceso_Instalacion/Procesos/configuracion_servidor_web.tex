\item \textbf{Configuración del servidor web}.
   Para llevar a cabo la configuración del servidor web, debemos hacer uso del
   contenido de la carpeta \textit{public} que acompaña al código fuente de
   esta aplicación.

   En primer lugar, para realizar la configuración del servidor \textit{Apache}
   con su módulo \textit{mod\_wsgi}, se proporciona al usuario el archivo
   \textit{httpd.conf} situado en \textit{public/wsgi-script/}. Este archivo
   es necesario sustituirlo por el archivo de configuración de \textit{Apache}
   del mismo nombre, situado en la ruta del sistema \textit{/etc/apache2/}. Para
   realizar dicha sustitución, introducimos el siguiente comando:

   \begin{verbatim}
   # cp /ruta-proyecto/public/wsgi-scripts/httpd.conf
   /etc/apache2/
   \end{verbatim}

   Nótese que \textit{ruta-proyecto} debe ser la ruta absoluta al directorio
   \textit{proyecto} que contiene el código fuente proporcionado por esta
   aplicación. Por ejemplo, si el código fuente de la aplicación ha sido
   descargado al directorio \textit{home} del usuario \textit{usuario},
   la ruta debería ser: \textit{/home/usuario/}

   Este archivo hay que modificarlo para configurar correctamente el servidor
   web, ya que hay que tener en cuenta la ruta absoluta del código fuente de
   la aplicación. Para ello, editamos el fichero con nuestro editor favorito
   (en este caso Nano \cite{nano}), teniendo en cuenta que necesitamos
   permisos de administrador del sistema:

   \begin{verbatim}
   # nano /etc/apache2/httpd.conf
   \end{verbatim}

   Una vez dentro, debemos sustituir nuevamente la cadena
   \textit{/ruta-proyecto/} tal y como hicimos en el paso anterior.

   Además, es necesario establecer que el archivo de configuración para el
   módulo \textit{mod\_wsgi} sea accesible para \textit{Apache}, por lo que
   realizaremos una copia al directorio empleado para tal fin:

   \begin{verbatim}
   # cp -R /ruta-proyecto/public/wsgi-script/ /var/www/
   \end{verbatim}

   El archivo que contiene esta carpeta también es necesario modificarlo, por
   lo que escribimos:

   \begin{verbatim}
   # nano /var/www/wsgi-scripts/django.wsgi
   \end{verbatim}

   Y una vez dentro del editor sustituimos de nuevo \textit{/ruta-proyecto/}
   (nótese que aparece dos veces en el fichero) tal y como lo hicimos en los
   pasos anteriores.

   Llegados a este punto es necesario proporcionar a \textit{Apache} los
   archivos multimedia (imágenes y hojas de estilo) necesarias para la ejecución
   de la aplicación. Para realizar dicha acción, introduciremos por línea de
   comandos:

   \begin{verbatim}
   # cp -R /ruta-proyecto/public/media /var/www/
   \end{verbatim}

   Reiniciaremos \textit{Apache} para comprobar que los cambios han tenido
   efecto:

   \begin{verbatim}
   # /etc/init.d/apache2 restart
   \end{verbatim}

   Lo que provocará el siguiente mensaje, si no hay errores:

   \begin{verbatim}
   Restarting web server: apache2 ... waiting .
   \end{verbatim}

   Para comprobar que todo ha ido bien, abriremos un navegador e introduciremos
   la URL \textit{http://localhost/asesorias/}. Si se ha realizado la
   instalación satisfactoriamente, debería aparecer la pantalla principal
   de la aplicación vista en el capítulo \ref{paginaPrincipal},
   \textit{Página principal}.