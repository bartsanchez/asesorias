\item \textbf{Instalación del \textit{framework}}.
   Para proceder con la instalación del \textit{framework} utilizado,
   \textit{Django}, deberá introducirse en la línea de
   comandos\footnote{Nótese que las órdenes ejecutadas en línea de comandos
   mostrados a continuación llevarán un carácter \textit{\#}, en caso de que la
   orden deba ser ejecutada por el usuario administrador del sistema, o por el
   contrario el carácter \textit{\$}, en caso de que la orden deba ser ejecutada
   por cualquier usuario, sin necesidad de tener los privilegios del
   administrador del sistema.}:

   \begin{verbatim}
   # aptitude install python-django
   \end{verbatim}

   Seguidamente aparecerá el siguiente mensaje:

   \begin{verbatim}
   Se instalarán los siguiente paquetes NUEVOS:
   python-django
   0 paquetes actualizados, 1 nuevos instalados, 0 para eliminar
   y 0 sin actualizar.
   Necesito descargar 4194 kB de ficheros. Después de
   desempaquetar se usarán 20,3 MB.
   Seleccionando el paquete python-django previamente no
   seleccionado.
   (Leyendo la base de datos ... 199153 files and directories
   currently installed.)
   Desempaquetando python-django
   (de .../python-django_1.2.3-2_all.deb) ...
   Procesando disparadores para man-db ...
   Configurando python-django (1.2.3-2) ...
   Procesando disparadores para python-support ...
   \end{verbatim}

   Si no ha habido ningún error, el \textit{prompt}\footnote{Según Wikipedia
   \cite{wikipedia2}: \textit{``Se llama prompt al carácter o conjunto de
   caracteres que se muestran en una línea de comandos para indicar que está a
   la espera de órdenes. Éste puede variar dependiendo del intérprete de
   comandos y suele ser configurable.''}} volverá a su situación inicial,
   quedando la aplicación correctamente instalada.