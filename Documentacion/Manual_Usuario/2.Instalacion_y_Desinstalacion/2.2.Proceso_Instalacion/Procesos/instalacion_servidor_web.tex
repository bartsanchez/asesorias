\item \textbf{Instalación del servidor web}.
   Se procede a instalar el servidor web utilizado, \textit{Apache}, junto
   con su módulo \textit{mod\_wsgi}. Este módulo es necesario para que
   \textit{Apache} sea capaz de interpretar el lenguaje \textit{Python}. Para
   llevar a cabo la instalación, escribiremos el siguiente
   comando:

   \begin{verbatim}
   # aptitude install apache2 libapache2-mod-wsgi
   \end{verbatim}

   Seguidamente aparecerá el siguiente mensaje:

   \begin{verbatim}
   Se instalarán los siguiente paquetes NUEVOS:
   apache2 apache2-mpm-worker{a} apache2-utils{a} apache2.2-bin{a}
   apache2.2-common{a} libapache2-mod-wsgi
   libaprutil1-dbd-sqlite3{a} libaprutil1-ldap{a} ssl-cert{a}
   0 paquetes actualizados, 9 nuevos instalados, 0 para eliminar
   y 0 sin actualizar.
   Necesito descargar 2016 kB de ficheros. Después de
   desempaquetar se usarán 6808 kB.
   ¿Quiere continuar? [Y/n/?]
   \end{verbatim}

   Introduciendo el carácter \textit{Y}, o pulsando \textit{Intro}, se procederá
   con la instalación completa. El sistema devuelve la siguiente información,
   devolviendo el \textit{prompt} a su estado inicial:

   \begin{verbatim}
   Des:1 http://ftp.fr.debian.org/debian/ testing/main
   libaprutil1-dbd-sqlite3 i386 1.3.9+dfsg-5 [27,2 kB]
   Des:2 http://ftp.fr.debian.org/debian/ testing/main
   libaprutil1-ldap i386 1.3.9+dfsg-5 [25,3 kB]
   Des:3 http://ftp.fr.debian.org/debian/ testing/main
   apache2.2-bin i386 2.2.16-4 [1345 kB]
   Des:4 http://ftp.fr.debian.org/debian/ testing/main
   apache2-utils i386 2.2.16-4 [164 kB]
   Des:5 http://ftp.fr.debian.org/debian/ testing/main
   apache2.2-common i386 2.2.16-4 [307 kB]
   Des:6 http://ftp.fr.debian.org/debian/ testing/main
   apache2-mpm-worker i386 2.2.16-4 [2220 B]
   Des:7 http://ftp.fr.debian.org/debian/ testing/main
   apache2 i386 2.2.16-4 [1384 B]
   Des:8 http://ftp.fr.debian.org/debian/ testing/main
   libapache2-mod-wsgi i386 3.3-1 [129 kB]
   Des:9 http://ftp.fr.debian.org/debian/ testing/main
   ssl-cert all 1.0.28 [14,8 kB]
   Descargados 2016 kB en 11s (172 kB/s).
   Preconfigurando paquetes ...
   Seleccionando el paquete libaprutil1-dbd-sqlite3
   previamente no seleccionado.
   (Leyendo la base de datos ... 200625 files and directories
   currently installed.)
   Desempaquetando libaprutil1-dbd-sqlite3
   (de .../libaprutil1-dbd-sqlite3_1.3.9+dfsg-5_i386.deb) ...
   Seleccionando el paquete libaprutil1-ldap previamente no
   seleccionado.
   Desempaquetando libaprutil1-ldap
   (de .../libaprutil1-ldap_1.3.9+dfsg-5_i386.deb) ...
   Seleccionando el paquete apache2.2-bin previamente no
   seleccionado.
   Desempaquetando apache2.2-bin
   (de .../apache2.2-bin_2.2.16-4_i386.deb) ...
   Seleccionando el paquete apache2-utils previamente no
   seleccionado.
   Desempaquetando apache2-utils
   (de .../apache2-utils_2.2.16-4_i386.deb) ...
   Seleccionando el paquete apache2.2-common previamente no
   seleccionado.
   Desempaquetando apache2.2-common
   (de .../apache2.2-common_2.2.16-4_i386.deb) ...
   Seleccionando el paquete apache2-mpm-worker previamente no
   seleccionado.
   Desempaquetando apache2-mpm-worker
   (de .../apache2-mpm-worker_2.2.16-4_i386.deb) ...
   Seleccionando el paquete apache2 previamente no
   seleccionado.
   Desempaquetando apache2
   (de .../apache2_2.2.16-4_i386.deb) ...
   Seleccionando el paquete libapache2-mod-wsgi previamente no
   seleccionado.
   Desempaquetando libapache2-mod-wsgi
   (de .../libapache2-mod-wsgi_3.3-1_i386.deb) ...
   Seleccionando el paquete ssl-cert previamente no
   seleccionado.
   Desempaquetando ssl-cert
   (de .../ssl-cert_1.0.28_all.deb) ...
   Procesando disparadores para man-db ...
   Configurando libaprutil1-dbd-sqlite3 (1.3.9+dfsg-5) ...
   Configurando libaprutil1-ldap (1.3.9+dfsg-5) ...
   Configurando apache2.2-bin (2.2.16-4) ...
   Configurando apache2-utils (2.2.16-4) ...
   Configurando apache2.2-common (2.2.16-4) ...
   Configurando apache2-mpm-worker (2.2.16-4) ...
   Starting web server: apache2.
   Configurando apache2 (2.2.16-4) ...
   Configurando libapache2-mod-wsgi (3.3-1) ...
   Restarting web server: apache2 ... waiting .
   Configurando ssl-cert (1.0.28) ...
   \end{verbatim}

   De esta forma, la aplicación queda correctamente instalada. Además, el
   servidor web se ha iniciado automáticamente. En caso de tener algún problema
   con el inicio del servidor web, se ha de indicar que el comando para
   iniciarlo es:

   \begin{verbatim}
   # /etc/init.d/apache2 start
   \end{verbatim}

   En nuestro caso, como se había iniciado con normalidad, el comando devuelve
   el mensaje:

   \begin{verbatim}
   Starting web server: apache2httpd (pid 13782) already running.
   \end{verbatim}


