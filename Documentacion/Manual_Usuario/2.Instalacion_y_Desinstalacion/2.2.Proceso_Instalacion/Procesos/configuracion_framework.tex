\item \textbf{Configuración del \textit{framework}}.
   Llegados a este punto es necesario configurar el \textit{framework}
   utilizado, \textit{Django}, para que le sea posible conectarse a la base de
   datos que acabamos de configurar. Para ello, debemos editar el archivo
   \textit{settings.py} que se encuentra en la carpeta \textit{proyecto} del
   código fuente que proporciona esta aplicación. Para ello:

   \begin{verbatim}
   $ nano /ruta-proyecto/proyecto/settings.py
   \end{verbatim}

   Una vez dentro, debemos localizar las líneas:

   \begin{verbatim}
   DATABASE_USER = ''
   DATABASE_PASSWORD = ''
   \end{verbatim}

   Se debe introducir el nombre de usuario PostgreSQL válido y la contraseña
   que utilizamos para tener acceso de lectura y escritura para nuestra base
   de datos \textit{asesorías}. Con esto estaría completa la configuración
   del \textit{framework Django}.
