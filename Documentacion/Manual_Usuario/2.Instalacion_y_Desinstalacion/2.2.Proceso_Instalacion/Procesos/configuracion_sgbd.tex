\item \textbf{Configuración del sistema gestor de bases de datos}.
   Se procede a configurar el sistema gestor de bases de datos utilizado,
   \textit{PostgreSQL}.Para llevar a cabo la configuración, debemos editar
   el archivo \textit{pg\_hba.conf} que se encuentra situado en
   \textit{/etc/postgresql/8.4/main/}. Por lo tanto:

   \begin{verbatim}
   # nano /etc/postgresql/8.4/main/pg_hba.conf
   \end{verbatim}

   Una vez dentro, debemos añadir al final del fichero la siguiente línea:

   \begin{verbatim}
   host  asesorias   [usuario]   127.0.0.1/32   [método]
   \end{verbatim}

   Nótese que hay que sustituir \textit{[usuario]} por el nombre de usuario del
   sistema que queremos que administre nuestra aplicación. También será
   necesario sustituir \textit{[método]} por el tipo de método de cifrado de la
   contraseña del usuario de PostgreSQL (normalmente md5). Si tiene problemas
   creando usuarios en PostgreSQL con permisos para leer y escribir bases de
   datos, por favor, diríjase a la documentación oficial de PostgreSQL
   \cite{postgresql}.