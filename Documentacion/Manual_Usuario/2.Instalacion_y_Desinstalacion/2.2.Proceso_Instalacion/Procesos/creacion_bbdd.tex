\item \textbf{Creación y sincronización de la base de datos}.
   A continuación procederemos a crear la base de datos que utilizará el sistema
   para almacenar toda la información que necesite. Para ello, basta con que
   ejecutemos el siguiente comando con el usuario del sistema con permisos en
   \textit{PostgreSQL} de lectura y escritura para manejar la tabla
   \textit{asesorias}, no importando la ruta donde nos encontremos:

   \begin{verbatim}
   $ createdb asesorias
   \end{verbatim}

   Si no hay ningún error, el \textit{prompt} volverá a su estado inicial,
   sin mensajes.

   Para sincronizarla y, por lo tanto, crear todas las tablas necesarias para
   ejecutar nuestra aplicación, debemos situarnos en el directorio
   \textit{/proyecto/} del código fuente proporcionado por esta aplicación
   y ejecutar la siguiente orden:

   \begin{verbatim}
   # python ./manage.py syncdb
   \end{verbatim}

   Si no existen problemas, debería aparecer la siguiente información:

   \begin{verbatim}
   Creating table auth_permission
   Creating table auth_group_permissions
   Creating table auth_group
   Creating table auth_user_user_permissions
   Creating table auth_user_groups
   Creating table auth_user
   Creating table auth_message
   Creating table django_content_type
   Creating table django_session
   Creating table django_site
   Creating table Centros
   Creating table AdministradoresCentro
   Creating table Titulaciones
   Creating table Asignaturas
   Creating table AsignaturasCursoAcademico
   Creating table Departamentos
   Creating table Asesores
   Creating table AsesoresCursoAcademico
   Creating table PlantillasEntrevistaAsesor
   Creating table PreguntasAsesores
   Creating table Alumnos
   Creating table AlumnosCursoAcademico
   Creating table Matriculas
   Creating table CalificacionesConvocatoria
   Creating table PlantillasEntrevistaOficial
   Creating table PreguntasOficiales
   Creating table Reuniones
   Creating table CentroAdministradorCentro
   Creating table ReunionPreguntaAsesor
   Creating table ReunionPreguntaOficial

   You just installed Django's auth system, which means
   you don't have any superusers defined.
   Would you like to create one now? (yes/no):
   \end{verbatim}

   Llegados a este punto, nos pide crear un usuario administrador principal, por
   lo que aceptamos (\textit{yes}):

   \begin{verbatim}
   Username (Leave blank to use '[usuario]'):
   E-mail address:
   Password:
   Password (again):
   \end{verbatim}

   Nos pide el nombre de usuario, correo electrónico y contraseña.
   Introducimos lo que mejor nos convenga. Por último aparecerá:

   \begin{verbatim}
   Superuser created successfully.
   Installing index for auth.Permission model
   Installing index for auth.Group_permissions model
   Installing index for auth.User_user_permissions model
   Installing index for auth.User_groups model
   Installing index for auth.Message model
   Installing index for asesorias.Titulacion model
   Installing index for asesorias.AsesorCursoAcademico model
   Installing index for asesorias.AlumnoCursoAcademico model
   Installing index for asesorias.PreguntaOficial model
   Installing index for asesorias.CentroAdministradorCentro model
   Installing json fixture 'initial_data' from absolute path.
   Installed 87 object(s) from 1 fixture(s)
   \end{verbatim}

   Con esto la creación y sincronización de la base de datos ha quedado
   completada satisfactoriamente.