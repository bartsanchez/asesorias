\paragraph{}Como punto y final a la realización de este Proyecto de Fin de
Carrera se va a proceder a describir cuáles han sido las metas alcanzadas en el
desarrollo del mismo, basándose en los objetivos descritos en el capítulo
\ref{objetivos}, \textit{Objetivos}.

\paragraph{}La principal conclusión es que se ha desarrollado una aplicación
 informática que permite la gestión de toda la información relacionada con las
 asesorías académicas en la Universidad de Córdoba.

\paragraph{}Además, hay que destacar los siguientes aspectos secundarios:

\begin{itemize}
 \item Se trata de una aplicación web multiplataforma y multiusuario.
 \item La aplicación distingue entre distintos tipos usuarios,
 \textit{Administrador principal}, \textit{Administrador de centro},
 \textit{Asesor} y \textit{Alumno}, cada uno de los cuales realiza una labor
 específica en la aplicación.
 \item Dicha aplicación está adaptada al reglamento de la Universidad de
 Córdoba en lo concerniente a la Asesoría Académica.
 \item Entre la información que se permite gestionar se encuentra toda la
 estructura organizativa docente de la Universidad de Córdoba, teniendo en
 cuenta los diferentes centros, con sus respectivas titulaciones y asignaturas.
 También permite la realización de entrevistas por parte de los asesores a sus
 respectivos alumnos, de los cuales tendrán información relevante para tal fin
 como puede ser su historial o los datos personales necesarios para realizar la
 labor de asesoría.
 \item El sistema permite la generación de informes y la realización de
 consultas que pueden ser de gran utilidad a la hora de realizar la labor
 de asesoría.
\end{itemize}

\paragraph{}En cuanto a los aspectos técnicos que presenta la aplicación cabe
destacar:

\begin{itemize}
 \item Se ha realizado el diseño de una base de datos relacional, que será la
 encargada de almacenar toda la información con la que se trabajará. Para llevar
 a cabo la implementación de esta base de datos diseñada, se ha utilizado el
 sistema gestor de bases de datos denominado PostgreSQL \cite{postgresql}, con
 magníficos resultados.
 \item El hecho de utilizar el lenguaje de programación Python \cite{python}, ha
 supuesto una gran ventaja a la hora de codificar la aplicación, debido a su
 simpleza unida a una gran robustez. El uso de Python conlleva, además,
 una mejor legibilidad del código generado, debido a la propia naturaleza del
 lenguaje.
 \item Al hacer uso del \textit{framework} Django, toda la información de la
 aplicación queda muy bien controlada y organizada. Si a esto le sumamos el uso
 de convenciones tenidas en cuenta, inherentes al propio Django, se facilita la
 posible futura edición para solventar algún probable error, o simplemente
 mejorar el sistema añadiendo nuevas funcionalidades, como las citadas en el
 capítulo \ref{futurasMejoras}, \textit{Futuras mejoras}. Además el tiempo de
 desarrollo de la aplicación resultante ha sido bastante bajo, teniendo en
 cuenta la envergadura del proyecto que se ha llevado a cabo.
\end{itemize}

\paragraph{}Al tener una interfaz web, la utilización de la aplicación se
simplifica bastante, ya que hoy por hoy el conocimiento y la utilización de
aplicaciones web están muy extendidos entre la mayoría de la población.

\paragraph{}También, se facilita la introducción de datos por parte del usuario,
se muestran los posibles valores que puede tomar un determinado campo, se
desactivan las órdenes que son inapropiadas en el contexto de otras acciones, se
proporcionan mensajes de error, se pide confirmación de cualquier acción
destructiva importante, etc., todo ello con el motivo de facilitar al usuario
final de la aplicación su navegación por el sistema software desarrollado.

\paragraph{}De esta forma:

\begin{itemize}
 \item El manejo de la aplicación resulta fácil para el usuario, ya que éste,
 para manejar el programa, no tiene por qué tener conocimientos previos de
 informática.
 \item Para orientar en todo momento al usuario se han diseñado varios tipos de
 ayuda que estarán disponibles en cualquier momento durante el transcurso de la
 ejecución del programa.
\end{itemize}

\paragraph{}Por otro lado, se ha elaborado un manual de usuario que describe
paso a paso tanto el proceso de instalación como la navegación a través de
pantallas que constituyen la aplicación. También incorpora unos ejemplos que
ilustran el manejo del programa en muchos de sus apartados, de tal forma que el
usuario podrá encontrar en ellos todo lo necesario para conocer y explotar las
posibilidades del sistema software implementado.

\paragraph{}En resumen, se considera que el trabajo realizado constituye una
aportación importante para mejorar la labor de la Asesoría Académica en la
Universidad de Córdoba.
