\item Prueba de privacidad.
  \begin{itemize}
    \item \textbf{Descripción:} Se intenta acceder a zonas del sistema donde el
    usuario identificado no tiene permiso de acceso.
    \item \textbf{Problemas encontrados:} Modificando manualmente la url del
    navegador es posible acceder a zonas del sistema no autorizadas para el
    usuario identificado. Ejemplo: un usuario Alumno que conozca la url a través
    de la cual un usuario Administrador principal accede a su zona de la
    aplicación puede ver su contenido.
    \item \textbf{Soluciones adoptadas:} Se han creado funciones
    decoradoras\footnote{Según la Wikipedia \cite{wikipedia1} : \textit{A
    decorator is a Python object that can be called with a single argument, and
    that modifies functions or methods. Decorators are a form of
    metaprogramming; they enhance the action of the function or method they
    decorate. Canonical uses of function decorators are for creating class
    methods or static methods, adding function attributes, tracing, setting pre-
    and postconditions, and synchronisation, but can be used for far more
    besides, including tail recursion elimination, memoization and even
    improving the writing of decorators.}}
    para cada tipo de rol que comprueban que se realice un acceso autorizado
    para cada unas de las funciones de la aplicación.
  \end{itemize}
