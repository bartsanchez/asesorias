\section{Introducción}

   \paragraph{}El modelo de datos relacional se obtiene a partir del diagrama
   obtenido en el capítulo \ref{modEntInt}, \textit{Modelo
   Entidad-Interrelación}.

   \paragraph{}El modelo de datos relacional representa la información mediante
   el uso de una representación tabular plana de la misma, que es una tabla
   bidimensional mediante la cual se representa tanto los objetos como las
   relaciones entre ellos existentes en el dominio del problema. Como cualquier
   modelo de datos, el modelo relacional introduce su propia terminología para
   nominar los objetos y elementos utilizados por el modelo para representar el
   dominio de la información. Por ejemplo, a una tabla o matriz rectangular se
   le denomina relación, a las filas de la misma se les denomina tuplas y al
   conjunto de sus columnas, dominio de la relación. Así, una base de datos
   relacional estará formada por un conjunto de relaciones.

   \paragraph{}Las ventajas de este modelo son:

   \begin{itemize}
    \item Genera esquemas que representan fielmente la información, los objetos
    y relaciones entre ellos existentes en el dominio del problema.
    \item La información contenida puede ser entendida fácilmente por los
    usuarios que no tienen una preparación previa en este área.
    \item Hace posible ampliar el esquema de la base de datos sin modificar la
    estructura lógica existente y, por tanto, sin modificar los programas de
    aplicación.
    \item Garantiza herramientas para evitar la duplicidad de registros, a
    través de campos claves o llaves.
    \item Garantiza la integridad referencial: Así, al eliminar un registro,
    se eliminan todos los registros relacionados dependientes.
    \item Favorece la normalización por ser más comprensible y aplicable.
   \end{itemize}

   \paragraph{}El proceso de traducción de esquemas conceptuales a lógicos
   consiste en la aplicación, por pasos, de una serie de reglas\footnote{Tanto
   para la definición como para la aplicación de dichas reglas se ha tenido
   como referencia \textit{Bases de datos: Desde Chen hasta Codd con ORACLE}, de
   Luque Ruiz et al. \cite{luqueRuiz}. Dichas reglas son conocidas como RTECAR;
   es decir, reglas
   de transformación de esquemas conceptuales a esquemas relacionales, no sin
   antes tener en cuenta unas reglas preparatorias llamadas PRTECAR, que
   faciliten y garanticen la fiabilidad del proceso de transformación.} que,
   aplicadas a los esquemas conceptuales, transforman los
   objetos de estos esquemas en objetos pertenecientes a los esquemas lógicos.
   La aplicación de las reglas va a dar lugar a la transformación de los tipos
   de entidad y los tipos de interrelación que forman parte de los esquemas
   conceptuales, en tablas o relaciones, los únicos objetos que intervienen en
   los esquemas lógicos relacionales. Para cada una de las tablas obtenidas, se
   va a proporcionar la siguiente información:

   \begin{itemize}
    \item Nombre.
    \item Tipo de entidad o tipo de interrelación del que proviene.
    \item Atributos que la componen de los cuales podemos distinguir entre:
    \begin{itemize}
     \item \underline{Clave primaria}
     \item \textsc{Clave alterna}
     \item \textbf{Clave foránea}
     \item Resto de atributos
    \end{itemize}
    \item Cualquier otro tipo de información que se considere de interés
          incluir.
   \end{itemize}

  \paragraph{}A continuación se van a detallar las tablas obtenidas tras la
  aplicación de las reglas para transformar el modelo conceptual a modelo
  relacional.
