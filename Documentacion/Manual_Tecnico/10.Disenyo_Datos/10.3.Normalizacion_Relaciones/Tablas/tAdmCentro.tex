\subsection{Tabla AdministradoresCentro}

  \paragraph{}Esta tabla se encuentra en FNBC, ya que cada determinante
  funcional es una clave candidata de la relación.

  \paragraph{}Existen dos dependencias funcionales, la que forma la clave
  primaria de la relación con el resto de atributos y, por otro lado, la que
  forma la clave alterna con el resto de atributos de la relación.

  \begin{center}
    \begin{minipage}{3.7cm}{\begin{flushright}\begin{tabular}{ | c | }
                  \hline
                  id\_adm\_centro \\
                  \hline
                 \end{tabular}\end{flushright} }
    \end{minipage}
    \begin{minipage}{0.7cm}{$\longrightarrow$}
    \end{minipage}
    \begin{minipage}{5.9cm}{\begin{tabular}{ | c | }
                  \hline
                  correo\_electrónico \\
                  nombre\_adm\_centro \\
                  \hline
                 \end{tabular} }
    \end{minipage}
  \end{center}

  \begin{center}
    \begin{minipage}{3.7cm}{\begin{flushright}\begin{tabular}{ | c | }
                  \hline
                  correo\_electrónico \\
                  \hline
                 \end{tabular}\end{flushright} }
    \end{minipage}
    \begin{minipage}{0.7cm}{$\longrightarrow$}
    \end{minipage}
    \begin{minipage}{5.9cm}{\begin{tabular}{ | c | }
                  \hline
                  id\_adm\_centro \\
                  nombre\_adm\_centro \\
                  \hline
                 \end{tabular} }
    \end{minipage}
  \end{center}