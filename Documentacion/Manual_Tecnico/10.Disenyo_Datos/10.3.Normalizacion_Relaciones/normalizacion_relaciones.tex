\section{Normalización de relaciones}

  \paragraph{}El modelo relacional está soportado sobre una teoría de igual
  nombre basada en los principios de la teoría general de conjuntos. El
  proceso de normalización debe aplicarse independientemente del
  sistema de gestión de base de datos utilizado.

  \paragraph{}Este proceso elimina redundancias superfluas, aminorando de
  esta forma el espacio físico requerido para el almacenamiento de la
  información, por lo que se reducen los posibles problemas de
  integridad de la información almacenada en la base de datos.

  \paragraph{}Aumenta el desempeño, tanto de las operaciones de
  actualización de la información almacenada en la base de datos,
  como de las interrogaciones sobre la información almacenada en la
  misma.

  \paragraph{}Representa de forma coherente los objetos y relaciones
  presentes en el dominio del problema, y cuya información es
  almacenada en la base de datos.

  \paragraph{}El proceso de normalización va a ser aplicado a cada una de
  las tablas obtenidas mediante el modelo relacional para lograr un
  modelo más óptimo que cumpla con las características anteriormente
  descritas.

  \paragraph{}Todas las tablas obtenidas en el modelo relacional satisfacen
  la primera forman normal, (FN1\footnote{FN1: \textit{Una relación R satisface
  la primera forma normal si, y sólo si, todos los dominios subyacentes de la
  relación R contienen valores atómicos}.}) ya que no existe en ninguna tabla
  atributos múltiples.

  \paragraph{}El que una relación se encuentre en FN1 es condición
  indispensable aunque no suficiente para garantizar la consistencia del
  modelo relacional.

  \paragraph{}El proceso de normalización que se va a aplicar a continuación
  hará uso de los diagramas de dependencias funcionales que permiten
  representar las dependencias existentes entre los atributos de las
  tablas.

  \paragraph{}El proceso de normalización concluirá cuando todas las tablas
  obtenidas mediante el modelo relacional satisfagan la forma normal
  de Boyce-Codd (FNBC\footnote{FNBC: \textit{Una relación R satisface la forma
  normal de Boyce-Codd si, y sólo si, se encuentra en FN1, y cada determinante
  funcional es una clave candidata de la relación R}.}), ya que en este momento
  se considera que el esquema relacional es lo suficientemente consistente.

  \paragraph{}A continuación se mostrarán cada una de las tablas obtenidas
  tras la aplicación de la FNBC.


