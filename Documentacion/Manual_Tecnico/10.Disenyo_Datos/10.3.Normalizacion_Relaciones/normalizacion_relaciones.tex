\section{Normalización de relaciones}

  \paragraph{}El modelo relacional está soportado sobre una teoría de igual
  nombre basada en los principios de la teoría general de conjuntos. El
  proceso de normalización debe aplicarse independientemente del
  sistema de gestión de base de datos utilizado.

  \paragraph{}Este proceso elimina redundancias superfluas, aminorando de
  esta forma el espacio físico requerido para el almacenamiento de la
  información, por lo que se reducen los posibles problemas de
  integridad de la información almacenada en la base de datos.

  \paragraph{}Aumenta el desempeño, tanto de las operaciones de
  actualización de la información almacenada en la base de datos,
  como de las interrogaciones sobre la información almacenada en la
  misma.

  \paragraph{}Representa de forma coherente los objetos y relaciones
  presentes en el dominio del problema, y cuya información es
  almacenada en la base de datos.

  \paragraph{}El proceso de normalización va a ser aplicado a cada una de
  las tablas obtenidas mediante el modelo relacional para lograr un
  modelo más óptimo que cumpla con las características anteriormente
  descritas.

  \paragraph{}Todas las tablas obtenidas en el modelo relacional satisfacen
  la primera forman normal, (FN1\footnote{FN1: \textit{Una relación R satisface
  la primera forma normal si, y sólo si, todos los dominios subyacentes de la
  relación R contienen valores atómicos}.}) ya que no existe en ninguna tabla
  atributos múltiples.

  \paragraph{}El que una relación se encuentre en FN1 es condición
  indispensable aunque no suficiente para garantizar la consistencia del
  modelo relacional.

  \paragraph{}El proceso de normalización que se va a aplicar a continuación
  hará uso de los diagramas de dependencias funcionales que permiten
  representar las dependencias existentes entre los atributos de las
  tablas.

  \paragraph{}El proceso de normalización concluirá cuando todas las tablas
  obtenidas mediante el modelo relacional satisfagan la forma normal
  de Boyce-Codd (FNBC\footnote{FNBC: \textit{Una relación R satisface la forma
  normal de Boyce-Codd si, y sólo si, se encuentra en FN1, y cada determinante
  funcional es una clave candidata de la relación R}.}), ya que en este momento
  se considera que el esquema relacional es lo suficientemente consistente.

  \paragraph{}A continuación se mostrarán cada una de las tablas obtenidas
  tras la aplicación de la FNBC.

   \subsection{Tabla Alumnos}

      \paragraph{}Esta tabla se obtiene a través del tipo de entidad
      \textit{Alumno}, tomando los atributos de ésta (regla
      RTECAR-1\footnote{Regla RTECAR-1: \textit{``Todos los tipos de entidad
      presentes en el esquema conceptual se transformarán en tablas o relaciones
      en el esquema relacional manteniendo el número y tipo de atributos, así
      como la característica de identificador de estos atributos.''}}).

      \paragraph{}La clave principal de esta tabla es el atributo
      \textit{dni\_pasaporte}. Posee una clave alterna que será la formada por
      el atributo \textit{correo\_electrónico}.

      \paragraph{}La tabla \textit{Alumnos} queda de la siguiente forma:

      \begin{description}
         \item[Alumnos] \begin{flushleft}(\underline{dni\_pasaporte},
         \textsc{correo\_electrónico}, nombre, apellidos, fecha\_nacimiento,
         dirección\_córdoba, teléfono, dirección\_familiar, localidad\_familiar,
         provincia\_familiar, código\_postal, teléfono\_familiar, ingreso,
         otros\_estudios\_universitarios, modalidad\_acceso\_universidad,
         calificación\_acceso)\end{flushleft}
      \end{description}

   \subsection{Tabla Asesores}

      \paragraph{}Esta tabla se obtiene a través del tipo de entidad
      \textit{Asesor}, tomando los atributos de ésta (regla RTECAR-1).

      \paragraph{}La clave principal de esta tabla es el atributo
      \textit{dni\_pasaporte}. Posee una clave alterna que será la formada por
      el atributo \textit{correo\_electrónico}.

      \paragraph{}La tabla \textit{Asesor} queda de la siguiente forma:

      \begin{description}
         \item[Asesores] \begin{flushleft}(\underline{dni\_pasaporte},
         \textsc{correo\_electrónico}, nombre, apellidos)\end{flushleft}
      \end{description}

   \subsection{Tabla Centros}

      \paragraph{}Esta tabla se obtiene a través del tipo de entidad
      \textit{Centro}, tomando los atributos de ésta (regla RTECAR-1\footnote{Regla
      RTECAR-1: \textit{``Todos los tipos de entidad
      presentes en el esquema conceptual se transformarán en tablas o relaciones
      en el esquema relacional manteniendo el número y tipo de atributos, así
      como la característica de identificador de estos atributos.''}}).

      \paragraph{}La clave principal de esta tabla es el atributo
      \textit{id\_centro}.

      \paragraph{}La tabla \textit{Centros} queda de la siguiente forma:

      \begin{description}
         \item[Centros] \begin{flushleft}(\underline{id\_centro},
         \textsc{nombre\_centro})\end{flushleft}
      \end{description}
\subsection{Tabla AdministradoresCentro}

\begin{verbatim}
  CREATE TABLE AdministradoresCentro(
  id_adm_centro      int(2)      NOT NULL auto_increment,
  correo_electrónico varchar(50) NOT NULL,
  nombre_adm_centro  varchar(25) NOT NULL,
  PRIMARY KEY pk_adminCentro (id_adm_centro),
  UNIQUE uk_adminCentro (correo_electrónico)
  );
\end{verbatim}
   \subsection{Tabla Departamentos}

      \paragraph{}Esta tabla se obtiene a través del tipo de entidad
      \textit{Departamento}, tomando los atributos de ésta (regla RTECAR-1).

      \paragraph{}La clave principal de esta tabla es el atributo
      \textit{id\_departamento}. Posee una clave alterna que será la formada por
      el atributo \textit{nombre\_departamento}.

      \paragraph{}La tabla \textit{Departamentos} queda de la siguiente forma:

      \begin{description}
         \item[Departamentos] \begin{flushleft}(\underline{id\_departamento},
         \textsc{nombre\_departamento}, teléfono)\end{flushleft}
      \end{description}

\subsection{Tabla Titulaciones}

\begin{verbatim}
  CREATE TABLE Titulaciones(
  id_centro          int(2)         NOT NULL,
  id_titulación      int(2)         NOT NULL AUTO_INCREMENT,
  nombre_titulación  varchar(100)   NOT NULL,
  plan_estudios      int(4)         NOT NULL,
  PRIMARY KEY pk_titulaciones (id_centro, id_titulación),
  UNIQUE uk_titulaciones (nombre_titulación, plan_estudios)
  );
\end{verbatim}
\subsection{Tabla Asignaturas}

\begin{verbatim}
  CREATE TABLE Asignaturas(
  id_centro          int(2)         NOT NULL,
  id_titulación      int(2)         NOT NULL,
  id_asignatura      int(2)         NOT NULL AUTO_INCREMENT,
  nombre_asignatura  varchar(50)    NOT NULL,
  curso              int(2),
  tipo               enum('Troncal',
                     'Obligatoria',
                     'Optativa',
                     'Libre Configuración')
                                    NOT NULL,
  nCréditosTeóricos  float(5,2)     NOT NULL,
  nCréditosPrácticos float(5,2)     NOT NULL,
  PRIMARY KEY pk_asignaturas (id_centro, id_titulación, id_asignatura),
  FOREIGN KEY asignatura_fk_titulacion (id_centro, id_titulación)
  REFERENCES Titulaciones (id_centro, id_titulación)
  ON DELETE CASCADE ON UPDATE CASCADE
  );
\end{verbatim}
% \subsection{Tabla AsignaturasCursoAcadémico}

  \paragraph{}Esta tabla se encuentra en FNBC puesto que el único
  determinante funcional existente es el propio identificador principal
  de la tabla.

 \begin{center}
    \begin{minipage}{4.2cm}{\begin{flushright}\begin{tabular}{ | c | }
                  \hline
                  (id\_centro + \\
                  id\_titulación + \\
                  id\_asignatura + \\
                  curso\_académico) \\
                  \hline
                 \end{tabular}\end{flushright} }
    \end{minipage}
    \begin{minipage}{0.7cm}{$\longrightarrow$}
    \end{minipage}
    \begin{minipage}{5.9cm}{\begin{tabular}{ | c | }
                  \hline
                  \\
                  \hline
                 \end{tabular} }
    \end{minipage}
  \end{center}

%    \subsection{Tabla AlumnosCursoAcadémico}

      \paragraph{}Esta tabla se obtiene a través del tipo de entidad
      \textit{Alumno Curso Académico}, tomando los atributos de ésta (regla
      RTECAR-1). Además, mantiene una referencia con la tabla \textit{Alumno} a
      través del atributo \textit{dni\_pasaporte} y con la
      tabla \textit{Asesor Curso Académico} a través de los atributos
      \textit{dni\_pasaporte} y \textit{curso\_académico} (regla RTECAR-3.1).
      Nótese que para diferenciar los atributos \textit{dni\_pasaporte} de las
      entidades \textit{Alumno Curso Académico} y \textit{Asesor Curso
      Académico} se renombrarán a \textit{dni\_pasaporte\_alumno} y
      \textit{dni\_pasaporte\_asesor}, respectivamente. Además, es necesario
      indicar que el atributo \textit{curso\_académico} que se hereda de la
      entidad \textit{Asesor Curso Académico} no se representa en esta tabla.
      Esto es debido a que dicho atributo debe coincidir con el atributo del
      mismo nombre ya existente en esta tabla, por la propia naturaleza de la
      interrelación.

      \paragraph{}La clave principal de la tabla está compuesta por la
      agregación de los atributos \textit{dni\_pasaporte} y
      \textit{curso\_académico}.

      \paragraph{}La tabla \textit{AlumnosCursoAcadémico} queda de la siguiente
      forma:

      \begin{description}
         \item[AlumnosCursoAcadémico] \begin{flushleft}(\underline{
         \textbf{dni\_pasaporte\_alumno}, curso\_académico}, observaciones,
         \textbf{dni\_pasaporte\_asesor})\end{flushleft}
      \end{description}

% \subsection{Tabla AsesoresCursoAcadémico}

\begin{verbatim}
  CREATE TABLE AsesoresCursoAcadémico(
  dni_pasaporte   varchar(9)     NOT NULL,
  curso_académico int(4)         NOT NULL,
  id_departamento int(2)         NOT NULL,
  observaciones   varchar(100),
  PRIMARY KEY pk_asesoresCA (dni_pasaporte, curso_académico),
  FOREIGN KEY asesoresCA_fk_asesores (dni_pasaporte)
  REFERENCES Asesores (dni_pasaporte)
  ON DELETE CASCADE ON UPDATE CASCADE,
  FOREIGN KEY asesoresCA_fk_departamentos (id_departamento)
  REFERENCES Departamentos (id_departamento)
  ON DELETE CASCADE ON UPDATE CASCADE
  );
\end{verbatim}
%    \subsection{Tabla Reuniones}

      \paragraph{}Esta tabla se obtiene a través del tipo de entidad
      \textit{Reunión}, tomando los atributos de ésta (regla RTECAR-1).
      Además, mantiene una referencia con la tabla \textit{Alumno Curso
      Académico} a través de los atributos \textit{dni\_pasaporte} y
      \textit{curso\_académico} (regla RTECAR-3.1).

      \paragraph{}La clave principal de esta tabla se compone con la agregación
      de los atributos \textit{dni\_pasaporte}, \textit{curso\_académico} e
      \textit{id\_reunión}.

      \paragraph{}La tabla \textit{Reuniones} queda de la siguiente forma:

      \begin{description}
         \item[Reuniones] \begin{flushleft}(\underline{\textbf{dni\_pasaporte},
         \textbf{curso\_académico}, id\_reunión}, fecha, tipo,
         comentario\_asesor, comentario\_alumno)\end{flushleft}
      \end{description}

%    \subsection{Tabla PlantillasEntrevistaOficial}

      \paragraph{}Esta tabla se obtiene a través del tipo de entidad
      \textit{Plantilla Entrevista Oficial}, tomando los atributos de ésta
      (regla RTECAR-1).

      \paragraph{}La clave principal de esta tabla es el atributo
      \textit{id\_entrevista\_oficial}.

      \paragraph{}La tabla \textit{PlantillasEntrevistaOficial} queda de la
      siguiente forma:

      \begin{description}
         \item[PlantillasEntrevistaOficial] \begin{flushleft}(\underline{id\_entrevista\_oficial}, última\_modificación)\end{flushleft}
      \end{description}

% \subsection{Tabla PlantillasEntrevistaAsesor}

\begin{verbatim}
  CREATE TABLE PlantillasEntrevistaAsesor(
  dni_pasaporte         varchar(9)     NOT NULL,
  curso_académico       int(4)         NOT NULL,
  id_entrevista_asesor  int(3)         NOT NULL AUTO_INCREMENT,
  descripción           varchar(100),
  última_modificación   date           NOT NULL,
  PRIMARY KEY pk_plantEntAse (dni_pasaporte, curso_académico,
                              id_entrevista_asesor),
  FOREIGN KEY plantEntAse_fk_asesoresCA (dni_pasaporte,
                                         curso_académico)
  REFERENCES AsesoresCursoAcadémico (dni_pasaporte, curso_académico)
  ON DELETE CASCADE ON UPDATE CASCADE
  );
\end{verbatim}
%    \subsection{Tabla PreguntasOficiales}

      \paragraph{}Esta tabla se obtiene a través del tipo de entidad
      \textit{Pregunta Oficial}, tomando los atributos de ésta
      (regla RTECAR-1). Además, mantiene una referencia con la tabla
      \textit{Plantilla Entrevista Oficial} a través del atributo
      \textit{id\_entrevista\_oficial} (regla RTECAR-3.1).

      \paragraph{}La clave principal de esta tabla se compone con la agregación
      de los atributos \textit{id\_entrevista\_oficial} e
      \textit{id\_pregunta\_oficial}.

      \paragraph{}La tabla \textit{PreguntasOficiales} queda de la
      siguiente forma:

      \begin{description}
         \item[PreguntasOficiales] \begin{flushleft}(\underline{\textbf{id\_entrevista\_oficial},
         id\_pregunta\_oficial}, enunciado, última\_modificación)\end{flushleft}
      \end{description}

%    \subsection{Tabla PreguntasAsesores}

      \paragraph{}Esta tabla se obtiene a través del tipo de entidad
      \textit{Pregunta Asesor}, tomando los atributos de ésta
      (regla RTECAR-1). Además, mantiene una referencia con la tabla
      \textit{Plantilla Entrevista Asesor} a través de los atributos
      \textit{dni\_pasaporte}, \textit{curso\_académico} e
      \textit{id\_entrevista\_asesor} (regla RTECAR-3.1).

      \paragraph{}La clave principal de esta tabla se compone con la agregación
      de los atributos \textit{dni\_pasaporte}, \textit{curso\_académico},
      \textit{id\_entrevista\_asesor} e \textit{id\_pregunta\_asesor}.

      \paragraph{}La tabla \textit{PreguntasAsesores} queda de la
      siguiente forma:

      \begin{description}
         \item[PreguntasAsesores] \begin{flushleft}(\underline{\textbf{dni\_pasaporte},
         \textbf{curso\_académico}, \textbf{id\_entrevista\_asesor},}
         \underline{id\_pregunta\_asesor}, enunciado, última\_modificación)\end{flushleft}
      \end{description}

% \subsection{Tabla Centro\_AdministradoresCentro}

  \paragraph{}Esta tabla se encuentra en FNBC puesto que el único
  determinante funcional existente es el propio identificador principal
  de la tabla.

 \begin{center}
    \begin{minipage}{4.2cm}{\begin{flushright}\begin{tabular}{ | c | }
                  \hline
                  (id\_centro + \\
                  id\_adm\_centro) \\
                  \hline
                 \end{tabular}\end{flushright} }
    \end{minipage}
    \begin{minipage}{0.7cm}{$\longrightarrow$}
    \end{minipage}
    \begin{minipage}{5.9cm}{\begin{tabular}{ | c | }
                  \hline
                  \\
                  \hline
                 \end{tabular} }
    \end{minipage}
  \end{center}

%    \subsection{Tabla Asignatura\_Alumnos\_CursoAcadémico}

      \paragraph{}Esta tabla surge de la interrelación ACA-AlCA, existente entre
      los tipos de entidad \textit{Asignatura Curso Académico} y \textit{Alumno
      Curso Académico} (regla RTECAR-4), tomando los atributos de
      ésta. Gracias a esta interrelación, se podrá conocer los diferentes
      alumnos matriculados en una determinada asignatura durante un curso
      académico.

      \paragraph{}La clave principal de esta tabla la forman la agregación de
      los atributos \textit{id\_centro}, \textit{id\_titulación},
      \textit{id\_asignatura}, \textit{curso\_académico},
      \textit{dni\_pasaporte} y \textit{convocatoria}. Estos atributos son a su
      vez claves foráneas excepto \textit{convocatoria}, el cual es un atributo
      identificador de la interrelación y no de otra entidad.

      \paragraph{}La tabla
      \textit{Asignatura\_Alumnos\_CursoAcadémico} queda, por tanto,
      de la siguiente forma:

      \begin{description}
         \item[Asignatura\_Alumnos\_CursoAcadémico] \begin{flushleft}(\underline{\textbf{id\_centro}, \textbf{id\_titulación},}
         \underline{\textbf{id\_asignatura}, \textbf{curso\_académico},
         \textbf{dni\_pasaporte}, convocatoria}, nota,
         comentario)\end{flushleft}
      \end{description}
% \subsection{Tabla Reunión\_PreguntasOficiales}

\begin{verbatim}
  CREATE TABLE Reunión_PreguntasOficiales(
  dni_pasaporte         varchar(9)     NOT NULL,
  curso_académico       int(4)         NOT NULL,
  id_reunión            int(3)         NOT NULL,
  id_entrevista_oficial int(3)         NOT NULL,
  id_pregunta_oficial   int(3)         NOT NULL,
  respuesta             varchar(150)   NOT NULL,
  PRIMARY KEY pk_reuPregOfi (dni_pasaporte, curso_académico,
                             id_reunión, id_entrevista_oficial,
                             id_pregunta_oficial),
  FOREIGN KEY reuPregOfi_fk_reuniones (dni_pasaporte, curso_académico,
                                       id_reunión)
  REFERENCES Reuniones (dni_pasaporte, curso_académico, id_reunión)
  ON DELETE CASCADE ON UPDATE CASCADE,
  FOREIGN KEY reuPregOfi_fk_pregOfi (id_entrevista_oficial,
                                     id_pregunta_oficial)
  REFERENCES PreguntasOficiales (id_entrevista_oficial,
                                 id_pregunta_oficial)
  ON DELETE CASCADE ON UPDATE CASCADE
  );
\end{verbatim}
% \subsection{Tabla Reunión\_PreguntasAsesores}

  \paragraph{}Esta tabla se encuentra en FNBC puesto que el único determinante
  funcional existente es el identificador principal y todas las dependencias
  funcionales con el resto de atributos son completas.

  \begin{center}
    \begin{minipage}{5.0cm}{\begin{flushright}\begin{tabular}{ | c | }
                  \hline
                  (dni\_pasaporte\_alumno + \\
                  curso\_académico + \\
                  id\_reunión + \\
                  dni\_pasaporte\_asesor + \\
                  id\_entrevista\_asesor + \\
                  id\_pregunta\_asesor) \\
                  \hline
                 \end{tabular}\end{flushright} }
    \end{minipage}
    \begin{minipage}{0.7cm}{$\longrightarrow$}
    \end{minipage}
    \begin{minipage}{5.9cm}{\begin{tabular}{ | c | }
                  \hline
                  respuesta \\
                  \hline
                 \end{tabular} }
    \end{minipage}
  \end{center}
