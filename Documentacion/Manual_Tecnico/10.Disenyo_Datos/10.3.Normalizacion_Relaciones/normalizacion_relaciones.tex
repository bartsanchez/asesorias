\section{Normalización de relaciones}

  \paragraph{}El modelo relacional está soportado sobre una teoría de igual
  nombre basada en los principios de la teoría general de conjuntos. El
  proceso de normalización debe aplicarse independientemente del
  sistema de gestión de base de datos utilizado.

  \paragraph{}Este proceso elimina redundancias superfluas, aminorando de
  esta forma el espacio físico requerido para el almacenamiento de la
  información, por lo que se reducen los posibles problemas de
  integridad de la información almacenada en la base de datos.

  \paragraph{}Aumenta el desempeño, tanto de las operaciones de
  actualización de la información almacenada en la base de datos,
  como de las interrogaciones sobre la información almacenada en la
  misma.

  \paragraph{}Representa de forma coherente los objetos y relaciones
  presentes en el dominio del problema, y cuya información es
  almacenada en la base de datos.

  \paragraph{}El proceso de normalización va a ser aplicado a cada una de
  las tablas obtenidas mediante el modelo relacional para lograr un
  modelo más óptimo que cumpla con las características anteriormente
  descritas.

  \paragraph{}Todas las tablas obtenidas en el modelo relacional satisfacen
  la primera forman normal, (FN1\footnote{Según Luque Ruiz et al.
  \cite{luqueRuiz}, FN1: \textit{Una relación R satisface
  la primera forma normal si, y sólo si, todos los dominios subyacentes de la
  relación R contienen valores atómicos}.}) ya que no existe en ninguna tabla
  atributos múltiples.

  \paragraph{}El que una relación se encuentre en FN1 es condición
  indispensable aunque no suficiente para garantizar la consistencia del
  modelo relacional.

  \paragraph{}El proceso de normalización que se va a aplicar a continuación
  hará uso de los diagramas de dependencias funcionales que permiten
  representar las dependencias existentes entre los atributos de las
  tablas.

  \paragraph{}El proceso de normalización concluirá cuando todas las tablas
  obtenidas mediante el modelo relacional satisfagan la forma normal
  de Boyce-Codd (FNBC\footnote{Según Luque Ruiz et al. \cite{luqueRuiz}, FNBC:
  \textit{Una relación R satisface la forma
  normal de Boyce-Codd si, y sólo si, se encuentra en FN1, y cada determinante
  funcional es una clave candidata de la relación R}.}), ya que en este momento
  se considera que el esquema relacional es lo suficientemente consistente.

  \paragraph{}A continuación se mostrarán cada una de las tablas obtenidas
  tras la aplicación de la FNBC.

   \subsection{Tabla Centros}

      \paragraph{}Esta tabla se obtiene a través del tipo de entidad
      \textit{Centro}, tomando los atributos de ésta (regla
      RTECAR-1\footnote{Según Luque Ruiz et al. \cite{luqueRuiz}, regla
      RTECAR-1: \textit{``Todos los tipos de entidad presentes en el esquema
      conceptual se transformarán en tablas o relaciones en el esquema
      relacional manteniendo el número y tipo de atributos, así como la
      característica de identificador de estos atributos.''}}).

      \paragraph{}La clave principal de esta tabla es el atributo
      \textit{id\_centro}.

      \paragraph{}La tabla \textit{Centros} queda de la siguiente forma:

      \begin{description}
         \item[Centros] \begin{flushleft}(\underline{id\_centro},
         \textsc{nombre\_centro})\end{flushleft}
      \end{description}
\subsection{Tabla AdministradoresCentro}

  \paragraph{}Esta tabla se encuentra en FNBC, ya que cada determinante
  funcional es una clave candidata de la relación.

  \paragraph{}Existen dos dependencias funcionales, la que forma la clave
  primaria de la relación con el resto de atributos y, por otro lado, la que
  forma la clave alterna con el resto de atributos de la relación.

  \begin{center}
    \begin{minipage}{3.7cm}{\begin{flushright}\begin{tabular}{ | c | }
                  \hline
                  id\_adm\_centro \\
                  \hline
                 \end{tabular}\end{flushright} }
    \end{minipage}
    \begin{minipage}{0.7cm}{$\longrightarrow$}
    \end{minipage}
    \begin{minipage}{5.9cm}{\begin{tabular}{ | c | }
                  \hline
                  correo\_electrónico \\
                  nombre\_adm\_centro \\
                  \hline
                 \end{tabular} }
    \end{minipage}
  \end{center}

  \begin{center}
    \begin{minipage}{3.7cm}{\begin{flushright}\begin{tabular}{ | c | }
                  \hline
                  correo\_electrónico \\
                  \hline
                 \end{tabular}\end{flushright} }
    \end{minipage}
    \begin{minipage}{0.7cm}{$\longrightarrow$}
    \end{minipage}
    \begin{minipage}{5.9cm}{\begin{tabular}{ | c | }
                  \hline
                  id\_adm\_centro \\
                  nombre\_adm\_centro \\
                  \hline
                 \end{tabular} }
    \end{minipage}
  \end{center}
   \subsection{Tabla Titulaciones}

      \paragraph{}Esta tabla se obtiene a través del tipo de entidad
      \textit{Titulación}, tomando los atributos de ésta (regla RTECAR-1).
      Además, mantiene una referencia con la tabla \textit{Centro} a través
      del atributo \textit{id\_centro} (regla RTECAR-3.1).

      \paragraph{}La clave principal de la tabla se compone con la agregación de
      los atributos \textit{id\_centro} e \textit{id\_titulación}. Posee una
      clave alterna que será la formada por la agregación de los atributos
      \textit{id\_centro}, \textit{nombre\_titulación} y
      \textit{plan\_estudios}.

      \paragraph{}La tabla \textit{Titulaciones} queda de la siguiente forma:

      \begin{description}
         \item[Titulaciones] \begin{flushleft}(\underline{\textbf{ID\_CENTRO},
         id\_titulación}, \textsc{nombre\_titulación},
         \textsc{plan\_estudios})\end{flushleft}
      \end{description}

\subsection{Tabla Asignaturas}

  \paragraph{}Esta tabla se encuentra en FNBC, ya que cada determinante
  funcional es una clave candidata de la relación.

  \paragraph{}Existen dos dependencias funcionales, la que forma la clave
  primaria de la relación con el resto de atributos y, por otro lado, la que
  forma la clave alterna con el resto de atributos de la relación.

 \begin{center}
    \begin{minipage}{4.2cm}{\begin{flushright}\begin{tabular}{ | c | }
                  \hline
                  (id\_centro + \\
                  id\_titulación + \\
                  id\_asignatura) \\
                  \hline
                 \end{tabular}\end{flushright} }
    \end{minipage}
    \begin{minipage}{0.7cm}{$\longrightarrow$}
    \end{minipage}
    \begin{minipage}{5.9cm}{\begin{tabular}{ | c | }
                  \hline
                  nombre\_asignatura \\
                  curso \\
                  tipo \\
                  nCréditosTeóricos \\
                  nCréditosPrácticos \\
                  \hline
                 \end{tabular} }
    \end{minipage}
  \end{center}

  \begin{center}
    \begin{minipage}{4.2cm}{\begin{tabular}{ | c | }
                  \hline
                  (id\_centro + \\
                  id\_titulación + \\
                  nombre\_asignatura) \\
                  \hline
                 \end{tabular} }
    \end{minipage}
    \begin{minipage}{0.7cm}{$\longrightarrow$}
    \end{minipage}
    \begin{minipage}{5.9cm}{\begin{tabular}{ | c | }
                  \hline
                  id\_asignatura \\
                  curso \\
                  tipo \\
                  nCréditosTeóricos \\
                  nCréditosPrácticos \\
                  \hline
                 \end{tabular} }
    \end{minipage}
  \end{center}

\subsection{Tabla AsignaturasCursoAcadémico}

  \paragraph{}Esta tabla se encuentra en FNBC puesto que el único
  determinante funcional existente es el propio identificador principal
  de la tabla.

 \begin{center}
    \begin{minipage}{4.2cm}{\begin{flushright}\begin{tabular}{ | c | }
                  \hline
                  (id\_centro + \\
                  id\_titulación + \\
                  id\_asignatura + \\
                  curso\_académico) \\
                  \hline
                 \end{tabular}\end{flushright} }
    \end{minipage}
    \begin{minipage}{0.7cm}{$\longrightarrow$}
    \end{minipage}
    \begin{minipage}{5.9cm}{\begin{tabular}{ | c | }
                  \hline
                  \\
                  \hline
                 \end{tabular} }
    \end{minipage}
  \end{center}

   \subsection{Tabla Departamentos}

      \paragraph{}Esta tabla se obtiene a través del tipo de entidad
      \textit{Departamento}, tomando los atributos de ésta (regla RTECAR-1).

      \paragraph{}La clave principal de esta tabla es el atributo
      \textit{id\_departamento}. Posee una clave alterna que será la formada por
      el atributo \textit{nombre\_departamento}.

      \paragraph{}La tabla \textit{Departamentos} queda de la siguiente forma:

      \begin{description}
         \item[Departamentos] \begin{flushleft}(\underline{id\_departamento},
         \textsc{nombre\_departamento}, teléfono)\end{flushleft}
      \end{description}

\subsection{Tabla Asesores}

  \paragraph{}Esta tabla se encuentra en FNBC, ya que cada determinante
  funcional es una clave candidata de la relación.

  \paragraph{}Existen dos dependencias funcionales, la que forma la clave
  primaria de la relación con el resto de atributos y, por otro lado, la que
  forma la clave alterna con el resto de atributos de la relación.

  \begin{center}
    \begin{minipage}{3.7cm}{\begin{flushright}\begin{tabular}{ | c | }
                  \hline
                  dni\_pasaporte \\
                  \hline
                 \end{tabular}\end{flushright} }
    \end{minipage}
    \begin{minipage}{0.7cm}{$\longrightarrow$}
    \end{minipage}
    \begin{minipage}{5.9cm}{\begin{tabular}{ | c | }
                  \hline
                  correo\_electrónico \\
                  nombre \\
                  apellidos \\
                  teléfono \\
                  \hline
                 \end{tabular} }
    \end{minipage}
  \end{center}

  \begin{center}
    \begin{minipage}{3.7cm}{\begin{tabular}{ | c | }
                  \hline
                  correo\_electrónico \\
                  \hline
                 \end{tabular} }
    \end{minipage}
    \begin{minipage}{0.7cm}{$\longrightarrow$}
    \end{minipage}
    \begin{minipage}{5.9cm}{\begin{tabular}{ | c | }
                  \hline
                  dni\_pasaporte \\
                  nombre \\
                  apellidos \\
                  teléfono \\
                  \hline
                 \end{tabular} }
    \end{minipage}
  \end{center}

   \subsection{Tabla AsesoresCursoAcadémico}

      \paragraph{}Esta tabla se obtiene a través del tipo de entidad
      \textit{Asesor Curso Académico}, tomando los atributos de ésta (regla
      RTECAR-1). Además, mantiene una referencia con la tabla \textit{Asesor}, a
      través del atributo \textit{dni\_pasaporte} y otra referencia con la tabla
      \textit{Departamento}, a través del atributo \textit{id\_departamento}
      (regla RTECAR-3.1).

      \paragraph{}La clave principal de la tabla está compuesta por la
      agregación de los atributos \textit{dni\_pasaporte} y
      \textit{curso\_académico}.

      \paragraph{}La tabla \textit{AlumnosCursoAcadémico} queda de la siguiente
      forma:

      \begin{description}
         \item[AsesoresCursoAcadémico] \begin{flushleft}(\underline{\textbf{dni\_pasaporte},
         curso\_académico}, observaciones, \textbf{id\_departamento})\end{flushleft}
      \end{description}

   \subsection{Tabla PlantillasEntrevistaAsesor}

      \paragraph{}Esta tabla se obtiene a través del tipo de entidad
      \textit{Plantilla Entrevista Asesor}, tomando los atributos de ésta
      (regla RTECAR-1). Además, mantiene una referencia con la tabla
      \textit{Asesor Curso Académico} a través de los atributos
      \textit{dni\_pasaporte} y \textit{curso\_académico} (regla RTECAR-3.1).

      \paragraph{}La clave principal de esta tabla se compone con la agregación
      de los atributos \textit{dni\_pasaporte}, \textit{curso\_académico} e
      \textit{id\_entrevista\_asesor}.

      \paragraph{}La tabla \textit{PlantillasEntrevistaAsesor} queda de la
      siguiente forma:

      \begin{description}
         \item[PlantillasEntrevistaAsesor] \begin{flushleft}(\underline{\textbf{dni\_pasaporte},
         \textbf{curso\_académico},} \underline{id\_entrevista\_asesor},
         última\_modificación)\end{flushleft}
      \end{description}

\subsection{Tabla PreguntasAsesores}

  \paragraph{}Esta tabla se encuentra en FNBC puesto que el único determinante
  funcional existente es el identificador principal y todas las dependencias
  funcionales con el resto de atributos son completas.

  \begin{center}
    \begin{minipage}{4.5cm}{\begin{flushright}\begin{tabular}{ | c | }
                  \hline
                  (dni\_pasaporte + \\
                  curso\_académico + \\
                  id\_entrevista\_asesor + \\
                  id\_pregunta\_asesor) \\
                  \hline
                 \end{tabular}\end{flushright} }
    \end{minipage}
    \begin{minipage}{0.7cm}{$\longrightarrow$}
    \end{minipage}
    \begin{minipage}{5.9cm}{\begin{tabular}{ | c | }
                  \hline
                  enunciado \\
                  última\_modificación \\
                  \hline
                 \end{tabular} }
    \end{minipage}
  \end{center}

\subsection{Tabla Alumnos}

  \paragraph{}Esta tabla se encuentra en FNBC, ya que cada determinante
  funcional es una clave candidata de la relación.

  \paragraph{}Existen dos dependencias funcionales, la que forma la clave
  primaria de la relación con el resto de atributos y, por otro lado, la que
  forma la clave alterna con el resto de atributos de la relación.

  \begin{center}
    \begin{minipage}{3.7cm}{\begin{flushright}\begin{tabular}{ | c | }
                  \hline
                  dni\_pasaporte \\
                  \hline
                 \end{tabular}\end{flushright} }
    \end{minipage}
    \begin{minipage}{0.7cm}{$\longrightarrow$}
    \end{minipage}
    \begin{minipage}{5.9cm}{\begin{tabular}{ | c | }
                  \hline
                  correo\_electrónico \\
                  nombre \\
                  apellidos \\
                  fecha\_nacimiento \\
                  dirección\_córdoba \\
                  teléfono \\
                  dirección\_familiar \\
                  localidad\_familiar \\
                  provincia\_familiar \\
                  código\_postal \\
                  teléfono\_familiar \\
                  ingreso \\
                  otros\_estudios\_universitarios \\
                  modalidad\_acceso\_universidad \\
                  calificación\_acceso \\
                  \hline
                 \end{tabular} }
    \end{minipage}
  \end{center}

  \begin{center}
    \begin{minipage}{3.7cm}{\begin{tabular}{ | c | }
                  \hline
                  correo\_electrónico \\
                  \hline
                 \end{tabular} }
    \end{minipage}
    \begin{minipage}{0.7cm}{$\longrightarrow$}
    \end{minipage}
    \begin{minipage}{5.9cm}{\begin{tabular}{ | c | }
                  \hline
                  dni\_pasaporte \\
                  nombre \\
                  apellidos \\
                  fecha\_nacimiento \\
                  dirección\_córdoba \\
                  teléfono \\
                  dirección\_familiar \\
                  localidad\_familiar \\
                  provincia\_familiar \\
                  código\_postal \\
                  teléfono\_familiar \\
                  ingreso \\
                  otros\_estudios\_universitarios \\
                  modalidad\_acceso\_universidad \\
                  calificación\_acceso \\
                  \hline
                 \end{tabular} }
    \end{minipage}
  \end{center}

   \subsection{Tabla AlumnosCursoAcadémico}

      \paragraph{}Esta tabla se obtiene a través del tipo de entidad
      \textit{Alumno Curso Académico}, tomando los atributos de ésta (regla
      RTECAR-1). Además, mantiene una referencia con la tabla \textit{Alumno} a
      través del atributo \textit{dni\_pasaporte} y con la
      tabla \textit{Asesor Curso Académico} a través de los atributos
      \textit{dni\_pasaporte} y \textit{curso\_académico} (regla RTECAR-3.1).
      Nótese que para diferenciar los atributos \textit{dni\_pasaporte} de las
      entidades \textit{Alumno Curso Académico} y \textit{Asesor Curso
      Académico} se renombrarán a \textit{dni\_pasaporte\_alumno} y
      \textit{dni\_pasaporte\_asesor}, respectivamente. Además, es necesario
      indicar que el atributo \textit{curso\_académico} que se hereda de la
      entidad \textit{Asesor Curso Académico} no se representa en esta tabla.
      Esto es debido a que dicho atributo debe coincidir con el atributo del
      mismo nombre ya existente en esta tabla, por la propia naturaleza de la
      interrelación.

      \paragraph{}La clave principal de la tabla está compuesta por la
      agregación de los atributos \textit{dni\_pasaporte} y
      \textit{curso\_académico}.

      \paragraph{}La tabla \textit{AlumnosCursoAcadémico} queda de la siguiente
      forma:

      \begin{description}
         \item[AlumnosCursoAcadémico] \begin{flushleft}(\underline{
         \textbf{dni\_pasaporte\_alumno}, curso\_académico}, observaciones,
         \textbf{dni\_pasaporte\_asesor})\end{flushleft}
      \end{description}

\subsection{Tabla Matrículas}

\begin{verbatim}
  CREATE TABLE Matriculas(
  id_centro          int(2)         NOT NULL,
  id_titulación      int(2)         NOT NULL,
  id_asignatura      int(2)         NOT NULL,
  curso_académico    int(4)         NOT NULL,
  dni_pasaporte      varchar(9)     NOT NULL,
  comentario         varchar(100),
  PRIMARY KEY pk_matriculas (id_centro, id_titulación, id_asignatura,
                             curso_académico, dni_pasaporte),
  FOREIGN KEY matriculas_fk_AlCA (curso_académico, dni_pasaporte)
  REFERENCES AlumnosCursoAcadémico (curso_académico, dni_pasaporte)
  ON DELETE CASCADE ON UPDATE CASCADE,
  FOREIGN KEY matriculas_fk_ACA ( id_centro, id_titulación,
                                  id_asignatura)
  REFERENCES AsignaturasCursoAcadémico (id_centro, id_titulación,
                                        id_asignatura)
  ON DELETE CASCADE ON UPDATE CASCADE
  );
\end{verbatim}
   \subsection{Tabla PlantillasEntrevistaOficial}

      \paragraph{}Esta tabla se obtiene a través del tipo de entidad
      \textit{Plantilla Entrevista Oficial}, tomando los atributos de ésta
      (regla RTECAR-1).

      \paragraph{}La clave principal de esta tabla es el atributo
      \textit{id\_entrevista\_oficial}.

      \paragraph{}La tabla \textit{PlantillasEntrevistaOficial} queda de la
      siguiente forma:

      \begin{description}
         \item[PlantillasEntrevistaOficial] \begin{flushleft}(\underline{id\_entrevista\_oficial}, última\_modificación)\end{flushleft}
      \end{description}

   \subsection{Tabla PreguntasOficiales}

      \paragraph{}Esta tabla se obtiene a través del tipo de entidad
      \textit{Pregunta Oficial}, tomando los atributos de ésta
      (regla RTECAR-1). Además, mantiene una referencia con la tabla
      \textit{Plantilla Entrevista Oficial} a través del atributo
      \textit{id\_entrevista\_oficial} (regla RTECAR-3.1).

      \paragraph{}La clave principal de esta tabla se compone con la agregación
      de los atributos \textit{id\_entrevista\_oficial} e
      \textit{id\_pregunta\_oficial}.

      \paragraph{}La tabla \textit{PreguntasOficiales} queda de la
      siguiente forma:

      \begin{description}
         \item[PreguntasOficiales] \begin{flushleft}(\underline{\textbf{id\_entrevista\_oficial},
         id\_pregunta\_oficial})\end{flushleft}
      \end{description}

   \subsection{Tabla Reuniones}

      \paragraph{}Esta tabla se obtiene a través del tipo de entidad
      \textit{Reunión}, tomando los atributos de ésta (regla RTECAR-1).
      Además, mantiene una referencia con la tabla \textit{Alumno Curso
      Académico} a través de los atributos \textit{dni\_pasaporte} y
      \textit{curso\_académico} (regla RTECAR-3.1).

      \paragraph{}La clave principal de esta tabla se compone con la agregación
      de los atributos \textit{dni\_pasaporte}, \textit{curso\_académico} e
      \textit{id\_reunión}.

      \paragraph{}La tabla \textit{Reuniones} queda de la siguiente forma:

      \begin{description}
         \item[Reuniones] \begin{flushleft}(\underline{\textbf{dni\_pasaporte},
         \textbf{curso\_académico}, id\_reunión}, fecha, tipo,
         comentario\_asesor, comentario\_alumno)\end{flushleft}
      \end{description}

   \subsection{Tabla Centro\_AdministradoresCentro}

      \paragraph{}Esta tabla surge de la interrelación AC-C, existente entre
      los tipos de entidad \textit{Administrador Centro} y \textit{Centro}
      (regla RTECAR-4\footnote{Según Luque Ruiz et al. \cite{luqueRuiz}, regla
      RTECAR-4: \textit{En un tipo de
      interrelación binaria N:N cada tipo de entidad se transforma en una tabla
      por aplicación de la regla RTECAR-1 y se genera una nueva tabla para
      representar al tipo de interrelación. Esta tabla estará formada por los
      identificadores de los tipos de entidad que intervienen en el tipo
      de interrelación y por todos los atributos asociados al tipo de
      interrelación. La clave principal de esta tabla será la agregación de los
      atributos identificadores correspondientes a los tipos de entidad que
      intervienen en el tipo de interrelación.}}), tomando los atributos de
      ésta. Gracias a esta interrelación, se podrá conocer los diferentes
      administradores de existen en un centro.

      \paragraph{}La clave principal de esta tabla la forman la agregación de
      los atributos \textit{id\_centro} e \textit{id\_adm\_centro}. Estos
      atributos son a su vez claves foráneas.

      \paragraph{}La tabla \textit{Centro\_AdministradoresCentro} queda de la
      siguiente forma:

      \begin{description}
         \item[Centro\_AdministradoresCentro] \begin{flushleft}(\underline{\textbf{id\_centro},
         \textbf{id\_adm\_centro}})\end{flushleft}
      \end{description}
\subsection{Tabla Reunión\_PreguntasAsesores}

  \paragraph{}Esta tabla se encuentra en FNBC puesto que el único determinante
  funcional existente es el identificador principal y todas las dependencias
  funcionales con el resto de atributos son completas.

  \begin{center}
    \begin{minipage}{5.0cm}{\begin{flushright}\begin{tabular}{ | c | }
                  \hline
                  (dni\_pasaporte\_alumno + \\
                  curso\_académico + \\
                  id\_reunión + \\
                  dni\_pasaporte\_asesor + \\
                  id\_entrevista\_asesor + \\
                  id\_pregunta\_asesor) \\
                  \hline
                 \end{tabular}\end{flushright} }
    \end{minipage}
    \begin{minipage}{0.7cm}{$\longrightarrow$}
    \end{minipage}
    \begin{minipage}{5.9cm}{\begin{tabular}{ | c | }
                  \hline
                  respuesta \\
                  \hline
                 \end{tabular} }
    \end{minipage}
  \end{center}

\subsection{Tabla Reunión\_PreguntasOficiales}

  \paragraph{}Esta tabla se encuentra en FNBC puesto que el único determinante
  funcional existente es el identificador principal y todas las dependencias
  funcionales con el resto de atributos son completas.

  \begin{center}
    \begin{minipage}{4.4cm}{\begin{flushright}\begin{tabular}{ | c | }
                  \hline
                  (dni\_pasaporte + \\
                  curso\_académico + \\
                  id\_reunión + \\
                  id\_entrevista\_oficial + \\
                  id\_pregunta\_oficial) \\
                  \hline
                 \end{tabular}\end{flushright} }
    \end{minipage}
    \begin{minipage}{0.7cm}{$\longrightarrow$}
    \end{minipage}
    \begin{minipage}{5.9cm}{\begin{tabular}{ | c | }
                  \hline
                  respuesta \\
                  \hline
                 \end{tabular} }
    \end{minipage}
  \end{center}

