\section{Modelo de datos relacional}

  \paragraph{}Las tablas obtenidas después de la normalización son las
  siguientes:

  \begin{itemize}
   \item Tabla Alumno.
   \item Tabla Asesor.
   \item Tabla Centro.
   \item Tabla AdministradorCentro.
   \item Tabla Departamento.
  \end{itemize}

\subsection{Tabla Alumnos}

  \paragraph{}Esta tabla se encuentra en FNBC, ya que cada determinante
  funcional es una clave candidata de la relación.

  \paragraph{}Existen dos dependencias funcionales, la que forma la clave
  primaria de la relación con el resto de atributos y, por otro lado, la que
  forma la clave alterna con el resto de atributos de la relación.

  \begin{center}
    \begin{minipage}{3.7cm}{\begin{flushright}\begin{tabular}{ | c | }
                  \hline
                  dni\_pasaporte \\
                  \hline
                 \end{tabular}\end{flushright} }
    \end{minipage}
    \begin{minipage}{0.7cm}{$\longrightarrow$}
    \end{minipage}
    \begin{minipage}{5.9cm}{\begin{tabular}{ | c | }
                  \hline
                  correo\_electrónico \\
                  nombre \\
                  apellidos \\
                  fecha\_nacimiento \\
                  dirección\_córdoba \\
                  teléfono \\
                  dirección\_familiar \\
                  localidad\_familiar \\
                  provincia\_familiar \\
                  código\_postal \\
                  teléfono\_familiar \\
                  ingreso \\
                  otros\_estudios\_universitarios \\
                  modalidad\_acceso\_universidad \\
                  calificación\_acceso \\
                  \hline
                 \end{tabular} }
    \end{minipage}
  \end{center}

  \begin{center}
    \begin{minipage}{3.7cm}{\begin{tabular}{ | c | }
                  \hline
                  correo\_electrónico \\
                  \hline
                 \end{tabular} }
    \end{minipage}
    \begin{minipage}{0.7cm}{$\longrightarrow$}
    \end{minipage}
    \begin{minipage}{5.9cm}{\begin{tabular}{ | c | }
                  \hline
                  dni\_pasaporte \\
                  nombre \\
                  apellidos \\
                  fecha\_nacimiento \\
                  dirección\_córdoba \\
                  teléfono \\
                  dirección\_familiar \\
                  localidad\_familiar \\
                  provincia\_familiar \\
                  código\_postal \\
                  teléfono\_familiar \\
                  ingreso \\
                  otros\_estudios\_universitarios \\
                  modalidad\_acceso\_universidad \\
                  calificación\_acceso \\
                  \hline
                 \end{tabular} }
    \end{minipage}
  \end{center}

\subsection{Tabla Asesores}

  \paragraph{}Esta tabla se encuentra en FNBC, ya que cada determinante
  funcional es una clave candidata de la relación.

  \paragraph{}Existen dos dependencias funcionales, la que forma la clave
  primaria de la relación con el resto de atributos y, por otro lado, la que
  forma la clave alterna con el resto de atributos de la relación.

  \begin{center}
    \begin{minipage}{3.7cm}{\begin{flushright}\begin{tabular}{ | c | }
                  \hline
                  dni\_pasaporte \\
                  \hline
                 \end{tabular}\end{flushright} }
    \end{minipage}
    \begin{minipage}{0.7cm}{$\longrightarrow$}
    \end{minipage}
    \begin{minipage}{5.9cm}{\begin{tabular}{ | c | }
                  \hline
                  correo\_electrónico \\
                  nombre \\
                  apellidos \\
                  teléfono \\
                  \hline
                 \end{tabular} }
    \end{minipage}
  \end{center}

  \begin{center}
    \begin{minipage}{3.7cm}{\begin{tabular}{ | c | }
                  \hline
                  correo\_electrónico \\
                  \hline
                 \end{tabular} }
    \end{minipage}
    \begin{minipage}{0.7cm}{$\longrightarrow$}
    \end{minipage}
    \begin{minipage}{5.9cm}{\begin{tabular}{ | c | }
                  \hline
                  dni\_pasaporte \\
                  nombre \\
                  apellidos \\
                  teléfono \\
                  \hline
                 \end{tabular} }
    \end{minipage}
  \end{center}

   \subsection{Tabla Centros}

      \paragraph{}Esta tabla se obtiene a través del tipo de entidad
      \textit{Centro}, tomando los atributos de ésta (regla
      RTECAR-1\footnote{Según Luque Ruiz et al. \cite{luqueRuiz}, regla
      RTECAR-1: \textit{``Todos los tipos de entidad presentes en el esquema
      conceptual se transformarán en tablas o relaciones en el esquema
      relacional manteniendo el número y tipo de atributos, así como la
      característica de identificador de estos atributos.''}}).

      \paragraph{}La clave principal de esta tabla es el atributo
      \textit{id\_centro}.

      \paragraph{}La tabla \textit{Centros} queda de la siguiente forma:

      \begin{description}
         \item[Centros] \begin{flushleft}(\underline{id\_centro},
         \textsc{nombre\_centro})\end{flushleft}
      \end{description}
\subsection{Tabla AdministradoresCentro}

  \paragraph{}Esta tabla se encuentra en FNBC, ya que cada determinante
  funcional es una clave candidata de la relación.

  \paragraph{}Existen dos dependencias funcionales, la que forma la clave
  primaria de la relación con el resto de atributos y, por otro lado, la que
  forma la clave alterna con el resto de atributos de la relación.

  \begin{center}
    \begin{minipage}{3.7cm}{\begin{flushright}\begin{tabular}{ | c | }
                  \hline
                  id\_adm\_centro \\
                  \hline
                 \end{tabular}\end{flushright} }
    \end{minipage}
    \begin{minipage}{0.7cm}{$\longrightarrow$}
    \end{minipage}
    \begin{minipage}{5.9cm}{\begin{tabular}{ | c | }
                  \hline
                  correo\_electrónico \\
                  nombre\_adm\_centro \\
                  \hline
                 \end{tabular} }
    \end{minipage}
  \end{center}

  \begin{center}
    \begin{minipage}{3.7cm}{\begin{flushright}\begin{tabular}{ | c | }
                  \hline
                  correo\_electrónico \\
                  \hline
                 \end{tabular}\end{flushright} }
    \end{minipage}
    \begin{minipage}{0.7cm}{$\longrightarrow$}
    \end{minipage}
    \begin{minipage}{5.9cm}{\begin{tabular}{ | c | }
                  \hline
                  id\_adm\_centro \\
                  nombre\_adm\_centro \\
                  \hline
                 \end{tabular} }
    \end{minipage}
  \end{center}
   \subsection{Tabla Departamentos}

      \paragraph{}Esta tabla se obtiene a través del tipo de entidad
      \textit{Departamento}, tomando los atributos de ésta (regla RTECAR-1).

      \paragraph{}La clave principal de esta tabla es el atributo
      \textit{id\_departamento}. Posee una clave alterna que será la formada por
      el atributo \textit{nombre\_departamento}.

      \paragraph{}La tabla \textit{Departamentos} queda de la siguiente forma:

      \begin{description}
         \item[Departamentos] \begin{flushleft}(\underline{id\_departamento},
         \textsc{nombre\_departamento}, teléfono)\end{flushleft}
      \end{description}


%    \subsection{Tabla Titulaciones}
%
%       \paragraph{}Esta tabla se obtiene a través del tipo de entidad
%       \textit{Titulación}, tomando los atributos de ésta (regla RTECAR-1).
%       Además, mantiene una referencia con la tabla \textit{Centro} a través
%       del atributo \textit{id\_centro} (regla RTECAR-3.1).
%
%       \paragraph{}La clave principal de la tabla se compone con la agregación de
%       los atributos \textit{id\_centro} e \textit{id\_titulación}. Posee una
%       clave alterna que será la formada por la agregación de los atributos
%       \textit{id\_centro}, \textit{nombre\_titulación} y
%       \textit{plan\_estudios}.
%
%       \paragraph{}La tabla \textit{Titulaciones} queda de la siguiente forma:
%
%       \begin{description}
%          \item[Titulaciones] \begin{flushleft}(\underline{\textbf{ID\_CENTRO}},
%          \underline{id\_titulación}, \textsc{nombre\_titulación},
%          \textsc{plan\_estudios})\end{flushleft}
%       \end{description}
%
%    \subsection{Tabla Asignaturas}
%
%       \paragraph{}Esta tabla se obtiene a través del tipo de entidad
%       \textit{Asignatura}, tomando los atributos de ésta (regla RTECAR-1).
%       Además, mantiene una referencia con la tabla \textit{Titulación} a través
%       de los atributos \textit{id\_centro} e \textit{id\_titulación} (regla
%       RTECAR-3.1).
%
%       \paragraph{}La clave principal de esta tabla se compone con la agregación
%       de los atributos \textit{id\_centro}, \textit{id\_titulación} e
%       \textit{id\_asignatura}. Posee una clave alterna que será la formada por
%       la agregación de los atributos \textit{id\_centro},
%       \textit{id\_titulación} y \textit{nombre\_asignatura}.
%
%       \paragraph{}La tabla \textit{Asignaturas} queda de la siguiente forma:
%
%       \begin{description}
%          \item[Asignaturas] \begin{flushleft}(\underline{\textbf{ID\_CENTRO}},
%          \underline{\textbf{ID\_TITULACIÓN}}, \underline{id\_asignatura},
%          \textsc{nombre\_asignatura}, curso, tipo, nCréditosTeóricos,
%          nCréditosPrácticos)\end{flushleft}
%       \end{description}
%
%    \subsection{Tabla AsignaturasPorCurso}
%
%       \begin{description}
%          \item[AsignaturasPorCurso] \begin{flushleft}(\underline{\textbf{id\_centro}},
%          \underline{\textbf{id\_titulación}}, \underline{\textbf{id\_asignatura}},
%          \underline{curso\_académico})\end{flushleft}
%       \end{description}
%
%    \subsection{Tabla AlumnosPorCurso}
%
%       \begin{description}
%          \item[AlumnosPorCurso] \begin{flushleft}(\underline{\textbf{dni\_pasaporte}},
%          \underline{curso\_académico})\end{flushleft}
%       \end{description}
%
%
%    \subsection{Tabla AsesoresPorCurso}
%
%       \begin{description}
%          \item[AsesoresPorCurso] \begin{flushleft}(\underline{\textbf{dni\_pasaporte}},
%          \underline{curso\_académico}, \underline{\textbf{id\_departamento}})
%          \end{flushleft}
%       \end{description}
%
%    \subsection{Tabla Administradores}
%
%       \begin{description}
%          \item[Administradores] \begin{flushleft}(\underline{\textbf{id\_adm\_centro}},
%          \underline{\textbf{id\_centro}})\end{flushleft}
%       \end{description}
%
%    \subsection{Tabla Matriculados}
%
%       \begin{description}
%          \item[Matriculados] \begin{flushleft}(\underline{\textbf{curso\_académico\_alumno}},
%          \underline{\textbf{dni\_pasaporte}}, \underline{\textbf{id\_centro}},
%          \underline{\textbf{id\_titulación}}, \underline{\textbf{id\_asignatura}},
%          \underline{\textbf{curso\_académico\_asignatura}}, \underline{convocatoria}, nota,
%          comentario)\end{flushleft}
%       \end{description}