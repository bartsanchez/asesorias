   \subsection{Tabla Preguntas}

      \paragraph{}Esta tabla se obtiene a través del tipo de entidad
      \textit{Preguntas}, tomando los atributos de ésta (regla RTECAR-1).
      Además, mantiene una referencia o bien con la tabla
      \textit{EntrevistasGenerales} o bien con \textit{EntrevistasAsesores }a
      través del atributo \textit{id\_entrevista} (regla RTECAR-3.1).

      \paragraph{}La clave principal de la tabla es la compuesta por la
      agregación de los atributos \textit{id\_entrevista} e
      \textit{id\_pregunta}.

      \paragraph{}La tabla \textit{Preguntas} queda de la siguiente forma:

      \begin{description}
         \item[Preguntas] \begin{flushleft}(\underline{\textbf{id\_entrevista},
         id\_pregunta}, enunciado)\end{flushleft}
      \end{description}
