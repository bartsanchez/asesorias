   \subsection{Tabla Reuniones}

      \paragraph{}Esta tabla se obtiene a través del tipo de entidad
      \textit{Reunión}, tomando los atributos de ésta (regla RTECAR-1).
      Además, mantiene una referencia con la tabla \textit{Alumno Curso
      Académico} a través de los atributos \textit{dni\_pasaporte} y
      \textit{curso\_académico} (regla RTECAR-3.1).

      \paragraph{}La clave principal de esta tabla se compone con la agregación
      de los atributos \textit{dni\_pasaporte}, \textit{curso\_académico} e
      \textit{id\_reunión}.

      \paragraph{}La tabla \textit{Reuniones} queda de la siguiente forma:

      \begin{description}
         \item[Reuniones] \begin{flushleft}(\underline{\textbf{dni\_pasaporte},
         \textbf{curso\_académico}, id\_reunión}, fecha, tipo,
         comentario\_asesor, comentario\_alumno)\end{flushleft}
      \end{description}
