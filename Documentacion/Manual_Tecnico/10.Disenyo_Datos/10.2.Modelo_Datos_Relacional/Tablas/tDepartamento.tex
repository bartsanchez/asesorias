   \subsection{Tabla Departamentos}

      \paragraph{}Esta tabla se obtiene a través del tipo de entidad
      \textit{Departamento}, tomando los atributos de ésta (regla RTECAR-1).

      \paragraph{}La clave principal de esta tabla es el atributo
      \textit{id\_departamento}. Posee una clave alterna que será la formada por
      el atributo \textit{nombre\_departamento}.

      \paragraph{}La tabla \textit{Departamentos} queda de la siguiente forma:

      \begin{description}
         \item[Departamentos] \begin{flushleft}(\underline{id\_departamento},
         \textsc{nombre\_departamento}, teléfono)\end{flushleft}
      \end{description}
