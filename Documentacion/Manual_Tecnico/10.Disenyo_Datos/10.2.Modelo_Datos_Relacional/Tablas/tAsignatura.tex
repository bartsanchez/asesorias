   \subsection{Tabla Asignaturas}

      \paragraph{}Esta tabla se obtiene a través del tipo de entidad
      \textit{Asignatura}, tomando los atributos de ésta (regla RTECAR-1).
      Además, mantiene una referencia con la tabla \textit{Titulación} a través
      de los atributos \textit{id\_centro} e \textit{id\_titulación} (regla
      RTECAR-3.1).

      \paragraph{}La clave principal de esta tabla se compone con la agregación
      de los atributos \textit{id\_centro}, \textit{id\_titulación} e
      \textit{id\_asignatura}. Posee una clave alterna que será la formada por
      la agregación de los atributos \textit{id\_centro},
      \textit{id\_titulación} y \textit{nombre\_asignatura}.

      \paragraph{}La tabla \textit{Asignaturas} queda de la siguiente forma:

      \begin{description}
         \item[Asignaturas] \begin{flushleft}(\underline{\textbf{ID\_CENTRO},
         \textbf{ID\_TITULACIÓN}, id\_asignatura},
         \textsc{nombre\_asignatura}, curso, tipo, nCréditosTeóricos,
         nCréditosPrácticos)\end{flushleft}
      \end{description}
