   \subsection{Tabla Alumnos}

      \paragraph{}Esta tabla se obtiene a través del tipo de entidad
      \textit{Alumno}, tomando los atributos de ésta (regla
      RTECAR-1\footnote{Regla RTECAR-1: \textit{``Todos los tipos de entidad
      presentes en el esquema conceptual se transformarán en tablas o relaciones
      en el esquema relacional manteniendo el número y tipo de atributos, así
      como la característica de identificador de estos atributos.''}}).

      \paragraph{}La clave principal de esta tabla es el atributo
      \textit{dni\_pasaporte}. Posee una clave alterna que será la formada por
      el atributo \textit{correo\_electrónico}.

      \paragraph{}La tabla \textit{Alumnos} queda de la siguiente forma:

      \begin{description}
         \item[Alumnos] \begin{flushleft}(\underline{dni\_pasaporte},
         \textsc{correo\_electrónico}, nombre, apellidos, fecha\_nacimiento,
         dirección\_córdoba, teléfono, dirección\_familiar, localidad\_familiar,
         provincia\_familiar, código\_postal, teléfono\_familiar, ingreso,
         otros\_estudios\_universitarios, modalidad\_acceso\_universidad,
         calificación\_acceso)\end{flushleft}
      \end{description}
