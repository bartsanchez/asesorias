   \subsection{Tabla Titulaciones}

      \paragraph{}Esta tabla se obtiene a través del tipo de entidad
      \textit{Titulación}, tomando los atributos de ésta (regla RTECAR-1).
      Además, mantiene una referencia con la tabla \textit{Centro} a través
      del atributo \textit{id\_centro} (regla RTECAR-3.1\footnote{Según Luque
      Ruiz et al. \cite{luqueRuiz}, regla RTECAR-3.1: \textit{``Si en un tipo de
      interrelación binaria 1:N ambos tipos de entidad participan de forma
      total, o el tipo de entidad que interviene con cardinalidad máxima muchos
      participa de forma parcial, entonces, cada tipo de entidad se transforma
      en una tabla por aplicación de la regla RTECAR-1, y el identificador del
      tipo de entidad que participa con cardinalidad máxima uno pasa a formar
      parte de la tabla correspondiente al tipo de entidad que participa con
      cardinalidad máxima muchos. Este atributo será definido como clave foránea
      de esta tabla (no pudiendo tomar valores nulos) manteniendo una referencia
      con la tabla correspondiente al tipo de entidad que participa con
      cardinalidad máxima uno. Si el tipo de interrelación tuviera atributos
      asociados, estos atributos pasan a formar parte de la tabla
      correspondiente al tipo de entidad que participa con cardinalidad máxima
      muchos.}}).

      \paragraph{}La clave principal de la tabla se compone con la agregación de
      los atributos \textit{id\_centro} e \textit{id\_titulación}. Posee una
      clave alterna que será la formada por la agregación de los atributos
      \textit{id\_centro}, \textit{nombre\_titulación} y
      \textit{plan\_estudios}.

      \paragraph{}La tabla \textit{Titulaciones} queda de la siguiente forma:

      \begin{description}
         \item[Titulaciones] \begin{flushleft}(\underline{\textbf{ID\_CENTRO},
         id\_titulación}, \textsc{nombre\_titulación},
         \textsc{plan\_estudios})\end{flushleft}
      \end{description}
