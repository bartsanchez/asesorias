   \subsection{Tabla Centros}

      \paragraph{}Esta tabla se obtiene a través del tipo de entidad
      \textit{Centro}, tomando los atributos de ésta (regla RTECAR-1\footnote{Regla
      RTECAR-1: \textit{``Todos los tipos de entidad
      presentes en el esquema conceptual se transformarán en tablas o relaciones
      en el esquema relacional manteniendo el número y tipo de atributos, así
      como la característica de identificador de estos atributos.''}}).

      \paragraph{}La clave principal de esta tabla es el atributo
      \textit{id\_centro}.

      \paragraph{}La tabla \textit{Centros} queda de la siguiente forma:

      \begin{description}
         \item[Centros] \begin{flushleft}(\underline{id\_centro},
         \textsc{nombre\_centro})\end{flushleft}
      \end{description}