   \subsection{Tabla AsignaturasCursoAcadémico}

      \paragraph{}Esta tabla se obtiene a través del tipo de entidad
      \textit{Asignatura Curso Académico}, tomando los atributos de ésta
      (regla RTECAR-1). Además, mantiene una referencia con la tabla
      \textit{Asignatura} a través de los atributos \textit{id\_centro},
      \textit{id\_titulación} e \textit{id\_asignatura} (regla
      RTECAR-3.1).

      \paragraph{}La clave principal de esta tabla se compone con la agregación
      de los atributos \textit{id\_centro}, \textit{id\_titulación},
      \textit{id\_asignatura} y \textit{curso\_académico}.

      \paragraph{}La tabla \textit{AsignaturasCursoAcadémico} queda de la
      siguiente forma:

      \begin{description}
         \item[AsignaturasCursoAcadémico] \begin{flushleft}(\underline{\textbf{id\_centro},
         \textbf{id\_titulación}, \textbf{id\_asignatura},} \underline{curso\_académico})
         \end{flushleft}
      \end{description}
