   \subsection{Tabla AlumnosCursoAcadémico}

      \paragraph{}Esta tabla se obtiene a través del tipo de entidad
      \textit{Alumno Curso Académico}, tomando los atributos de ésta (regla
      RTECAR-1). Además, mantiene una referencia con la tabla \textit{Alumno} a
      través del atributo \textit{dni\_pasaporte} (regla RTECAR-3.1).

      \paragraph{}La clave principal de la tabla está compuesta por la
      agregación de los atributos \textit{dni\_pasaporte} y
      \textit{curso\_académico}.

      \paragraph{}La tabla \textit{AlumnosCursoAcadémico} queda de la siguiente
      forma:

      \begin{description}
         \item[AlumnosCursoAcadémico] \begin{flushleft}(\underline{\textbf{dni\_pasaporte},
         curso\_académico}, observaciones)\end{flushleft}
      \end{description}
