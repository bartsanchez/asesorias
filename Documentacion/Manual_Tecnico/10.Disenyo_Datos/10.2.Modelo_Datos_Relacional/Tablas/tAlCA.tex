   \subsection{Tabla AlumnosCursoAcadémico}

      \paragraph{}Esta tabla se obtiene a través del tipo de entidad
      \textit{Alumno Curso Académico}, tomando los atributos de ésta (regla
      RTECAR-1). Además, mantiene una referencia con la tabla \textit{Alumno} a
      través del atributo \textit{dni\_pasaporte} y con la
      tabla \textit{Asesor Curso Académico} a través de los atributos
      \textit{dni\_pasaporte} y \textit{curso\_académico} (regla RTECAR-3.1).
      Nótese que para diferenciar los atributos \textit{dni\_pasaporte} de las
      entidades \textit{Alumno Curso Académico} y \textit{Asesor Curso
      Académico} se renombrarán a \textit{dni\_pasaporte\_alumno} y
      \textit{dni\_pasaporte\_asesor}, respectivamente. Además, es necesario
      indicar que el atributo \textit{curso\_académico} que se hereda de la
      entidad \textit{Asesor Curso Académico} no se representa en esta tabla.
      Esto es debido a que dicho atributo debe coincidir con el atributo del
      mismo nombre ya existente en esta tabla, por la propia naturaleza de la
      interrelación.

      \paragraph{}La clave principal de la tabla está compuesta por la
      agregación de los atributos \textit{dni\_pasaporte} y
      \textit{curso\_académico}.

      \paragraph{}La tabla \textit{AlumnosCursoAcadémico} queda de la siguiente
      forma:

      \begin{description}
         \item[AlumnosCursoAcadémico] \begin{flushleft}(\underline{
         \textbf{dni\_pasaporte\_alumno}, curso\_académico}, observaciones,
         \textbf{dni\_pasaporte\_asesor})\end{flushleft}
      \end{description}
