   \subsection{Tabla CalificacionesAlumnoAsignaturaCA}

      \paragraph{}Esta tabla se obtiene a través del tipo de entidad
      \textit{Calificación Alumno Asignatura CA}, tomando los atributos de ésta
      (regla RTECAR-1). Además, mantiene una referencia con la tabla
      \textit{Asignatura Curso Académico} a través de los atributos
      \textit{id\_centro}, \textit{id\_titulación}, \textit{id\_asignatura} y
      \textit{curso\_académico} (regla RTECAR-3.1) y con la tabla
      \textit{Alumno Curso Académico} a través del atributo
      \textit{dni\_pasaporte}(regla RTECAR-3.1). Es necesario puntualizar que el
      atributo \textit{curso\_académico} del tipo de entidad \textit{Alumno Curso
      Académico} no se tiene en cuenta, ya que se está haciendo referencia al
      atributo del mismo nombre del tipo de entidad \textit{Asignatura Curso
      Académico}, que contiene, forzosamente, la misma información.

      \paragraph{}La clave principal de esta tabla se compone con la agregación
      de los atributos \textit{dni\_pasaporte}, \textit{curso\_académico},
      \textit{id\_centro}, \textit{id\_titulación}, \textit{id\_asignatura} y
      \textit{convocatoria}.

      \paragraph{}La tabla \textit{AsignaturasCursoAcadémico} queda de la
      siguiente forma:

      \begin{description}
         \item[CalificacionesAlumnoAsignaturaCA] \begin{flushleft}(         \underline{\textbf{dni\_pasaporte}, \textbf{curso\_académico},}
         \underline{\textbf{id\_centro}, \textbf{id\_titulación},
         \textbf{id\_asignatura}, convocatoria}, nota, comentario)
         \end{flushleft}
      \end{description}
