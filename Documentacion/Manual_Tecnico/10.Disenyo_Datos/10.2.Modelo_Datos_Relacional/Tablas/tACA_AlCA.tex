   \subsection{Tabla Asignatura\_Alumnos\_CursoAcadémico}

      \paragraph{}Esta tabla surge de la interrelación ACA-AlCA, existente entre
      los tipos de entidad \textit{Asignatura Curso Académico} y \textit{Alumno
      Curso Académico} (regla RTECAR-4), tomando los atributos de
      ésta. Gracias a esta interrelación, se podrá conocer los diferentes
      alumnos matriculados en una determinada asignatura durante un curso
      académico.

      \paragraph{}La clave principal de esta tabla la forman la agregación de
      los atributos \textit{id\_centro}, \textit{id\_titulación},
      \textit{id\_asignatura}, \textit{curso\_académico},
      \textit{dni\_pasaporte} y \textit{convocatoria}. Estos atributos son a su
      vez claves foráneas excepto \textit{convocatoria}, el cual es un atributo
      identificador de la interrelación y no de otra entidad.

      \paragraph{}La tabla
      \textit{Asignatura\_Alumnos\_CursoAcadémico} queda, por tanto,
      de la siguiente forma:

      \begin{description}
         \item[Asignatura\_Alumnos\_CursoAcadémico] \begin{flushleft}(\underline{\textbf{id\_centro}, \textbf{id\_titulación},}
         \underline{\textbf{id\_asignatura}, \textbf{curso\_académico},
         \textbf{dni\_pasaporte}, convocatoria}, nota,
         comentario)\end{flushleft}
      \end{description}