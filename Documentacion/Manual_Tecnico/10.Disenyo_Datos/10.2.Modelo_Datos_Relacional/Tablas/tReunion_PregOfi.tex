  \subsection{Tabla Reunión\_PreguntasOficiales}

      \paragraph{}Esta tabla surge de la interrelación R-PO, existente entre
      los tipos de entidad \textit{Reunión} y \textit{Pregunta Oficial}
      (regla RTECAR-4), tomando los atributos de ésta. Gracias a esta
      interrelación, se podrá conocer las diferentes preguntas oficiales que
      componen una reunión.

      \paragraph{}La clave principal de esta tabla la forman la agregación de
      los atributos \textit{dni\_pasaporte}, \textit{curso\_académico},
      \textit{id\_reunión}, \textit{id\_entrevista\_oficial} e
      \textit{id\_pregunta\_oficial}. Estos atributos son a su vez claves
      foráneas.

      \paragraph{}La tabla \textit{Reunión\_PreguntasOficiales} queda de la
      siguiente forma:

      \begin{description}
         \item[Reunión\_PreguntasOficiales] \begin{flushleft}(\underline{\textbf{dni\_pasaporte}, \textbf{curso\_académico},
         \textbf{id\_reunión}}, \underline{\textbf{id\_entrevista\_oficial},
         \textbf{id\_pregunta\_oficial}}, respuesta)\end{flushleft}
      \end{description}