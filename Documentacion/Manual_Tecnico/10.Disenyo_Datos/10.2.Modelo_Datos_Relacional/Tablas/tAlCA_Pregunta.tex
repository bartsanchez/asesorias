   \subsection{Tabla AlumnoCursoAcadémico\_Preguntas}

      \paragraph{}Esta tabla surge de la interrelación P-AlCA, existente entre
      los tipos de entidad \textit{Pregunta} y \textit{Alumno Curso Académico}
      (regla RTECAR-4), tomando los atributos de ésta. Gracias a esta
      interrelación, se podrá conocer toda la información relativa a las
      preguntas realizadas a un determinado alumno, como tipo de pregunta, fecha
      de realización o respuesta, por ejemplo.

      \paragraph{}La clave principal de esta tabla es la agregación de los
      atributos \textit{dni\_pasaporte}, \textit{curso\_académico},
      \textit{id\_entrevista} e \textit{id\_pregunta}. Estos atributos son a su
      vez claves foráneas.

      \paragraph{}La tabla
      \textit{AlumnoCursoAcadémico\_Preguntas} queda de la siguiente forma:

      \begin{description}
         \item[AlumnoCursoAcadémico\_Preguntas] \begin{flushleft}(\underline{\textbf{dni\_pasaporte}, \textbf{curso\_académico},}
         \underline{\textbf{id\_entrevista}, \textbf{id\_pregunta}},
         tipo, fecha, respuesta)\end{flushleft}
      \end{description}