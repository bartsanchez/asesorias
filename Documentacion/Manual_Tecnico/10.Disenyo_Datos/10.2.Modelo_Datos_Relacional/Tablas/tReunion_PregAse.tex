  \subsection{Tabla Reunión\_PreguntasAsesores}

      \paragraph{}Esta tabla surge de la interrelación R-PA, existente entre
      los tipos de entidad \textit{Reunión} y \textit{Pregunta Asesor}
      (regla RTECAR-4), tomando los atributos de ésta. Gracias a esta
      interrelación, se podrá conocer las diferentes preguntas de asesor que
      componen una reunión.

      \paragraph{}La clave principal de esta tabla la forman la agregación de
      los atributos \textit{dni\_pasaporte}, \textit{curso\_académico},
      \textit{id\_reunión}, \textit{dni\_pasaporte},
      \textit{id\_entrevista\_asesor} e \textit{id\_pregunta\_asesor}. Estos
      atributos son a su vez claves foráneas.

      \paragraph{}Además, el atributo \textit{curso\_académico} de la tabla
      resultante corresponde al tipo de entidad \textit{Reunión}, no teniendo en
      cuenta el atributo del mismo nombre de la entidad
      \textit{Pregunta Asesor}, por tener forzosamente el mismo valor.

      \paragraph{}La tabla \textit{Reunión\_PreguntasAsesores} queda de la
      siguiente forma:

      \begin{description}
         \item[Reunión\_PreguntasAsesores] \begin{flushleft}(\underline{\textbf{dni\_pasaporte}, \textbf{curso\_académico},
         \textbf{id\_reunión}}, \underline{\textbf{dni\_pasaporte},
         \textbf{id\_entrevista\_asesor}, \textbf{id\_pregunta\_asesor}},
         respuesta)\end{flushleft}
      \end{description}