   \subsection{Tabla Centro\_AdministradoresCentro}

      \paragraph{}Esta tabla surge de la interrelación AC-C, existente entre
      los tipos de entidad \textit{Administrador Centro} y \textit{Centro}
      (regla RTECAR-4\footnote{Según Luque Ruiz et al. \cite{luqueRuiz}, regla
      RTECAR-4: \textit{En un tipo de
      interrelación binaria N:N cada tipo de entidad se transforma en una tabla
      por aplicación de la regla RTECAR-1 y se genera una nueva tabla para
      representar al tipo de interrelación. Esta tabla estará formada por los
      identificadores de los tipos de entidad que intervienen en el tipo
      de interrelación y por todos los atributos asociados al tipo de
      interrelación. La clave principal de esta tabla será la agregación de los
      atributos identificadores correspondientes a los tipos de entidad que
      intervienen en el tipo de interrelación.}}), tomando los atributos de
      ésta. Gracias a esta interrelación, se podrá conocer los diferentes
      administradores de existen en un centro.

      \paragraph{}La clave principal de esta tabla la forman la agregación de
      los atributos \textit{id\_centro} e \textit{id\_adm\_centro}. Estos
      atributos son a su vez claves foráneas.

      \paragraph{}La tabla \textit{Centro\_AdministradoresCentro} queda de la
      siguiente forma:

      \begin{description}
         \item[Centro\_AdministradoresCentro] \begin{flushleft}(\underline{\textbf{id\_centro},
         \textbf{id\_adm\_centro}})\end{flushleft}
      \end{description}