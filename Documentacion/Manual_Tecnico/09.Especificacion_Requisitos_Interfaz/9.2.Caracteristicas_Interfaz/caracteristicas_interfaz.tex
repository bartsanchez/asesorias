\section{Características de la interfaz}

  \paragraph{}Para conseguir los objetivos citados anteriormente, la interfaz se
  va a desarrollar siguiendo varios puntos a tener en cuenta en el diseño de la
  aplicación:

  \begin{itemize}
   \item La página ha de ser rápida, sencilla y directa, con el fin de facilitar
   la navegación por la misma.
   \item Ha de disponer de buena legibilidad; el color de los textos debe
   contrastar con el del fondo y el tamaño de la letra debe ser suficientemente
   grade.
   \item Se limitará el número de acciones que el usuario puede llevar a cabo,
   facilitándole el uso del sitio web.
   \item Se sustituirá la introducción por la selección de elementos siempre
   que sea posible y usando nombres en lugar de números en la mayor parte de
   los casos.
   \item Cuando ocurra algún error, se notificará al usuario de forma clara y
   concisa, de forma que éste pueda subsanar el mismo.
   \item Los nombres de los diferentes elementos serán lo más autodescriptivos
   que sea posible, para facilitar la comprensión de su cometido.
   \item La aplicación dispondrá de una ayuda en la que se explicará el
   funcionamiento de aquella, con el fin de que cualquier usuario pueda
   resolverlas de inmediato. La ayuda será lo más sencilla pero completa
   posible.
   \item En aquellas operaciones en las que aparezcan campos de la base de
   datos, se distinguirán aquellos que formen parte de la clave, de los que no,
   resaltando aquellos en un color negro más intenso.
   \item En los formularios, se diferenciarán los campos obligatorios de los
   optativos mediante un asterisco (*) que indicará que el campo es de carácter
   obligatorio.
   \item Cuando la inserción de un elemento dependa de la existencia y selección
   de otro (clave foránea), se permitirá, mediante un botón, acceder al módulo
   de creación de dicho elemento.
  \end{itemize}
