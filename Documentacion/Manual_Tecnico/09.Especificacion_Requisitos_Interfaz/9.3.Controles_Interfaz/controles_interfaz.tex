\section{Controles de la interfaz}

  \subsection{Apariencia}

  \paragraph{}La aplicación tendrá una interfaz de tipo web, con lo que se
  aprovecharán las facilidades que, para estos cometidos, tienen los
  navegadores.

  \paragraph{}La aplicación se desarrollará en una resolución de pantalla de
  800*600 píxeles, con el objetivo de estar accesible para la gran mayoría de
  los monitores actuales.

  \paragraph{}En la página principal, se encontrará el formulario con tres
  campos: identificador de cada usuario, rol con el que desea acceder y
  su contraseña, para que el usuario registrado entre al sistema.

  \paragraph{}En las siguientes páginas, habrá una botonera situada en la parte
  superior de la ventana, donde se dispondrán las acciones de mayor complejidad
  (menú principal). A su vez, en la zona izquierda aparecerán las opciones
  dependientes de la opción elegida en el menú principal. Estas opciones
  cambiarán dependiendo del rol con el que se haya accedido al sistema, solo
  permitiendo el acceso a las partes concernientes a dicho rol. Estos botones
  estarán disponibles en todas las páginas del generador para facilitar el
  desplazamiento desde cualquier punto de la página a otro.

  \subsection{Usabilidad}

  \paragraph{}La forma en que el usuario introduce datos en el sistema se
  realiza mediante los elementos típicos de una página web, como por ejemplo
  botones, enlaces, menús y formularios.

  \paragraph{}En la zona de consultas aparecerá un formulario que permitirá la
  realización de búsquedas tan complejas como se desee. Se permitirá la búsqueda
  en función de múltiples campos de información. Además, en un mismo campo se
  podrán introducir varios valores relacionados por un operador lógico. Los
  resultados de las búsquedas se presentarán de forma que no haya una excesiva
  acumulación de información mostrada en pantalla.

  \subsection{Seguridad}

  \paragraph{}Debido a los efectos indeseables que podría ocasionar el uso
  fraudulento de los distintos tipos de usuario, se diseñará un sistema de
  protección consistente en los siguientes puntos:

  \begin{itemize}
   \item Protección contra el acceso/creación/eliminación de información, por
   parte de un usuario no autorizado, cuando dicha información no sea
   responsabilidad del usuario. Ejemplo: un usuario administrador de centro que
   pretendiera borrar una titulación de un centro que no le pertenece.
   \item Disposición de un botón de cierre de sesión, lo que provoca que el
   usuario deje de estar debidamente identificado en el sistema. De esta forma
   ningún usuario sin privilegios podría entrar a las páginas de otros usuarios
   cuando dicho usuario sale de la sesión.
   \item Control de tiempo de inactividad de la aplicación. Si el
   usuario está más de quince minutos sin realizar acción alguna, cuando intente
   volver a la actividad será devuelto a la página principal de acceso al
   sistema, tras su cierre de sesión.
   \item Cierre automático de sesión cuando el usuario cierra las ventanas
   del navegador, pese a no haber cerrado sesión de forma explícita.
  \end{itemize}

  \paragraph{}Además, en el momento en el que el usuario solicite a la
  aplicación que realice alguna opción destructiva de carácter importante, el
  sistema pedirá confirmación para la realización de dicha acción, como por
  ejemplo la eliminación de un usuario por parte de algún usuario administrador.
