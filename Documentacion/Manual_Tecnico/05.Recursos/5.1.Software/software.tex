\section{Software}

   \paragraph{}La mayoría de las aplicaciones que a continuación se detallan,
   o bien han sido justificadas en el apartado \ref{facEst}, \textit{Factores
   estratégicos}, o bien se justifican llegado el momento.

   \subsection{Sistemas operativos}

      \subsubsection{Sistema operativo de desarrollo}

      \paragraph{}Para el desarrollo del sistema, incluyendo la documentación,
      se va a disponer del siguiente sistema operativo:

      \begin{itemize}
         \item Nombre: Debian GNU/Linux \cite{debian}.
         \item Versión: 6.0 (Squeeze).
         \item Descripción: Sistema operativo basado en software libre.
         \item Disponibilidad: Toda la duración del proyecto.
         \item Tiempo utilización: Toda la duración del proyecto.
      \end{itemize}

      \subsubsection{Sistema operativo servidor de aplicaciones}

      \paragraph{}El servidor que actuará como sistema de control de versiones
      utiliza el mismo sistema operativo pero una versión inferior, 5.0 (Lenny),
      con las mismas condiciones de disponibilidad y tiempo de utilización que
      el expuesto anteriormente.

   \subsection{Editores}

      \subsubsection{Editores de documentación}

      \paragraph{}A la hora de generar la documentación, se hará uso de la
      aplicación Latex, que se detalla a continuación:

      \begin{itemize}
         \item Nombre: Latex \cite{latex}.
         \item Versión: LaTeX2e.
         \item Descripción: Lenguaje de marcado y preparación de documentos.
         \item Disponibilidad: Toda la duración del proyecto.
         \item Tiempo utilización: Durante la documentación.
         \item Justificación: Son varios los motivos para realizar la
         documentación en este lenguaje, entre los que destacan:
            \begin{itemize}
               \item Permite estructurar documentos fácilmente.
               \item Alta calidad tipográfica.
               \item Gran cantidad de paquetes externos disponibles que
               potencian el lenguaje.
               \item Posibilidad de realizar un mantenimiento de los elementos
               generados a través del control de versiones.
            \end{itemize}
      \end{itemize}

      \paragraph{}Para generar el código fuente que necesita Latex, se utilizará
      un editor, que se expone seguidamente:

      \begin{itemize}
         \item Nombre: Kile \cite{kile}.
         \item Versión: 2.0.3.
         \item Descripción: Editor de TeX/LaTeX.
         \item Disponibilidad: Toda la duración del proyecto.
         \item Tiempo de utilización: Durante la documentación.
         \item Justificación: Esta aplicación facilita la edición de texto en
         Latex.
      \end{itemize}

      \subsubsection{Editores de código}

      \paragraph{}ELEGIR.

   \subsection{Lenguajes de programación}

   \paragraph{}ELEGIR.

   \subsection{Servidores}

      \subsubsection{Sistema de control de versiones}

         \begin{itemize}
            \item Nombre: Subversion \cite{subversion}.
            \item Versión: 1.4.2.
            \item Descripción: Software de sistema de control de versiones.
            \item Disponibilidad: Toda la duración del proyecto.
            \item Tiempo de utilización: Toda la duración del proyecto.
         \end{itemize}

      \subsubsection{Sistema gestor de bases de datos}

         \begin{itemize}
          \item Nombre: MySQL \cite{mysql}.
          \item Versión: 5.0.51a.
          \item Descripción: Sistema de gestión de base de datos relacional,
          multihilo y multiusuario.
          \item Disponibilidad: Toda la duración del proyecto.
          \item Tiempo de utilización: Toda la duración del proyecto.
         \end{itemize}
