\section{Diccionario de datos}

  \paragraph{}Para concluir el análisis funcional, se especificará la
  transformación que los elementos sufren en el interior del sistema, así como
  todas las relaciones existentes entre ellos. Esto se realizará mediante otra
  técnica de descripción muy extendida como es el \textit{Diccionario de datos}
  (DD).

  \paragraph{}El Diccionario de datos es una lista organizada de los datos
  utilizados en el sistema y que gráficamente se encuentran presentes en los
  flujos de datos y en los almacenes del conjunto de los diagramas de flujo de
  datos. Esta técnica es utilizada para representar a cualquier dato elemental,
  independientemente de su complejidad, que ha sido o no considerado en la
  moderación de la información mediante las técnicas correspondientes. Su
  función es la especificación de cualquier elemento de información que es
  manejado por una función, definiéndose claramente de esta forma la interfaz
  de la misma.

  \paragraph{}El Diccionario de datos va a estar estructurado de la siguiente
  manera:

  \begin{description}
   \item[Nombre] Denominación del elemento de datos o de control, de la
                 base de datos, o de la entidad externa.
   \item[Alias]  Otras denominaciones utilizadas para la misma entrada.
   \item[Uso]    Relación de procesos que hacen uso del elemento de datos o de
                 control.
   \item[Descripción] Contenido de lo representado mediante una notación.
   \item[Información adicional] Otras anotaciones de interés.
  \end{description}

\subsection{Entidades externas}

  \subsubsection{Teclado}

  \paragraph{}prueba.

