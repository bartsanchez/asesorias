\section{Diagramas de flujo de datos}

  \paragraph{}Los diagramas de flujo de datos son una representación gráfica en
  forma de red que refleja el flujo de la información y las transformaciones que
  se aplican sobre ella al moverse desde la entrada hasta la salida de un
  sistema software. Un DFD representa qué funciones o qué transformaciones deben
  realizarse sobre los datos, pero no cuándo se realizan o en qué orden.

  \paragraph{}Los diagramas de flujo de datos ayudan a modelar las funciones que
  debe realizar el sistema, permitiendo una descomposición funcional del sistema
  en distintos niveles de detalle.

  \paragraph{}El refinamiento de estos diagramas se hace mediante niveles,
  comenzando por el nivel 0 o diagrama de contexto del sistema y finalizando en
  un nivel que ya no pueda descomponerse más debido a su sencillez.

  \paragraph{}Existen diferentes metodologías para la representación de los
  diagramas de flujo de datos, aquí se usará la metodología de Yourdon-DeMarco
  por ser una de las más extendidas.

  \paragraph{}La notación básica de esta metodología hace uso de los siguientes
  componentes: el proceso (que se clasifica en simple o compuesto), el flujo, el
  almacén y la entidad externa.

  \begin{description}
   \item[Proceso simple] El proceso simple se representa gráficamente mediante
        un círculo blanco y muestra una parte del sistema que transforma
        entradas en salidas: es decir, muestra cómo es que una o más entradas
        se transforman en salidas. Este tipo de proceso no se refinará o
        descompondrá más. La figura \ref{diagramaProcesoSimple}, muestra un
        ejemplo de proceso simple.

        \begin{figure}[!ht]
            \begin{center}
            \includegraphics[]{08.Analisis_Funcional/8.2.DFDs/Diagramas/proceso_simple.pdf}
            \caption{Ejemplo de proceso simple.}
            \label{diagramaProcesoSimple}
            \end{center}
         \end{figure}

   \item[Proceso compuesto] El proceso compuesto se representa gráficamente
        mediante un círculo gris y representa lo mismo que el proceso simple con
        la diferencia de que este proceso sí se refinará o descompondrá más en
        el siguiente nivel de abstracción. La figura
        \ref{diagramaProcesoCompuesto}, muestra un ejemplo de proceso compuesto.

        \begin{figure}[!ht]
            \begin{center}
            \includegraphics[]{08.Analisis_Funcional/8.2.DFDs/Diagramas/proceso_compuesto.pdf}
            \caption{Ejemplo de proceso compuesto.}
            \label{diagramaProcesoCompuesto}
            \end{center}
         \end{figure}

   \item[Flujo] El flujo se representa gráficamente mediante una flecha que
        entra o sale de un proceso, un almacén o una entidad externa. El flujo
        se utiliza para describir el movimiento de bloques o paquetes de
        información de una parte del sistema a otra.
   \item[Almacén] El almacén se representa gráficamente mediante dos líneas
        paralelas y se utiliza para modelar una colección de paquetes
        de datos en reposo.
   \item[Entidad externa] La entidad externa es representada gráficamente
        por un rectángulo. Es la fuente o el destino de la información que
        fluye por el sistema. Dicho de otra forma, es un productor o consumidor
        de información que reside fuera de los límites del sistema.
  \end{description}

\subsection{Nivel de abstracción 0: Diagrama de contexto}

  \paragraph{}En este diagrama se representará el funcionamiento de la
  aplicación de manera general. En él se detalla la interacción con un almacén
  de datos, denominado Copia de seguridad, en el que se almacenarán y
  recuperarán las copias de seguridad realizadas por el administrador principal,
  así como la interacción que mantiene el usuario con el sistema por medio de
  las entidades externas que generan flujos de entrada de información (Teclado,
  Ratón y Sistema) y los que reciben y procesan los flujos de salida de
  información del sistema (Pantalla, Impresora). A continuación se describirán
  las entidades externas que intervienen en el Diagrama de contexto.

  \begin{description}
   \item[Teclado] Permitirá al usuario introducir información en el sistema.

   \item[Ratón] Su función será la de permitir al usuario moverse por la
                aplicación.
   \item[Sistema] La información que proporcionará a la aplicación será la fecha
                  y hora del sistema en el que se encuentre instalada.
   \item[Pantalla] Recibirá los flujos de información de salida de la aplicación
                   y será la encargada de mostrarlos al usuario.
   \item[Impresora] Recibirá información de salida de la aplicación y se
                    encargará de imprimirla en papel.
  \end{description}

  \paragraph{}La figura \ref{diagramaContexto} muestra el Diagrama de contexto.

        \begin{figure}[!ht]
            \begin{center}
            \includegraphics[]{08.Analisis_Funcional/8.2.DFDs/Niveles/Diagramas/diagrama_contexto.pdf}
            \caption{Diagrama de contexto.}
            \label{diagramaContexto}
            \end{center}
         \end{figure}