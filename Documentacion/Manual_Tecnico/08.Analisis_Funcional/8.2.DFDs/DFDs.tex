\section{Diagramas de flujo de datos}

  \paragraph{}Los diagramas de flujo de datos son una representación gráfica en
  forma de red que refleja el flujo de la información y las transformaciones que
  se aplican sobre ella al moverse desde la entrada hasta la salida de un
  sistema software. Un DFD representa qué funciones o qué transformaciones deben
  realizarse sobre los datos, pero no cuándo se realizan o en qué orden.

  \paragraph{}Los diagramas de flujo de datos ayudan a modelar las funciones que
  debe realizar el sistema, permitiendo una descomposición funcional del sistema
  en distintos niveles de detalle.

  \paragraph{}El refinamiento de estos diagramas se hace mediante niveles,
  comenzando por el nivel 0 o diagrama de contexto del sistema y finalizando en
  un nivel que ya no pueda descomponerse más debido a su sencillez.

  \paragraph{}Existen diferentes metodologías para la representación de los
  diagramas de flujo de datos, aquí se usará la metodología de Yourdon-DeMarco
  por ser una de las más extendidas.

  \paragraph{}La notación básica de esta metodología hace uso de los siguientes
  componentes: el proceso (que se clasifica en simple o compuesto), el flujo, el
  almacén y la entidad externa.

  \begin{description}
   \item[Proceso simple] El proceso simple se representa gráficamente mediante
        un círculo blanco y muestra una parte del sistema que transforma
        entradas en salidas: es decir, muestra cómo es que una o más entradas
        se transforman en salidas. Este tipo de proceso no se refinará o
        descompondrá más.
   \item[Proceso compuesto] El proceso compuesto se representa gráficamente
        mediante un círculo gris y representa lo mismo que el proceso simple con
        la diferencia de que este proceso sí se refinará o descompondrá más en
        el siguiente nivel de abstracción.
   \item[Flujo] El flujo se representa gráficamente mediante una flecha que
        entra o sale de un proceso, un almacén o una entidad externa. El flujo
        se utiliza para describir el movimiento de bloques o paquetes de
        información de una parte del sistema a otra.
   \item[Almacén] El almacén se representa gráficamente mediante dos líneas
        paralelas y se utiliza para modelar una colección de paquetes
        de datos en reposo.
   \item[Entidad externa] La entidad externa es representada gráficamente
        por un rectángulo. Es la fuente o el destino de la información que
        fluye por el sistema. Dicho de otra forma, es un productor o consumidor
        de información que reside fuera de los límites del sistema.
  \end{description}
