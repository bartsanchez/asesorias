\subsection{Nivel de abstracción 0: Diagrama de contexto}

  \paragraph{}En este diagrama se representará el funcionamiento de la
  aplicación de manera general. En él se detalla la interacción con un almacén
  de datos, denominado Copia de seguridad, en el que se almacenarán y
  recuperarán las copias de seguridad realizadas por el administrador principal,
  así como la interacción que mantiene el usuario con el sistema por medio de
  las entidades externas que generan flujos de entrada de información (Teclado,
  Ratón y Sistema) y los que reciben y procesan los flujos de salida de
  información del sistema (Pantalla, Impresora). A continuación se describirán
  las entidades externas que intervienen en el Diagrama de contexto.

  \begin{description}
   \item[Teclado] Permitirá al usuario introducir información en el sistema.

   \item[Ratón] Su función será la de permitir al usuario moverse por la
                aplicación.
   \item[Sistema] La información que proporcionará a la aplicación será la fecha
                  y hora del sistema en el que se encuentre instalada.
   \item[Pantalla] Recibirá los flujos de información de salida de la aplicación
                   y será la encargada de mostrarlos al usuario.
   \item[Impresora] Recibirá información de salida de la aplicación y se
                    encargará de imprimirla en papel.
  \end{description}

  \paragraph{}La figura \ref{diagramaContexto} muestra el Diagrama de contexto.

        \begin{figure}[!ht]
            \begin{center}
            \includegraphics[]{08.Analisis_Funcional/8.2.DFDs/Niveles/Diagramas/diagrama_contexto.pdf}
            \caption{Diagrama de contexto.}
            \label{diagramaContexto}
            \end{center}
         \end{figure}