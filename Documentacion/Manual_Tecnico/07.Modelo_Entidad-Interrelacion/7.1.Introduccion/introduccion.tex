\section{Introducción}

   \paragraph{}En este sistema de software, la cantidad de información a manejar
   es muy elevada; por tanto, es importante realizar una descripción detallada
   de la misma. Para llevarla a cabo, se cree conveniente emplear una técnica
   denominada \textit{Modelo de datos}, la cual nos permite describir los
   elementos de la realidad que intervienen en este problema y la forma en que
   se relacionan estos elementos entre sí.

   \paragraph{}A través del Modelo de datos, se especificarán cada uno de los
   ítems de datos individuales que son aceptados por el sistema. Para cada uno
   de estos objetos, se determinarán sus propiedades y el dominio o tipo de
   datos básico en el cual pueden ser medidas, así como las restricciones o
   límites de los valores que pueden presentarse para cada una de ellas.
   Además, se especificarán detalladamente las relaciones existentes entre
   dichos objetos.

   \paragraph{}Posteriormente, con el objeto de intentar expresar
   conceptualmente toda la información obtenida mediante este proceso, se hará
   uso del \textit{Modelo Entidad-Interrelación}\footnote{El Modelo
   Entidad-Interrelación fue propuesto por Peter Chen a mediados de los años
   setenta para la representación conceptual de los problemas y como un medio
   para representar la visión de un sistema de forma global. Las características
   actuales de este modelo permiten la representación de cualquier tipo de
   sistema y a cualquier nivel de abstracción o refinamiento, lo cual da a lugar
   a que se aplique a la representación de problemas que vayan a ser
   tratados mediante un sistema computerizado o manual.}.
