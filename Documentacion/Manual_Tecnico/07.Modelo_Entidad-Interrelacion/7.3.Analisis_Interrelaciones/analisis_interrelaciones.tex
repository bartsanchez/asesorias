\section{Análisis de interrelaciones}

   \paragraph{}Una vez definidas las distintas entidades presentes en el
   modelo, a continuación se detallarán las distintas relaciones que mantienen
   entre ellas. Las interrelaciones por describir son las siguientes:

   \begin{itemize}
    \item Interrelación Administrador Centro - Centro.
    \item Interrelación Centro - Titulación.
    \item Interrelación Titulación - Asignatura.
    \item Interrelación Asignatura - Asignatura Curso Académico.
    \item Interrelación Alumno - Alumno Curso Académico.
    \item Interrelación Asignatura Curso Académico - Alumno Curso Académico.
    \item Interrelación Asesor - Asesor Curso Académico.
    \item Interrelación Departamento - Asesor Curso Académico.
    \item Interrelación Asesor Curso Académico - Alumno Curso Académico.
    \item Interrelación Alumno Curso Académico - Reunión.
    \item Interrelación Reunión - Pregunta Oficial.
    \item Interrelación Reunión - Pregunta Asesor.
    \item Interrelación Pregunta Oficial - Plantilla Entrevista Oficial.
    \item Interrelación Pregunta Asesor - Plantilla Entrevista Asesor.
    \item Interrelación Plantilla Entrevista Asesor - Asesor Curso Académico.
   \end{itemize}

   \paragraph{}Para la descripción de cada relación entre entidades, se
   utilizarán los siguientes apartados:

   \begin{description}
      \item[Definición] Especifica el motivo por el cual se produce la
           interrelación y entre qué tipos de entidad se da lugar.

      \item[Características] En este aparatado se mostrará la siguiente
           información:

            \begin{itemize}
             \item Nombre del tipo de interrelación.
             \item Tipo de la interrelación, es decir, fuerte o débil. En el
                   caso de que sea débil se especifica el tipo de debilidad:
                   existencia o identificación.
             \item Cardinalidad de la interrelación y cardinalidad con la que
                   cada tipo de entidad participa en la interrelación.
             \item Número de atributos del tipo de interrelación.
             \item Posibles restricciones que pueda tener la interrelación, en
                   el caso de que hubiera.
            \end{itemize}

      \item[Diagrama] Representación gráfica del tipo de interrelación.

      \item[Descripción de los atributos] Se describirán cada uno de los
           atributos que forman parte del tipo de interrelación. Para cada uno
           de ellos se indicará lo siguiente:

           \begin{itemize}
            \item Definición del atributo.
            \item Dominio en el cual se encuentra.
            \item Ejemplo práctico de cada atributo.
           \end{itemize}

      \item[Ejemplo práctico del tipo de interrelación] En este apartado se
           muestra una ocurrencia concreta del tipo de interrelación.
   \end{description}

\subsection{Interrelación Administrador Centro - Centro}

   \begin{description}
      \item[Definición] En esta interrelación se deja constancia de que pueden
      existir administradores de centro que administren los posibles centros
      que se establezcan en el sistema.

      \item[Características] La interrelación presenta las siguientes
                             características:

         \begin{itemize}
            \item \textbf{Nombre:} AC-C
            \item \textbf{Tipo de la interrelación:} Fuerte.
            \item \textbf{Cardinalidad de la interrelación:} N:M
                  \begin{itemize}
                     \item Administrador Centro: administra (0,n)
                     \item Centro: es\_administrado\_por (0,n)
                  \end{itemize}
            \item \textbf{Número de atributos:} Ninguno.
         \end{itemize}

      \item[Diagrama] La figura \ref{diagramaAC-C} muestra el diagrama de la
                      interrelación.
      \item \begin{figure}[!ht]
            \begin{center}
            \includegraphics[]{07.Modelo_Entidad-Interrelacion/7.3.Analisis_Interrelaciones/diagramas/AC-C.pdf}
            \caption{Diagrama de la interrelación AC-C.}
            \label{diagramaAC-C}
            \end{center}
         \end{figure}

      \item[Ejemplo práctico del tipo de interrelación]

      \item \begin{center}
            \begin{tabular}{ | r r | }
            \hline
            \multicolumn{2}{ | c | }{\textbf{Tipo de interrelación AC-C}} \\
            \hline
            \textbf{Administrador Centro} & \\
            id\_adm\_centro & 9 \\
            \hline
            \textbf{Centro} & \\
            id\_centro & 15 \\
            \hline
            \end{tabular}
         \end{center}

   \end{description}

\subsection{Interrelación Centro - Titulación}

   \begin{description}
      \item[Definición] En esta interrelación se deja constancia de que cada
      centro establecido en el sistema podrá disponer de varias titulaciones.

      \item[Características] La interrelación presenta las siguientes
                             características:

         \begin{itemize}
            \item \textbf{Nombre:} C-T
            \item \textbf{Tipo de la interrelación:} El tipo de entidad
                  Titulación es débil por identificación respecto al tipo de
                  entidad Centro.
            \item \textbf{Cardinalidad de la interrelación:} 1:N
                  \begin{itemize}
                     \item Centro: dispone\_de (0,n)
                     \item Titulación: pertenece\_a (1,1)
                  \end{itemize}
            \item \textbf{Número de atributos:} Ninguno.
         \end{itemize}

      \item[Diagrama] La figura \ref{diagramaC-T} muestra el diagrama de la
                      interrelación.
      \item \begin{figure}[!ht]
            \begin{center}
            \includegraphics[]{07.Modelo_Entidad-Interrelacion/7.3.Analisis_Interrelaciones/diagramas/C-T.pdf}
            \caption{Diagrama de la interrelación C-T.}
            \label{diagramaC-T}
            \end{center}
         \end{figure}

      \item[Ejemplo práctico del tipo de interrelación]

      \item \begin{center}
            \begin{tabular}{ | r r | }
            \hline
            \multicolumn{2}{ | c | }{\textbf{Tipo de interrelación C-T}} \\
            \hline
            \textbf{Centro} & \\
            id\_centro & 15 \\
            \hline
            \textbf{Titulación} & \\
            id\_centro & 15 \\
            id\_titulación & 3 \\
            \hline
            \end{tabular}
         \end{center}

   \end{description}

\subsection{Interrelación Titulación - Asignatura}

   \begin{description}
      \item[Definición] En esta interrelación se deja constancia de que cada
      titulación establecida en el sistema podrá disponer de varias asignaturas.

      \item[Características] La interrelación presenta las siguientes
                             características:

         \begin{itemize}
            \item \textbf{Nombre:} T-A
            \item \textbf{Tipo de la interrelación:} El tipo de entidad
                  Asignatura es débil por identificación respecto al tipo de
                  entidad Titulación.
            \item \textbf{Cardinalidad de la interrelación:} 1:N
                  \begin{itemize}
                     \item Titulación: dispone\_de (0,n)
                     \item Asignatura: pertenece\_a (1,1)
                  \end{itemize}
            \item \textbf{Número de atributos:} Ninguno.
         \end{itemize}

      \item[Diagrama] La figura \ref{diagramaT-A} muestra el diagrama de la
                      interrelación.

      \item \begin{figure}[!ht]
            \begin{center}
            \includegraphics[]{07.Modelo_Entidad-Interrelacion/7.3.Analisis_Interrelaciones/diagramas/T-A.pdf}
            \caption{Diagrama de la interrelación T-A.}
            \label{diagramaT-A}
            \end{center}
         \end{figure}

      \item[Ejemplo práctico del tipo de interrelación]

      \item \begin{center}
            \begin{tabular}{ | r r | }
            \hline
            \multicolumn{2}{ | c | }{\textbf{Tipo de interrelación T-A}} \\
            \hline
            \textbf{Titulación} & \\
            id\_centro & 15 \\
            id\_titulación & 3 \\
            \hline
            \textbf{Asignatura} & \\
            id\_centro & 15 \\
            id\_titulación & 3 \\
            id\_asignatura & 17 \\
            \hline
            \end{tabular}
         \end{center}
   \end{description}

\subsection{Interrelación Asignatura - Asignatura Curso Académico}

   \begin{description}
      \item[Definición] En esta interrelación se deja constancia de que una
      asignatura se desarrolla durante un determinado curso académico.

      \begin{itemize}
       \item Una \textit{Asignatura} puede disponer de varias \textit{Asignatura
             Curso Académico}.
       \item Una \textit{Asignatura Curso Académico} solamente puede pertenecer
             a una \textit{Asignatura}.
      \end{itemize}

      \item[Características] La interrelación presenta las siguientes
                             características:

         \begin{itemize}
            \item \textbf{Nombre:} A-ACA
            \item \textbf{Tipo de la interrelación:} El tipo de entidad
                  Asignatura Curso Académico es débil por identificación
                  respecto al tipo de entidad Asignatura.
            \item \textbf{Cardinalidad de la interrelación:} 1:N
                  \begin{itemize}
                     \item Asignatura: dispone\_de (0,n)
                     \item Asignatura Curso Académico: pertenece\_a (1,1)
                  \end{itemize}
            \item \textbf{Número de atributos:} Ninguno.
         \end{itemize}

      \item[Diagrama] La figura \ref{diagramaA-ACA} muestra el diagrama de la
                      interrelación.

      \item \begin{figure}[!ht]
            \begin{center}
            \includegraphics[]{07.Modelo_Entidad-Interrelacion/7.3.Analisis_Interrelaciones/diagramas/A-ACA.pdf}
            \caption{Diagrama de la interrelación A-ACA.}
            \label{diagramaA-ACA}
            \end{center}
         \end{figure}

      \item[Ejemplo práctico del tipo de interrelación]

      \item \begin{center}
            \begin{tabular}{ | r r | }
            \hline
            \multicolumn{2}{ | c | }{\textbf{Tipo de interrelación A-ACA}} \\
            \hline
            \textbf{Asignatura} & \\
            id\_centro & 15 \\
            id\_titulación & 3 \\
            id\_asignatura & 17 \\
            \hline
            \textbf{Asignatura Curso Académico} & \\
            id\_centro & 15 \\
            id\_titulación & 3 \\
            id\_asignatura & 17 \\
            curso\_académico & 2008 \\
            \hline
            \end{tabular}
         \end{center}
   \end{description}

\subsection{Interrelación Alumno - Alumno Curso Académico}

   \begin{description}
      \item[Definición] En esta interrelación se deja constancia de que un
      alumno puede estar matriculado durante un número indeterminado de cursos
      académicos.

      \item[Características] La interrelación presenta las siguientes
                             características:

         \begin{itemize}
            \item \textbf{Nombre:} A-AlCA
            \item \textbf{Tipo de la interrelación:} El tipo de entidad
                  Alumno Curso Académico es débil por identificación respecto al
                  tipo de entidad Alumno.
            \item \textbf{Cardinalidad de la interrelación:} 1:N
            \item \textbf{Número de atributos:} Ninguno.
         \end{itemize}

      \item[Diagrama] La figura \ref{diagramaA-AlCA} muestra el diagrama de la
                      interrelación.

      \item \begin{figure}[!ht]
            \begin{center}
            \includegraphics[]{07.Modelo_Entidad-Interrelacion/7.3.Analisis_Interrelaciones/diagramas/A-AlCA.pdf}
            \caption{Diagrama de la interrelación A-AlCA.}
            \label{diagramaA-AlCA}
            \end{center}
         \end{figure}

      \item[Ejemplo práctico del tipo de interrelación]

      \item \begin{center}
            \begin{tabular}{ | r r | }
            \hline
            \multicolumn{2}{ | c | }{\textbf{Tipo de interrelación A-AlCA}} \\
            \hline
            \textbf{Alumno} & \\
            dni\_pasaporte & 01234567A \\
            \hline
            \textbf{Alumno Curso Académico} & \\
            dni\_pasaporte & 01234567A \\
            id\_centro & 2008 \\
            \hline
            \end{tabular}
         \end{center}
   \end{description}

\subsection{Interrelación Asignatura Curso Académico - Alumno Curso Académico}

   \begin{description}
      \item[Definición] En esta interrelación se deja constancia de que un
      alumno está matriculado de un determinado número de asignaturas en un
      curso académico.

      \item[Características] La interrelación presenta las siguientes
                             características:

         \begin{itemize}
            \item \textbf{Nombre:} ACA-AlCA
            \item \textbf{Tipo de la interrelación:} Fuerte.
            \item \textbf{Cardinalidad de la interrelación:} N:M
                  \begin{itemize}
                     \item Asignatura Curso Académico: dispone\_de (0,n)
                     \item Alumno Curso Académico: matriculado\_en (0,n)
                  \end{itemize}
            \item \textbf{Número de atributos:} 1, nota.
         \end{itemize}

      \item[Diagrama] La figura \ref{diagramaACA-AlCA} muestra el diagrama de la
                      interrelación.

       \item \begin{figure}[!ht]
            \begin{center}
            \includegraphics[]{07.Modelo_Entidad-Interrelacion/7.3.Analisis_Interrelaciones/diagramas/ACA-AlCA.pdf}
            \caption{Diagrama de la interrelación ACA-AlCA.}
            \label{diagramaACA-AlCA}
            \end{center}
         \end{figure}

      \item[Descripción de los atributos] La interrelación presenta el siguiente
      atributo:

       \begin{itemize}
          \item \textbf{nota}
          \begin{itemize}
            \item \textbf{Definición:}Establece la calificación obtenida por un
            alumno en una asignatura durante un curso académico.
            \item \textbf{Dominio:} Conjunto de reales positivos.
            \item \textbf{Carácter:} Opcional.
            \item \textbf{Ejemplo práctico:} 8.4.
            \item \textbf{Información adicional:} El dato \textbf{¿¿QUIÉN LO INTRODUCE??}.
         \end{itemize}
       \end{itemize}

      \item[Ejemplo práctico del tipo de interrelación]

      \item \begin{center}
            \begin{tabular}{ | r r | }
            \hline
            \multicolumn{2}{ | c | }{\textbf{Tipo de interrelación ACA-AlCA}} \\
            \hline
            \textbf{Asignatura Curso Académico} & \\
            id\_centro & 15 \\
            id\_titulación & 3 \\
            id\_asignatura & 17 \\
            curso\_académico & 2008\\
            \hline
            \textbf{Alumno Curso Académico} & \\
            dni\_pasaporte & 01234567A \\
            curso\_académico & 2008 \\
            \hline
            \textbf{Atributos} & \\
            nota & 7,5 \\
            \hline
            \end{tabular}
         \end{center}
   \end{description}

\subsection{Interrelación Asesor - Asesor Curso Académico}

   \begin{description}
      \item[Definición] En esta interrelación se deja constancia de que un
      asesor puede ofrecer servicios de asesoría durante distintos cursos
      académicos.

      \begin{itemize}
       \item Un \textit{Asesor} puede asesorar a varios \textit{Asesor Curso Académico}.
       \item Un \textit{Asesor Curso Académico} es un \textit{Asesor}.
      \end{itemize}

      \item[Características] La interrelación presenta las siguientes
                             características:

         \begin{itemize}
            \item \textbf{Nombre:} Ase-AseCA
            \item \textbf{Tipo de la interrelación:} El tipo de entidad
                  Asesor Curso Académico es débil por identificación respecto al
                  tipo de entidad Asesor.
            \item \textbf{Cardinalidad de la interrelación:} 1:N
                  \begin{itemize}
                     \item Asesor: asesora (0,n)
                     \item Asesor Curso Académico: es\_un (1,1)
                  \end{itemize}
            \item \textbf{Número de atributos:} Ninguno.
         \end{itemize}

      \item[Diagrama] La figura \ref{diagramaAse-AseCA} muestra el diagrama de la
                      interrelación.

      \item \begin{figure}[!ht]
            \begin{center}
            \includegraphics[]{07.Modelo_Entidad-Interrelacion/7.3.Analisis_Interrelaciones/diagramas/Ase-AseCA.pdf}
            \caption{Diagrama de la interrelación Ase-AseCA.}
            \label{diagramaAse-AseCA}
            \end{center}
         \end{figure}

      \item[Ejemplo práctico del tipo de interrelación]

      \item \begin{center}
            \begin{tabular}{ | r r | }
            \hline
            \multicolumn{2}{ | c | }{\textbf{Tipo de interrelación Ase-AseCA}} \\
            \hline
            \textbf{Asesor} & \\
            dni\_pasaporte & 98765432Z \\
            \hline
            \textbf{Asesor Curso Académico} & \\
            dni\_pasaporte & 98765432Z \\
            id\_centro & 2008 \\
            \hline
            \end{tabular}
         \end{center}
   \end{description}

\subsection{Interrelación Departamento - Asesor Curso Académico}

   \begin{description}
      \item[Definición] En esta interrelación se deja constancia de que un
      asesor puede ofrecer servicios de asesoría perteneciendo a un departamento
      durante un determinado curso académico.

      \item[Características] La interrelación presenta las siguientes
                             características:

         \begin{itemize}
            \item \textbf{Nombre:} D-AseCA
            \item \textbf{Tipo de la interrelación:} El tipo de entidad
                  Asesor Curso Académico es débil por existencia respecto al
                  tipo de entidad Departamento.
            \item \textbf{Cardinalidad de la interrelación:} 1:N
                  \begin{itemize}
                     \item Departamento: dispone\_de (0,n)
                     \item Asesor Curso Académico: pertenece\_a (1,1)
                  \end{itemize}
            \item \textbf{Número de atributos:} Ninguno.
         \end{itemize}

      \item[Diagrama] La figura \ref{diagramaD-AseCA} muestra el diagrama de la
                      interrelación.

      \item \begin{figure}[!ht]
            \begin{center}
            \includegraphics[]{07.Modelo_Entidad-Interrelacion/7.3.Analisis_Interrelaciones/diagramas/D-AseCA.pdf}
            \caption{Diagrama de la interrelación D-AseCA.}
            \label{diagramaD-AseCA}
            \end{center}
         \end{figure}

      \item[Ejemplo práctico del tipo de interrelación]

      \item \begin{center}
            \begin{tabular}{ | r r | }
            \hline
            \multicolumn{2}{ | c | }{\textbf{Tipo de interrelación D-AseCA}} \\
            \hline
            \textbf{Departamento} & \\
            id\_departamento & 22 \\
            \hline
            \textbf{Asesor Curso Académico} & \\
            dni\_pasaporte & 98765432Z \\
            curso\_académico & 2008 \\
            \hline
            \end{tabular}
         \end{center}
   \end{description}

\subsection{Interrelación Asesor Curso Académico - Alumno Curso Académico}

   \begin{description}
      \item[Definición] En esta interrelación se deja constancia de que un
      asesor puede ofrecer servicios de asesoría a un número indeterminado de
      alumnos matriculados durante un determinado curso académico.

      \item[Características] La interrelación presenta las siguientes
                             características:

         \begin{itemize}
            \item \textbf{Nombre:} AseCA-AlCA
            \item \textbf{Tipo de la interrelación:} El tipo de entidad
                  Alumno Curso Académico es débil por existencia respecto al
                  tipo de entidad Asesor Curso Académico.
            \item \textbf{Cardinalidad de la interrelación:} 1:N
                  \begin{itemize}
                     \item Asesor Curso Académico: asesora\_a (0,n)
                     \item Alumno Curso Académico: es\_asesorado\_por (1,1)
                  \end{itemize}
            \item \textbf{Número de atributos:} Ninguno.
         \end{itemize}

      \item[Diagrama] La figura \ref{diagramaAseCA-AlCA} muestra el diagrama de
                      la interrelación.

       \item \begin{figure}[!ht]
            \begin{center}
            \includegraphics[]{07.Modelo_Entidad-Interrelacion/7.3.Analisis_Interrelaciones/diagramas/AseCA-AlCA.pdf}
            \caption{Diagrama de la interrelación AseCA-AlCA.}
            \label{diagramaAseCA-AlCA}
            \end{center}
         \end{figure}

      \item[Ejemplo práctico del tipo de interrelación]

      \item \begin{center}
            \begin{tabular}{ | r r | }
            \hline
            \multicolumn{2}{ | c | }{\textbf{Tipo de interrelación AseCA-AlCA}} \\
            \hline
            \textbf{Asesor Curso Académico} & \\
            dni\_pasaporte & 98765432Z \\
            id\_centro & 2007 \\
            \hline
            \textbf{Alumno Curso Académico} & \\
            dni\_pasaporte & 01234567A \\
            curso\_académico & 2008 \\
            \hline
            \end{tabular}
         \end{center}
   \end{description}

\subsection{Interrelación Alumno Curso Académico - Reunión}

   \begin{description}
      \item[Definición] En esta interrelación se deja constancia de que un
      alumno puede participar en reuniones con su asesor durante un determinado
      curso académico.

      \item[Características] La interrelación presenta las siguientes
                             características:

         \begin{itemize}
            \item \textbf{Nombre:} AlCA-R
            \item \textbf{Tipo de la interrelación:} El tipo de entidad Reunión
                  es débil por identificación respecto al tipo de entidad Alumno
                  Curso Académico.
            \item \textbf{Cardinalidad de la interrelación:} 1:N
                  \begin{itemize}
                     \item Alumno Curso Académico: participa\_en (0,n)
                     \item Reunión: realizada\_a (1,1)
                  \end{itemize}
            \item \textbf{Número de atributos:} Ninguno.
         \end{itemize}

      \item[Diagrama] La figura \ref{diagramaAlCA-R} muestra el diagrama de la
                      interrelación.

      \item \begin{figure}[!ht]
            \begin{center}
            \includegraphics[]{07.Modelo_Entidad-Interrelacion/7.3.Analisis_Interrelaciones/diagramas/AlCA-R.pdf}
            \caption{Diagrama de la interrelación AlCA-R.}
            \label{diagramaAlCA-R}
            \end{center}
         \end{figure}

      \item[Ejemplo práctico del tipo de interrelación]

      \item \begin{center}
            \begin{tabular}{ | r r | }
            \hline
            \multicolumn{2}{ | c | }{\textbf{Tipo de interrelación AlCA-R}} \\
            \hline
            \textbf{Alumno Curso Académico} & \\
            dni\_pasaporte & 01234567A \\
            curso\_académico & 2008 \\
            \hline
            \textbf{Reunión} & \\
            dni\_pasaporte & 98765432Z \\
            curso\_académico & 2008 \\
            id\_reunión & 121 \\
            \hline
            \end{tabular}
         \end{center}
   \end{description}

\subsection{Interrelación Reunión - Pregunta Oficial}

   \begin{description}
      \item[Definición] En esta interrelación se deja constancia de que una
      reunión puede estar compuesta por varias preguntas oficiales.

      \item[Características] La interrelación presenta las siguientes
                             características:

         \begin{itemize}
            \item \textbf{Nombre:} R-PO
            \item \textbf{Tipo de la interrelación:} Fuerte.
            \item \textbf{Cardinalidad de la interrelación:} N:M
                  \begin{itemize}
                     \item Reunión: compuesta\_por (0,n)
                     \item Pregunta Oficial: aparece\_en (0,n)
                  \end{itemize}
            \item \textbf{Número de atributos:} Uno: respuesta.
         \end{itemize}

      \item[Diagrama] La figura \ref{diagramaR-PO} muestra el diagrama de la
                      interrelación.

      \item \begin{figure}[!ht]
            \begin{center}
            \includegraphics[]{07.Modelo_Entidad-Interrelacion/7.3.Analisis_Interrelaciones/diagramas/R-PO.pdf}
            \caption{Diagrama de la interrelación R-PO.}
            \label{diagramaR-PO}
            \end{center}
         \end{figure}

      \item[Descripción de los atributos] La interrelación presenta el
      siguiente atributo:

       \begin{itemize}
        \item \textbf{respuesta}
          \begin{itemize}
            \item \textbf{Definición:} Establece la contestación del alumno a la
            pregunta realizada.
            \item \textbf{Dominio:} Conjunto de caracteres alfanuméricos.
            \item \textbf{Carácter:} Obligatorio.
            \item \textbf{Ejemplo práctico:} Antiguo alumno.
            \item \textbf{Información adicional:} El dato lo introduce el
            usuario alumno al contestar a la pregunta realizada.
         \end{itemize}
       \end{itemize}

      \item[Ejemplo práctico del tipo de interrelación]

      \item \begin{center}
            \begin{tabular}{ | r r | }
            \hline
            \multicolumn{2}{ | c | }{\textbf{Tipo de interrelación R-PO}} \\
            \hline
            \textbf{Reunión} & \\
            dni\_pasaporte & 98765432Z \\
            curso\_académico & 2008 \\
            id\_reunión & 121 \\
            \hline
            \textbf{Pregunta Oficial} & \\
            id\_entrevista\_oficial & 24 \\
            id\_pregunta\_oficial & 55 \\
            enunciado & ¿Quién le ha informado de esta carrera? \\
            \hline
            \textbf{Atributos} & \\
            respuesta & Antiguo alumno \\
            \hline
            \end{tabular}
         \end{center}
   \end{description}

\subsection{Interrelación Reunión - Pregunta Asesor}

   \begin{description}
      \item[Definición] En esta interrelación se deja constancia de que una
      reunión puede estar compuesta por varias preguntas de asesor.

      \item[Características] La interrelación presenta las siguientes
                             características:

         \begin{itemize}
            \item \textbf{Nombre:} R-PA
            \item \textbf{Tipo de la interrelación:} Fuerte.
            \item \textbf{Cardinalidad de la interrelación:} N:M
                  \begin{itemize}
                     \item Reunión: compuesta\_por (0,n)
                     \item Pregunta Oficial: aparece\_en (0,n)
                  \end{itemize}
            \item \textbf{Número de atributos:} Uno: respuesta.
         \end{itemize}

      \item[Diagrama] La figura \ref{diagramaR-PA} muestra el diagrama de la
                      interrelación.

      \item \begin{figure}[!ht]
            \begin{center}
            \includegraphics[]{07.Modelo_Entidad-Interrelacion/7.3.Analisis_Interrelaciones/diagramas/R-PA.pdf}
            \caption{Diagrama de la interrelación R-PA.}
            \label{diagramaR-PA}
            \end{center}
         \end{figure}

      \item[Descripción de los atributos] La interrelación presenta el
      siguiente atributo:

       \begin{itemize}
        \item \textbf{respuesta}
          \begin{itemize}
            \item \textbf{Definición:} Establece la contestación del alumno a la
            pregunta realizada.
            \item \textbf{Dominio:} Conjunto de caracteres alfanuméricos.
            \item \textbf{Carácter:} Obligatorio.
            \item \textbf{Ejemplo práctico:} Antiguo alumno.
            \item \textbf{Información adicional:} El dato lo introduce el
            usuario alumno al contestar a la pregunta realizada.
         \end{itemize}
       \end{itemize}

      \item[Ejemplo práctico del tipo de interrelación]

      \item \begin{center}
            \begin{tabular}{ | r r | }
            \hline
            \multicolumn{2}{ | c | }{\textbf{Tipo de interrelación R-PA}} \\
            \hline
            \textbf{Reunión} & \\
            dni\_pasaporte & 98765432Z \\
            curso\_académico & 2008 \\
            id\_reunión & 121 \\
            fecha & 01/01/2009 \\
            tipo & Individual \\
            \hline
            \textbf{Pregunta Asesor} & \\
            dni\_pasaporte & 98765432Z \\
            curso\_académico & 2008 \\
            id\_entrevista\_asesor & 36 \\
            id\_pregunta\_asesor & 72 \\
            enunciado & Nivel de inglés \\
            \hline
            \textbf{Pregunta Asesor} & \\
            respuesta & Alto \\
            \hline
            \end{tabular}
         \end{center}
   \end{description}
