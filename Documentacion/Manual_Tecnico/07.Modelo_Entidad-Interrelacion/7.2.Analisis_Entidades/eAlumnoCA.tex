\subsection{Tipo de entidad Alumno Curso Académico}

   \begin{description}

   \item[Definición] Se refiere al objeto del mundo real: \emph{``Estudiante de
   una titulación matriculado en cierta asignatura durante un curso académico''}.

   \item[Características] La entidad presenta las siguientes características:
      \begin{itemize}
         \item \textbf{Nombre:} Alumno Curso Académico.
         \item \textbf{Tipo:} Débil por identificación con respecto a Alumno.
         \item \textbf{Número de atributos:} 1 propio y 1 heredado.
         \item \textbf{Atributo/s identificador/es principal/es:} dni\_pasaporte y \\curso\_académico.
         \item \textbf{Atributo/s identificador/es alternativo/s:} -
         \item \textbf{Atributo/s heredado/s:} dni\_pasaporte del tipo
         de entidad Alumno.
      \end{itemize}

   \item[Diagrama] La figura \ref{diagramaAlumnoCA} muestra el diagrama de la entidad.
   \item \begin{figure}[!ht]
            \begin{center}
            \includegraphics[]{07.Modelo_Entidad-Interrelacion/7.2.Analisis_Entidades/diagramas/alumnoca.pdf}
            \caption{Diagrama de la entidad Alumno Curso Académico.}
            \label{diagramaAlumnoCA}
            \end{center}
         \end{figure}

   \item[Descripción de los atributos propios] Esta entidad presenta los
   siguientes atributos propios:

   \begin{itemize}
   \item \textbf{curso\_académico}
      \begin{itemize}
         \item \textbf{Definición:} Hace referencia al periodo de tiempo en que se imparte una determinada asignatura.
         \item \textbf{Dominio:} Formato de fecha: aaaa.
         \item \textbf{Carácter:}  Obligatorio.
         \item \textbf{Ejemplo práctico:} 2008.
         \item \textbf{Información adicional:} El dato se hereda del tipo de entidad Asignatura Curso Académico. Es la clave primaria junto con dni\_pasaporte.
      \end{itemize}
   \item \textbf{pregunta\_asesorX} ($\forall X \in N : 1 \leq X \leq 10$)
      \begin{itemize}
         \item \textbf{Definición:} Hace referencia a los atributos donde se
         almacenará la información de las preguntas que realice el asesor
         a los alumnos a través de las plantillas para entrevistas.
         \item \textbf{Dominio:} Conjunto de caracteres alfanuméricos.
         \item \textbf{Carácter:}  Opcional.
         \item \textbf{Ejemplo práctico:} Dominio de inglés.
         \item \textbf{Información adicional:} Conjunto de diez atributos
         (desde \textit{pregunta\_asesor1} hasta \textit{pregunta\_asesor10})
         que por motivos de redundancia, y teniendo en cuenta su naturaleza, se
         ha decidido detallar de manera uniforme. La información de cada uno
         de estos atributos se obtiene a través de las entrevistas que puede
         realizar el asesor a sus alumnos.
      \end{itemize}
   \item \textbf{respuesta\_alumnoX} ($\forall X \in N : 1 \leq X \leq 10$)
      \begin{itemize}
         \item \textbf{Definición:} Hace referencia a los atributos donde se
         almacenará la información de las respuestas de los alumnos a las
         preguntas realizadas por los asesores a través de las plantillas para
         entrevistas.
         \item \textbf{Dominio:} Conjunto de caracteres alfanuméricos.
         \item \textbf{Carácter:}  Opcional.
         \item \textbf{Ejemplo práctico:} Muy alto.
         \item \textbf{Información adicional:} Conjunto de diez atributos
         (desde \textit{respuesta\_alumno1} hasta \textit{respuesta\_alumno10})
         que por motivos de redundancia, y teniendo en cuenta su naturaleza, se
         ha decidido detallar de manera uniforme. La información de cada uno
         de estos atributos se obtiene a través de las entrevistas que puede
         realizar el asesor a sus alumnos.
      \end{itemize}
   \end{itemize}

   \item[Ejemplo práctico]

   \item \begin{center}
            \begin{tabular}{ | l | l | }
            \hline
            \multicolumn{2}{ | c | }{\textbf{Tipo de entidad Alumno Curso Académico}} \\
            \hline
            dni\_pasaporte & 01234567A \\
            \hline
            curso\_académico & 2008 \\
            \hline
            pregunta\_asesor1 & Nivel de inglés \\
            \hline
            \multicolumn{2}{ | c | }{.} \\
            \multicolumn{2}{ | c | }{.} \\
            \multicolumn{2}{ | c | }{.} \\
            \hline
            pregunta\_asesor10 & ¿Compagina trabajo y estudios? \\
            \hline
            respuesta\_alumno1 & Muy alto \\
            \hline
            \multicolumn{2}{ | c | }{.} \\
            \multicolumn{2}{ | c | }{.} \\
            \multicolumn{2}{ | c | }{.} \\
            \hline
            respuesta\_alumno10 & Sí \\
            \hline
            \end{tabular}
         \end{center}
   \end{description}
