\subsection{Tipo de entidad Titulación}

   \begin{description}

   \item[Definición] Se refiere al objeto del mundo real: \emph{``Conjunto de
        materias cuya superación supone la obtención de un título académico''}.

   \item[Características]

   \item \begin{center}
            \begin{tabular}{ | l | l | }
            \hline
            Nombre & Titulación \\
            \hline
            Tipo & Fuerte \\
            \hline
            Número de atributos & Dos \\
            \hline
            Atributo/s identificador/es principal/es & id\_titulacion \\
            \hline
            Atributo/s identificador/es alternativo/s & nombre\_titulacion \\
            \hline
            Atributo/s heredado/s & - \\
            \hline
            \end{tabular}
         \end{center}

   \item[Diagrama]

   \item[Descripción de los atributos]

   \item \begin{center}
            \begin{tabular}{ | l | p{10cm} | }
            \hline
            \multicolumn{2}{ | l | }{\textbf{id\_titulacion}} \\
            \hline
            Definición & Código que sirve como número identificativo para cada titulación dentro del sistema. \\
            \hline
            Dominio & Números naturales. \\
            \hline
            Carácter & Obligatorio. \\
            \hline
            Ejemplo práctico & 3 \\
            \hline
            Información adicional & El dato lo genera el sistema cuando el administrador introduce la titulación en el sistema. Es la clave primaria.\\
            \hline
            \end{tabular}
         \end{center}

   \item \begin{center}
            \begin{tabular}{ | l | p{10cm} | }
            \hline
            \multicolumn{2}{ | l | }{\textbf{nombre\_titulacion}} \\
            \hline
            Definición & Denominación de una titulación dentro del sistema. \\
            \hline
            Dominio & Conjunto de los caracteres del alfabeto. \\
            \hline
            Carácter & Obligatorio. \\
            \hline
            Ejemplo práctico & Ingeniería Técnica en Informática de Gestión. \\
            \hline
            Información adicional & El dato lo proporciona el administrador en el momento de introducir la titulación en el sistema. Es la clave alterna. \\
            \hline
            \end{tabular}
         \end{center}

   \item[Ejemplo práctico]

   \item \begin{center}
            \begin{tabular}{ | l | l | }
            \hline
            \multicolumn{2}{ | c | }{\textbf{Tipo de entidad Titulación}} \\
            \hline
            id\_titulacion & 3 \\
            \hline
            nombre\_titulacion & Ingeniería Técnica en Informática de Gestión. \\
            \hline
            \end{tabular}
         \end{center}
   \end{description}