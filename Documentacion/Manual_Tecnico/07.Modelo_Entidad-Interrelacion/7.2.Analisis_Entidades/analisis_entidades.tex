\section{Análisis de entidades}

   \subsection{Descripción de entidad}

   \paragraph{}Una entidad es un tipo de objeto definido mediante la agregación
   de una serie de atributos. Una entidad corresponde a la caracterización de
   objetos del mundo real, los cuales son definidos y diferenciados del resto de
   los objetos, sobre la base del conjunto de atributos que se agregan. Dentro
   de los atributos de una entidad, tenemos que destacar el identificador
   principal, el cual es uno o un conjunto de atributos que nos permite
   identificar de forma unívoca al tipo de entidad. Éste debe satisfacer la
   propiedad de ser mínimo, es decir, el conjunto de atributos identificadores
   será tal que si se eliminase algún atributo de este conjunto dejaría de
   cumplir la propiedad de identificador.

   \paragraph{}Para la descripción de cada tipo de entidad, se utilizarán los
   siguientes apartados:

   \begin{description}
      \item[Definición] Especifica la función o el papel que desempeña el tipo
           de entidad en el mundo real. Además, también se describirán las
           posibles dependencias que tenga el tipo de entidad con otros
           existentes en el sistema.

      \item[Características] En este aparatado se mostrará la siguiente
           información:

            \begin{itemize}
             \item Nombre de la entidad.
             \item Tipo de la entidad, es decir, fuerte, débil o subtipo de
                   entidad.
             \item Número de atributos.
             \item Atributo/s identificador/es principal/es.
             \item Atributo/s identificador/es alternativo/s.
             \item Atributo/s heredado/s.
            \end{itemize}

      \item[Diagrama] Representación gráfica del tipo de entidad.

      \item[Descripción de los atributos] Se describirán cada uno de los
           atributos que forman parte del tipo de entidad. Para cada uno de
           ellos se indicará lo siguiente:

           \begin{itemize}
            \item Definición del atributo.
            \item Dominio en el cual se encuentra.
            \item Carácter del atributo, indicando si es opcional u obligatorio.
            \item Ejemplo práctico de cada atributo.
            \item Información adicional que sea interesante comentar.
           \end{itemize}

      \item[Ejemplo práctico del tipo de entidad] En este apartado se muestra
           una ocurrencia concreta del tipo de entidad. A continuación, se
           describirán los tipos de entidades a tratar por el sistema,
           describiendo para cada una de ellas sus características principales.

   \end{description}