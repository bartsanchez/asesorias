\subsection{Tipo de entidad Asignatura}

   \begin{description}

   \item[Definición] Se refiere al objeto del mundo real: \emph{``Materia que
   forma parte del plan de estudios de una titulación''}.

   \item[Características]

   \item \begin{center}
            \begin{tabular}{ | l | p{6cm} | }
            \hline
            Nombre & Asignatura \\
            \hline
            Tipo & Débil por identificación respecto a Titulación \\
            \hline
            Número de atributos & 6 \\
            \hline
            Atributo/s identificador/es principal/es & id\_titulacion, id\_asignatura \\
            \hline
            Atributo/s identificador/es alternativo/s & id\_titulacion, nombre\_asignatura \\
            \hline
            Atributo/s heredado/s & id\_titulacion, del tipo de entidad Titulación \\
            \hline
            \end{tabular}
         \end{center}

   \item[Diagrama]

   \item[Descripción de los atributos]

   \item \begin{center}
            \begin{tabular}{ | l | p{10cm} | }
            \hline
            \multicolumn{2}{ | l | }{\textbf{id\_asignatura}} \\
            \hline
            Definición & Código que sirve como número identificativo para cada asignatura dentro del sistema. \\
            \hline
            Dominio & Números naturales. \\
            \hline
            Carácter & Obligatorio. \\
            \hline
            Ejemplo práctico & 17 \\
            \hline
            Información adicional & El dato lo genera el sistema cuando el administrador introduce la asignatura en el sistema. Es la clave primaria junto con id\_titulacion.\\
            \hline
            \end{tabular}
         \end{center}

   \item \begin{center}
            \begin{tabular}{ | l | p{10cm} | }
            \hline
            \multicolumn{2}{ | l | }{\textbf{nombre\_asignatura}} \\
            \hline
            Definición & Denominación de una asignatura dentro del sistema. \\
            \hline
            Dominio & Conjunto de caracteres alfanuméricos. \\
            \hline
            Carácter & Obligatorio. \\
            \hline
            Ejemplo práctico & Lenguajes de Inteligencia Artificial. \\
            \hline
            Información adicional & El dato lo proporciona el administrador en el momento de introducir la asignatura en el sistema. Es la clave alterna junto con id\_titulacion. \\
            \hline
            \end{tabular}
         \end{center}

   \item \begin{center}
            \begin{tabular}{ | l | p{10cm} | }
            \hline
            \multicolumn{2}{ | l | }{\textbf{curso}} \\
            \hline
            Definición & Nivel académico de la asignatura en una titulación. \\
            \hline
            Dominio & Números naturales. \\
            \hline
            Carácter & Optativo. \\
            \hline
            Ejemplo práctico & 2 \\
            \hline
            Información adicional & El dato lo proporciona el administrador en el momento de introducir la asignatura en el sistema. \\
            \hline
            \end{tabular}
         \end{center}

   \item \begin{center}
            \begin{tabular}{ | l | p{10cm} | }
            \hline
            \multicolumn{2}{ | l | }{\textbf{nCreditos}} \\
            \hline
            Definición & Valor relacionado con el número de horas que se estima necesario para dedicar a una asignatura. \\
            \hline
            Dominio & Números reales positivos. \\
            \hline
            Carácter & Optativo. \\
            \hline
            Ejemplo práctico & 4.50 \\
            \hline
            Información adicional & El dato lo proporciona el administrador en el momento de introducir la asignatura en el sistema. \\
            \hline
            \end{tabular}
         \end{center}

   \item \begin{center}
            \begin{tabular}{ | l | p{10cm} | }
            \hline
            \multicolumn{2}{ | l | }{\textbf{tipo}} \\
            \hline
            Definición & Clasificación de la asignatura según su tipo. \\
            \hline
            Dominio & Uno de los valores: Troncal, Obligatoria, Optativa o Libre Configuración. \\
            \hline
            Carácter & Optativo. \\
            \hline
            Ejemplo práctico & Optativa. \\
            \hline
            Información adicional & El dato lo proporciona el administrador en el momento de introducir la asignatura en el sistema. \\
            \hline
            \end{tabular}
         \end{center}

   \item[Ejemplo práctico]

   \item \begin{center}
            \begin{tabular}{ | l | l | }
            \hline
            \multicolumn{2}{ | c | }{\textbf{Tipo de entidad Asignatura}} \\
            \hline
            id\_titulacion & 3 \\
            \hline
            id\_asignatura & 17 \\
            \hline
            nombre\_asignatura & Lenguajes de Inteligencia Artificial \\
            \hline
            curso & 2 \\
            \hline
            nCreditos & 4.50 \\
            \hline
            tipo & Optativa \\
            \hline
            \end{tabular}
         \end{center}
   \end{description}
