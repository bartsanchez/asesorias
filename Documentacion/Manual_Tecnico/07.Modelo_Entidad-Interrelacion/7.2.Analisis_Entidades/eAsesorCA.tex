\subsection{Tipo de entidad Asesor Curso Académico}

   \begin{description}

   \item[Definición] Se refiere al objeto del mundo real: \emph{``Asesor
   que realiza labor de tutoría durante un curso académico''}.

   \item[Características] La entidad presenta las siguientes características:
      \begin{itemize}
         \item \textbf{Nombre:} Asesor Curso Académico.
         \item \textbf{Tipo:} Débil por identificación con respecto a Asesor y débil por existencia respecto a Departamento.
         \item \textbf{Número de atributos:} 2 propios y 2 heredados.
         \item \textbf{Atributo/s identificador/es principal/es:} dni\_pasaporte\\ y curso\_académico.
         \item \textbf{Atributo/s identificador/es alternativo/s:} -
         \item \textbf{Atributo/s heredado/s:} id\_departamento, del tipo de entidad Departamento y dni\_pasaporte, del tipo de entidad Asesor.
      \end{itemize}

   \item[Diagrama] La figura \ref{diagramaAsesorCA} muestra el diagrama de la entidad.
   \item \begin{figure}[!ht]
            \begin{center}
            \includegraphics[]{07.Modelo_Entidad-Interrelacion/7.2.Analisis_Entidades/diagramas/asesorca.pdf}
            \caption{Diagrama de la entidad Asesor Curso Académico.}
            \label{diagramaAsesorCA}
            \end{center}
         \end{figure}

   \item[Descripción de los atributos propios] Esta entidad presenta los
   siguientes atributos propios:

   \begin{itemize}
   \item \textbf{curso\_académico}
      \begin{itemize}
         \item \textbf{Definición:} Hace referencia al periodo de tiempo en que se imparte una determinada asignatura.
         \item \textbf{Dominio:} Formato de fecha: aaaa.
         \item \textbf{Carácter:}  Obligatorio.
         \item \textbf{Ejemplo práctico:} 2008.
         \item \textbf{Información adicional:} El dato representa el primer año
         del curso académico. Es la clave primaria junto con dni\_pasaporte.
      \end{itemize}
   \item \textbf{observaciones}
      \begin{itemize}
         \item \textbf{Definición:} Información extra que pueda ser necesario
         conocer.
         \item \textbf{Dominio:} Conjunto de caracteres alfanuméricos.
         \item \textbf{Carácter:}  Opcional.
         \item \textbf{Ejemplo práctico:} Ejemplo de observación.
         \item \textbf{Información adicional:} El dato lo proporicona el usuario
         asesor al modificar su información personal.
      \end{itemize}
   \end{itemize}

   \item[Ejemplo práctico]

   \item \begin{center}
            \begin{tabular}{ | l | l | }
            \hline
            \multicolumn{2}{ | c | }{\textbf{Tipo de entidad Asesor Curso Académico}} \\
            \hline
            id\_departamento & 22 \\
            \hline
            dni\_pasaporte & 98765432Z \\
            \hline
            curso\_académico & 2008 \\
            \hline
            observaciones & Ejemplo de observación \\
            \hline
            \end{tabular}
         \end{center}
   \end{description}
