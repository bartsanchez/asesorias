\subsection{Tipo de entidad Alumno}

   \begin{description}

   \item[Definición] Se refiere al objeto del mundo real: \emph{``Estudiante de
        una titulación que recibe asesoría''}.

   \item[Características]

   \item \begin{center}
            \begin{tabular}{ | l | l | }
            \hline
            Nombre & Alumno \\
            \hline
            Tipo & Fuerte \\
            \hline
            Número de atributos & 20 \\
            \hline
            Atributo/s identificador/es principal/es & dni\_pasaporte \\
            \hline
            Atributo/s identificador/es alternativo/s & correo\_electronico \\
            \hline
            Atributo/s heredado/s & - \\
            \hline
            \end{tabular}
         \end{center}

   \item[Diagrama]

   \item[Descripción de los atributos]

   \item \begin{center}
            \begin{tabular}{ | l | p{10cm} | }
            \hline
            \multicolumn{2}{ | l | }{\textbf{dni\_pasaporte}} \\
            \hline
            Definición & Coincide con el documento nacional de identidad o pasaporte de la persona. \\
            \hline
            Dominio & Números naturales. \\
            \hline
            Carácter & Obligatorio. \\
            \hline
            Ejemplo práctico & 01234567A \\
            \hline
            Información adicional & El dato lo genera el sistema cuando un usuario alumno es registrado en el sistema. Es la clave primaria.\\
            \hline
            \end{tabular}
         \end{center}

   \item \begin{center}
            \begin{tabular}{ | l | p{10cm} | }
            \hline
            \multicolumn{2}{ | l | }{\textbf{nombre\_alumno}} \\
            \hline
            Definición & Designa el nombre de pila del usuario alumno que interviene en el sistema. \\
            \hline
            Dominio & Conjunto de caracteres alfanuméricos. \\
            \hline
            Carácter & Obligatorio. \\
            \hline
            Ejemplo práctico & Bartolomé \\
            \hline
            Información adicional & El dato lo proporciona el propio usuario alumno, bien a la hora de registrarse o bien cuando modifica su información personal. \\
            \hline
            \end{tabular}
         \end{center}

   \item \begin{center}
            \begin{tabular}{ | l | p{10cm} | }
            \hline
            \multicolumn{2}{ | l | }{\textbf{apellidos\_alumno}} \\
            \hline
            Definición & Hace referencia a los apellidos del usuario alumno que interviene en el sistema. \\
            \hline
            Dominio & Conjunto de caracteres alfanuméricos. \\
            \hline
            Carácter & Obligatorio. \\
            \hline
            Ejemplo práctico & Sánchez Salado \\
            \hline
            Información adicional & El dato lo proporciona el propio usuario alumno, bien a la hora de registrarse o bien cuando modifica su información personal. \\
            \hline
            \end{tabular}
         \end{center}

   \item \begin{center}
            \begin{tabular}{ | l | p{10cm} | }
            \hline
            \multicolumn{2}{ | l | }{\textbf{correo\_electronico}} \\
            \hline
            Definición & Contiene la dirección de correo electrónico del alumno. \\
            \hline
            Dominio & Conjunto de caracteres alfanuméricos permitidos en una dirección de correo electrónico. \\
            \hline
            Carácter & Opcional. \\
            \hline
            Ejemplo práctico & i42sasab@uco.es \\
            \hline
            Información adicional & El dato lo proporciona el propio usuario alumno, bien a la hora de registrarse o bien cuando modifica su información personal. \\
            \hline
            \end{tabular}
         \end{center}

   \item \begin{center}
            \begin{tabular}{ | l | p{10cm} | }
            \hline
            \multicolumn{2}{ | l | }{\textbf{fecha\_nacimiento}} \\
            \hline
            Definición & Contiene la fecha de nacimiento del alumno. \\
            \hline
            Dominio & Formato de fecha: dd-mm-aaaa \\
            \hline
            Carácter & Opcional. \\
            \hline
            Ejemplo práctico & 13-12-1984 \\
            \hline
            Información adicional & El dato lo proporciona el propio usuario alumno, bien a la hora de registrarse o bien cuando modifica su información personal. \\
            \hline
            \end{tabular}
         \end{center}

   \item \begin{center}
            \begin{tabular}{ | l | p{10cm} | }
            \hline
            \multicolumn{2}{ | l | }{\textbf{direccion\_cordoba}} \\
            \hline
            Definición & Hace referencia a la dirección en la ciudad de Córdoba del alumno. \\
            \hline
            Dominio & Conjunto de caracteres alfanuméricos. \\
            \hline
            Carácter & Opcional. \\
            \hline
            Ejemplo & 13 Rue del Percebe \\
            \hline
            Información adicional & El dato lo proporciona el propio usuario alumno, bien a la hora de registrarse o bien cuando modifica su información personal. \\
            \hline
            \end{tabular}
         \end{center}

   \item \begin{center}
            \begin{tabular}{ | l | p{10cm} | }
            \hline
            \multicolumn{2}{ | l | }{\textbf{telefono}} \\
            \hline
            Definición & Hace referencia a un número de teléfono perteneciente al alumno. \\
            \hline
            Dominio & Conjunto de enteros positivos. \\
            \hline
            Carácter & Opcional. \\
            \hline
            Ejemplo & 601234567 \\
            \hline
            Información adicional & El dato lo proporciona el propio usuario alumno, bien a la hora de registrarse o bien cuando modifica su información personal. \\
            \hline
            \end{tabular}
         \end{center}

   \item \begin{center}
            \begin{tabular}{ | l | p{10cm} | }
            \hline
            \multicolumn{2}{ | l | }{\textbf{residencia}} \\
            \hline
            Definición & Hace referencia a la residencia del alumo durante el curso. \\
            \hline
            Dominio & Conjunto de caracteres alfanuméricos. \\
            \hline
            Carácter & Opcional. \\
            \hline
            Ejemplo & \textit{CREO QUE ÉSTE LO QUITO} \\
            \hline
            Información adicional & El dato lo proporciona el propio usuario alumno, bien a la hora de registrarse o bien cuando modifica su información personal. \\
            \hline
            \end{tabular}
         \end{center}

   \item \begin{center}
            \begin{tabular}{ | l | p{10cm} | }
            \hline
            \multicolumn{2}{ | l | }{\textbf{direccion\_familiar}} \\
            \hline
            Definición & Hace referencia a la dirección del domicilio familiar del alumno. \\
            \hline
            Dominio & Conjunto de caracteres alfanuméricos. \\
            \hline
            Carácter & Opcional. \\
            \hline
            Ejemplo & Calle Edsger Dijkstra, 30 \\
            \hline
            Información adicional & El dato lo proporciona el propio usuario alumno, bien a la hora de registrarse o bien cuando modifica su información personal. \\
            \hline
            \end{tabular}
         \end{center}

   \item \begin{center}
            \begin{tabular}{ | l | p{10cm} | }
            \hline
            \multicolumn{2}{ | l | }{\textbf{localidad\_familiar}} \\
            \hline
            Definición & Hace referencia a la localidad del domicilio familiar del alumno. \\
            \hline
            Dominio & Conjunto de caracteres alfanuméricos. \\
            \hline
            Carácter & Opcional. \\
            \hline
            Ejemplo & La Carlota \\
            \hline
            Información adicional & El dato lo proporciona el propio usuario alumno, bien a la hora de registrarse o bien cuando modifica su información personal. \\
            \hline
            \end{tabular}
         \end{center}

   \item \begin{center}
            \begin{tabular}{ | l | p{10cm} | }
            \hline
            \multicolumn{2}{ | l | }{\textbf{provincia\_familiar}} \\
            \hline
            Definición & Hace referencia a la provincia del domicilio familiar del alumno. \\
            \hline
            Dominio & Conjunto de caracteres alfanuméricos \\
            \hline
            Carácter & Opcional. \\
            \hline
            Ejemplo & Córdoba \\
            \hline
            Información adicional & El dato lo proporciona el propio usuario alumno, bien a la hora de registrarse o bien cuando modifica su información personal. \\
            \hline
            \end{tabular}
         \end{center}

   \item \begin{center}
            \begin{tabular}{ | l | p{10cm} | }
            \hline
            \multicolumn{2}{ | l | }{\textbf{codigo\_postal}} \\
            \hline
            Definición & Hace referencia al código postal de la localidad del domicilio familiar del alumno. \\
            \hline
            Dominio & Conjunto de enteros positivos \\
            \hline
            Carácter & Opcional. \\
            \hline
            Ejemplo & 14100 \\
            \hline
            Información adicional & El dato lo proporciona el propio usuario alumno, bien a la hora de registrarse o bien cuando modifica su información personal. \\
            \hline
            \end{tabular}
         \end{center}

   \item \begin{center}
            \begin{tabular}{ | l | p{10cm} | }
            \hline
            \multicolumn{2}{ | l | }{\textbf{telefono\_familiar}} \\
            \hline
            Definición & Hace referencia al teléfono del domicilio familiar del alumno. \\
            \hline
            Dominio & Conjunto de enteros positivos \\
            \hline
            Carácter & Opcional. \\
            \hline
            Ejemplo & 957123456 \\
            \hline
            Información adicional & El dato lo proporciona el propio usuario alumno, bien a la hora de registrarse o bien cuando modifica su información personal. \\
            \hline
            \end{tabular}
         \end{center}

   \item \begin{center}
            \begin{tabular}{ | l | p{10cm} | }
            \hline
            \multicolumn{2}{ | l | }{\textbf{estudios\_curso\_anterior}} \\
            \hline
            Definición & Hace referencia a los estudios realizados por el alumno el curso anterior. \\
            \hline
            Dominio & Conjunto de caracteres alfanuméricos. \\
            \hline
            Carácter & Opcional. \\
            \hline
            Ejemplo & ¿?. \\
            \hline
            Información adicional & El dato lo proporciona el propio usuario alumno, bien a la hora de registrarse o bien cuando modifica su información personal. \\
            \hline
            \end{tabular}
         \end{center}

   \item \begin{center}
            \begin{tabular}{ | l | p{10cm} | }
            \hline
            \multicolumn{2}{ | l | }{\textbf{estudios\_actuales}} \\
            \hline
            Definición & Hace referencia a los estudios realizados actualmente por el alumno. \\
            \hline
            Dominio & Conjunto de caracteres alfanuméricos. \\
            \hline
            Carácter & Opcional. \\
            \hline
            Ejemplo & Ingeniería Técnica en Informática de Gestión. \\
            \hline
            Información adicional & El dato lo proporciona el propio usuario alumno, bien a la hora de registrarse o bien cuando modifica su información personal. \\
            \hline
            \end{tabular}
         \end{center}

   \item \begin{center}
            \begin{tabular}{ | l | p{10cm} | }
            \hline
            \multicolumn{2}{ | l | }{\textbf{curso}} \\
            \hline
            Definición & Hace referencia al curso de una titulación en el que se encuentra actualmente  el alumno. \\
            \hline
            Dominio & Conjunto de enteros positivos. \\
            \hline
            Carácter & Opcional. \\
            \hline
            Ejemplo & 3. \\
            \hline
            Información adicional & El dato lo proporciona el propio usuario alumno, bien a la hora de registrarse o bien cuando modifica su información personal. \\
            \hline
            \end{tabular}
         \end{center}

   \item \begin{center}
            \begin{tabular}{ | l | p{10cm} | }
            \hline
            \multicolumn{2}{ | l | }{\textbf{ingreso}} \\
            \hline
            Definición & Hace referencia al año de ingreso en la Universidad por parte del alumno. \\
            \hline
            Dominio & Formato de fecha: aaaa \\
            \hline
            Carácter & Opcional. \\
            \hline
            Ejemplo & 2004. \\
            \hline
            Información adicional & El dato lo proporciona el propio usuario alumno, bien a la hora de registrarse o bien cuando modifica su información personal. \\
            \hline
            \end{tabular}
         \end{center}

   \item \begin{center}
            \begin{tabular}{ | l | p{10cm} | }
            \hline
            \multicolumn{2}{ | l | }{\textbf{otros\_estudios\_universitarios}} \\
            \hline
            Definición & Hace referencia a otros estudios universitarios que posea el alumno. \\
            \hline
            Dominio & Conjunto de caracteres alfanuméricos. \\
            \hline
            Carácter & Opcional. \\
            \hline
            Ejemplo & - \\
            \hline
            Información adicional & El dato lo proporciona el propio usuario alumno, bien a la hora de registrarse o bien cuando modifica su información personal. \\
            \hline
            \end{tabular}
         \end{center}

   \item \begin{center}
            \begin{tabular}{ | l | p{10cm} | }
            \hline
            \multicolumn{2}{ | l | }{\textbf{modalidad\_acceso\_universidad}} \\
            \hline
            Definición & Hace referencia al modo en que el alumno accedió a la Universidad. \\
            \hline
            Dominio & Conjunto de caracteres alfanuméricos. \\
            \hline
            Carácter & Opcional. \\
            \hline
            Ejemplo & Selectividad \\
            \hline
            Información adicional & El dato lo proporciona el propio usuario alumno, bien a la hora de registrarse o bien cuando modifica su información personal. \\
            \hline
            \end{tabular}
         \end{center}

   \item \begin{center}
            \begin{tabular}{ | l | p{10cm} | }
            \hline
            \multicolumn{2}{ | l | }{\textbf{calificacion\_acceso}} \\
            \hline
            Definición & Hace referencia a la calificación obtenida en las distintas modalidades de acceso a la Universidad por parte del alumno. \\
            \hline
            Dominio & Conjunto de reales positivos. \\
            \hline
            Carácter & Opcional. \\
            \hline
            Ejemplo & 7.2 \\
            \hline
            Información adicional & El dato lo proporciona el propio usuario alumno, bien a la hora de registrarse o bien cuando modifica su información personal. \\
            \hline
            \end{tabular}
         \end{center}

   \item[Ejemplo práctico]

   \item \begin{center}
            \begin{tabular}{ | l | l | }
            \hline
            \multicolumn{2}{ | c | }{\textbf{Tipo de entidad Alumno}} \\
            \hline
            dni\_pasaporte & 01234567A \\
            \hline
            nombre\_alumno & Bartolomé \\
            \hline
            apellidos\_alumno & Sánchez Salado \\
            \hline
            correo\_electronico & i42sasab@uco.es \\
            \hline
            fecha\_nacimiento & 13-12-1984 \\
            \hline
            direccion\_cordoba & 13 Rue del Percebe \\
            \hline
            telefono & 601234567 \\
            \hline
            residencia & \textit{CREO QUE ÉSTE LO QUITO} \\
            \hline
            direccion\_familiar & Calle Edsger Dijkstra, 30 \\
            \hline
            localidad\_familiar & La Carlota \\
            \hline
            provincia\_familiar & Córdoba \\
            \hline
            codigo\_postal & 14100 \\
            \hline
            telefono\_familiar & 957123456 \\
            \hline
            estudios\_curso\_anterior & ¿? \\
            \hline
            estudios\_actuales & Ingeniería Técnica en Informática de Gestión \\
            \hline
            curso & 3 \\
            \hline
            ingreso & 2004 \\
            \hline
            otros\_estudios\_universitarios & - \\
            \hline
            modalidad\_acceso\_universidad & Selectividad \\
            \hline
            calificacion\_acceso & 7.2 \\
            \hline
            \end{tabular}
         \end{center}
   \end{description}