\subsection{Factores Iniciales}

\paragraph{}Observando la naturaleza del problema que se pretende resolver, se
tienen en cuenta los siguientes factores iniciales:

\begin{itemize}
   \item Se deberá realizar un sistema software que gestione la información
   personal de los alumnos asesorados.
   \item Este mismo sistema debe ser capaz además de permitir a los asesores
   llevar cierto seguimiento sobre los estudiantes.
   \item La nueva aplicación deberá llevar a cabo un mantenimiento de toda la
   información personal necesaria referente a los alumnos, que será:
   \begin{itemize}
      \item Datos Personales.
      \begin{itemize}
         \item \textit{¿Fotografía.?}
         \item D.N.I.
         \item Fecha de nacimiento.
         \item Nombre.
         \item Apellidos.
         \item Dirección en Córdoba.
         \item Teléfonos.
         \item Correo electrónico.
         \item Residencia durante el curso.
      \end{itemize}
      \item Datos Familiares.
      \begin{itemize}
         \item Dirección familiar.
         \item Localidad.
         \item Código postal.
         \item Provicia.
         \item Teléfonos.
      \end{itemize}
      \item Datos Académicos.
      \begin{itemize}
         \item Estudios realizados en año pasado.
         \item Estudios que está realizando.
         \item Curso.
         \item Año de ingreso en la Universidad.
         \item Otros estudios universitarios.
         \item Modalidad de acceso a la universidad.
         \item Calificación de: acceso/estudios previos.
      \end{itemize}
   \end{itemize}
   \item Para llevar a cabo el seguimiento de los alumnos, el sistema deberá
   poder gestionar la siguiente información:
   \begin{itemize}
      \item Datos de la reunión.
      \begin{itemize}
         \item Asesor.
         \item Alumno (en caso de seguimiento individual).
         \item Asistentes (en caso de seguimiento grupal).
         \item Fecha.
         \item Hora.
         \item Duración.
      \end{itemize}
      \item Motivo.
      \item Orientaciones realizadas.
      \item Observaciones.
      \item Próxima reunión, temas a tratar.
      \begin{itemize}
         \item Fecha.
         \item Hora.
      \end{itemize}
   \end{itemize}
   \item La aplicación permitirá a los asesores generar entrevistas
   personalizadas, las cuales pueden ser complementadas con el uso de
   plantillas.
   \item Se permitirá realizar consultas de la información existente,
   mediante unos mecanismos de filtro.
   \item Deberá contar con la posibilidad de emitir informes de la
   información existente en el sistema de software, de carácter parametrizable.
   \item Estas emisiones deberán realizarse por algunos de los siguientes
   medios: pantalla, impresora o correo electrónico.
   \item Deberá permitir la realización copias de seguridad de la
   información existente en el sistema.
\end{itemize}
