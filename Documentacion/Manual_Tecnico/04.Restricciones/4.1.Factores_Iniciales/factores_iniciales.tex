\section{Factores Iniciales}

\paragraph{}Observando la naturaleza del problema que se pretende resolver, se
tienen en cuenta los siguientes factores iniciales:

\begin{itemize}
   \item Se deberá realizar un sistema de software que gestione la información
   relativa a la Asesoría Académica.
   \item Dicho sistema deberá tener en cuenta la información que proporciona el
   Vicerrectorado de Planificación y Calidad de la Universidad de Córdoba en
   relación a la asesoría académica.
   \item Además,el sistema debe ser accesible por red como servicio web, ya sea
   a través de una red local o internet; y, por tanto, multiplataforma.
   \item Para su funcionamiento interno, se hará uso de una base de datos que
   gestione la información existente.
   \item Por otra parte, permitirá la realización copias de seguridad de la
   información existente en el sistema.
   \item Cada usuario podrá acceder al sistema de forma personalizada; es decir,
   a través de un sistema de identificación personal. La información manejada
   por cada usuario es particular, y dependiente del rol que desempeñe.
   \item Las entrevistas generadas por los asesores se podrán hacer llegar a los
   alumnos mediante correo electrónico.
   \item La interfaz gráfica a usar deberá ser de fácil manejo e intuitiva,
   basada en elementos gráficos usados comúnmente, como por ejemplo: cuadros de
   texto, botones, etiquetas, etc.
\end{itemize}

\paragraph{}Además, se ha decidido utilizar un sistema de control de versiones
   para el desarrollo del proyecto, abarcando tanto la documentación como la
   generación de código, con el objetivo de facilitar la administración de los
   elementos que se generen a medida que evolucionan.

   \paragraph{}En este caso no se han estudiado alternativas, al existir con
   anterioridad un sistema de control de versiones implantado en los equipos
   que forman parte de los recursos hardware del proyecto.

   \paragraph{}La aplicación software encargada de este cometido se denomina
   Subversion \cite{subversion}, y entre sus características se encuentran:
   \begin{itemize}
      \item Es software libre.
      \item Existe bastante documentación y de fácil acceso.
      \item Se envían sólo las diferencias en ambas direcciones.
      \item Maneja eficientemente archivos binarios.
      \item Permite selectivamente el bloqueo de archivos.
   \end{itemize}

\paragraph{}Toda esta información se completará con más nivel de detalle en el
capítulo \ref{espReq}, \textit{Especificación de Requisitos}.
