\section{Factores estratégicos}\label{facEst}

\paragraph{}Con el objeto de determinar las características técnicas de
desarrollo, se ha realizado un estudio de los diferentes componentes que afectan
y conforman el proyecto a realizar, que servirá para determinar los distintos
factores estratégicos necesarios para la consecución de este proyecto. Dicho
estudio ha tenido en cuenta:

\begin{itemize}
 \item Sistema operativo.
 \item Servicios web.
 \item Sistema gestor de bases de datos.
 \item Sistema control de versiones.
 \item Otras decisiones.
\end{itemize}

\subsection{Sistema operativo}
   \paragraph{}Los distintos sistemas operativos que se han tenido en cuenta a
   la hora de elegir el sistema en el que se desarrollará el proyecto son:

   \begin{itemize}
    \item Debian GNU/Linux \cite{debian}.
      \begin{itemize}
         \item Ventajas
            \begin{itemize}
             \item Software libre.
             \item Gran estabilidad.
             \item Gran cantidad de software disponible.
             \item Amplia comunidad de usuarios.
             \item Basado en Unix.
             \item Disponibilidad del código fuente del sistema.
             \item Se distribuye gratuitamente.
            \end{itemize}
         \item Inconvenientes
            \begin{itemize}
            \item No tiene sustituto directo para determinadas aplicaciones.
            \item Algunos dispositivos de última generación no están soportados.
            \end{itemize}
      \end{itemize}
    \item Microsoft Windows XP/Vista \cite{microsoft}.
      \begin{itemize}
         \item Ventajas
            \begin{itemize}
             \item Ampliamente difundido.
             \item Facilidad de uso.
            \end{itemize}
         \item Inconvenientes
            \begin{itemize}
            \item Requerimientos técnicos.
            \item Coste respecto a otros sistemas.
            \item Estabilidad frente a otros sistemas.
            \end{itemize}
      \end{itemize}
   \end{itemize}

   \paragraph{} El sistema operativo que se utilizará para el desarrollo del
   proyecto será Debian GNU/Linux \cite{debian}, además de por las ventajas
   anteriormente señaladas, por ser éste el sistema habitual del desarrollador
   principal y estar claramente familiarizado con él.

\subsection{Servicios web}
   \paragraph{} Al tratarse de un servicio web, el contenido generado por la
   aplicación deberá hacer uso del lenguaje de marcado HTML, o alguna de sus
   variantes (XHTML), al ser el lenguaje de marcado estándar.

   \paragraph{}La elección del lenguaje del lado del servidor utilizado para el
   desarrollo de la aplicación, se ha ha llevado a cabo contemplando las
   siguientes alternativas:
   \begin{itemize}
    \item PHP \cite{php}.
      \begin{itemize}
         \item Ventajas
            \begin{itemize}
             \item Relativamente fácil de aprender.
             \item Lenguaje multiplataforma.
             \item Amplia documentación.
             \item Muy extensible a través de módulos.
             \item Software libre.
            \end{itemize}
         \item Inconvenientes
            \begin{itemize}
             \item Pobre seguimiento de errores.
             \item Limitaciones en la programación orientada a objetos.
             \item Mala legibilidad de código.
            \end{itemize}
      \end{itemize}
    \item Ruby \cite{ruby}.
      \begin{itemize}
         \item Ventajas
            \begin{itemize}
             \item Lenguaje multiplataforma.
             \item Muy extensible a través de módulos.
             \item Software libre.
            \end{itemize}
         \item Inconvenientes
            \begin{itemize}
             \item Pobre documentación extraoficial.
             \item Velocidad, al ser un lenguaje interpretado.
            \end{itemize}
      \end{itemize}
    \item Python \cite{python}.
      \begin{itemize}
         \item Ventajas
            \begin{itemize}
             \item Lenguaje de propósito general multiplataforma.
             \item Muy extensible a través de módulos.
             \item Software libre.
             \item Buena legibilidad de código.
            \end{itemize}
         \item Inconvenientes
            \begin{itemize}
             \item Velocidad, al ser un lenguaje interpretado.
            \end{itemize}
      \end{itemize}
   \end{itemize}

   \paragraph{}Se ha decidido utilizar Python por satisfacer las necesidades
   de la aplicación a desarrollar, teniendo en cuenta las ventajas anteriormente
   comentadas, así como por cumplir el deseo personal del desarrollador
   principal de dicha aplicación de aprender este lenguaje.

   \paragraph{}Además, con objeto de agilizar el desarrollo, se utilizará
   un \textit{framework} \footnote{Según Wikipedia \cite{wikipedia}:\textit{``Un
   framework, en el desarrollo de software, es una estructura de soporte
   definida, mediante la cual otro proyecto de software puede ser organizado y
   desarrollado. Típicamente, puede incluir soporte de programas, bibliotecas y
   un lenguaje interpretado entre otros software para ayudar a desarrollar y
   unir los diferentes componentes de un proyecto``.}} escrito en Python llamado
   Django \cite{django}.

\subsection{Sistema gestor de bases de datos}
   \paragraph{}En cuanto al funcionamiento de la base de datos, como sistema
   gestor se han estudiado dos en particular:

   \begin{itemize}
    \item MySQL \cite{mysql}.
      \begin{itemize}
         \item Ventajas
            \begin{itemize}
             \item Velocidad a la hora de realizar las operaciones.
             \item Bajo consumo de recursos.
             \item Buenas utilidades de administración.
             \item Poca probabilidad de corromper los datos.
            \end{itemize}
         \item Inconvenientes
            \begin{itemize}
             \item Carece de soporte para transacciones, \textit{rollback's} y
                   subconsulta.
             \item No maneja la integridad referencial.
             \item No implementa una buena escalabilidad.
            \end{itemize}
      \end{itemize}
    \item PostgreSQL \cite{postgresql}.
      \begin{itemize}
         \item Ventajas
            \begin{itemize}
             \item Implementa el uso de \textit{rollback's}, subconsultas y
                   transacciones.
             \item Tiene la capacidad de comprobar la integridad referencial.
             \item Recomendado por el \textit{framework} elegido (Django).
            \end{itemize}
         \item Inconvenientes
            \begin{itemize}
             \item Gran consumo de recursos.
             \item De 2 a 3 veces más lento que MySQL.
            \end{itemize}
      \end{itemize}
      \item Oracle \cite{oracle}.
      \begin{itemize}
         \item Ventajas
            \begin{itemize}
             \item Gran escalabilidad.
             \item Gran estabilidad.
             \item Implementa el uso de \textit{rollback's}, subconsultas y
                   transacciones.
             \item Tiene la capacidad de comprobar la integridad referencial.
             \item Buen soporte técnico.
            \end{itemize}
         \item Inconvenientes
            \begin{itemize}
             \item Licencia privativa.
            \end{itemize}
      \end{itemize}
   \end{itemize}

   \paragraph{}Debido a que las diferencias entre ambos sistemas gestores de
   bases de datos son prácticamente despreciables al nivel de desarrollo que
   se pretende llegar, se hará uso de PostreSQL, al ser el recomendado por el
   \textit{framework} utilizado.

\subsection{Sistema de control de versiones}
   \paragraph{} Se ha decidido utilizar un sistema de control de versiones para
   el desarrollo del proyecto, abarcando tanto la documentación como la
   generación de código, con el objetivo de facilitar la administración de los
   elementos que se generen a medida que evolucionan.

   \paragraph{}En este caso no se han estudiado alternativas, al existir con
   anterioridad un sistema de control de versiones implantado en los equipos
   que forman parte de los recursos hardware del proyecto.

   \paragraph{}La aplicación software encargada de este cometido se denomina
   Subversion \cite{subversion}, y entre sus características se encuentran:
   \begin{itemize}
      \item Es software libre.
      \item Existe bastante documentación y de fácil acceso.
      \item Se envían sólo las diferencias en ambas direcciones.
      \item Maneja eficientemente archivos binarios.
      \item Permite selectivamente el bloqueo de archivos.
   \end{itemize}

\subsection{Otras decisiones}

\paragraph{}En lo que respecta a funcionalidades de la aplicación, se han tomado
las siguientes decisiones:

\begin{itemize}
   \item Cada usuario podrá acceder al sistema de forma personalizada; es decir,
   a través de un sistema de identificación personal. La información manejada
   por cada usuario es particular, y dependiente del rol que desempeñe.
   \item Las entrevistas generadas por los asesores se podrán hacer llegar a los
   alumnos mediante correo electrónico.
   \item La interfaz gráfica a usar deberá ser de fácil manejo e intuitiva,
   basada en elementos gráficos usados comúnmente como por ejemplo: cuadros de
   texto, botones, etiquetas, etc.
\end{itemize}

