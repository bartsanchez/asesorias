\section{Factores estratégicos}\label{facEst}

\paragraph{}En cuanto a las características técnicas, se han decidido como
factores estratégicos, para la consecución de este proyecto, los siguientes:

\begin{itemize}
   \item El sistema operativo que se utilizará para el desarrollo del proyecto
   será Debian GNU/Linux 6.0 \cite{debian}, por ser éste el sistema habitual
   del desarrollador principal y estar claramente familiarizado con él.
   \item Se ha decidido utilizar un sistema de control de versiones para el
   desarrollo del proyecto, abarcando tanto la documentación como la generación
   de código, con el objetivo de facilitar la administración de los elementos
   que se generen a medida que evolucionan. La aplicación software encargada de
   este cometido se denomina Subversion \cite{subversion}, y ha sido elegida por
   varias razones, entre las que se encuentran:
   \begin{itemize}
      \item Es software libre.
      \item Es fácil de instalar en el sistema operativo sobre el que se
            trabaja.
      \item Existe bastante documentación y de fácil acceso.
   \end{itemize}
   \item Para el funcionamiento de la base de datos, se ha decidido utilizar
   MySQL 5.0 \cite{mysql}. Entre los motivos se encuentran el ser software
   libre y la fácil instalación/configuración en el sistema operativo sobre el
   que se desarrolla la aplicación.
   \item Al ser un servicio web, el contenido generado por la aplicación deberá
   hacer uso del lenguaje de marcado HTML, o alguna de sus variantes (XHTML), al
   ser el lenguaje de marcado estándar.
   \item Para la generación de contenido, se hará uso de un lenguaje del lado
   del servidor. Se ha decidido utilizar PHP/Ruby(On Rails)/Python(Django) por
   el motivo X.
\end{itemize}

\paragraph{}En lo que respecta a funcionalidades de la aplicación, se han tomado
las siguientes decisiones:

\begin{itemize}
   \item Cada usuario podrá acceder al sistema de forma personalizada; es decir,
   a través de un sistema de identificación personal. La información manejada
   por cada usuario es particular, y dependiente del rol que desempeñe.
   \item Las entrevistas generadas por los asesores se podrán hacer llegar a los
   alumnos mediante correo electrónico.
   \item La interfaz gráfica a usar deberá ser de fácil manejo e intuitiva,
   basada en elementos gráficos usados comúnmente como por ejemplo: cuadros de
   texto, botones, etiquetas, etc.
\end{itemize}

