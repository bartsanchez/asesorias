\paragraph{}El Consejo de Gobierno de la Universidad de Córdoba de 27 de junio
de 2008 acordó la aprobación del Reglamento regulador de la figura del Asesor
Académico\footnote{Acuerdo del Consejo de Gobierno de 27 de junio de 2008, por
el que aprueba el Reglamento regulador de la figura del Asesor Académico.
Universidad de Córdoba. Ver Apéndice \ref{a1}.}. En dicho reglamento se
establecen las competencias de los asesores académicos así como el proceso de
asignación de los estudiantes asesorados.

\paragraph{}De dicho reglamento se destacan las siguientes cuestiones:

\begin{itemize}
   \item \textit{``En cada Centro, bajo la responsabilidad de su Dirección,
   existirá una lista o registro de Asesores Académicos, en el que se podrán
   inscribir todos aquellos profesores adscritos al Centro que quieran ejercer
   funciones de asesoría académica.}
   \item \textit{``En el momento de formalizar su primera matrícula, a cada
   estudiante se le informará sobre la figura del Asesor Académico y se le
   asignará uno de los inscritos en la lista o registro del Centro.''}
   \item \textit{``Cada Asesor Académico tendrá a su cargo un máximo de 25
   estudiantes.''}
   \item \textit{``El Asesor tendrá a su cargo a los estudiantes que se le
   asignen durante el tiempo que permanezcan en la titulación, es decir, desde
   su ingreso hasta que finalicen sus estudios o los abandonen.''}
   \item \textit{``En la segunda y sucesivas matrículas, y antes de cada proceso
   de matriculación, el Asesor, por iniciativa de éste, tendrá una entrevista
   con sus estudiantes asignados. Asimismo, el Asesor deberá entrevistarse con
   los estudiantes asignados al menos una vez cada cuatrimestre.''}
   \item \textit{``El Asesor deberá aceptar las peticiones debidamente
   fundamentadas de entrevista por parte de sus estudiantes en cualquier momento
   del curso. Asimismo, podrá convocar a todos o a parte de ellos cuando lo crea
   conveniente.''}
   \item \textit{``El Asesor tendrá un historial de cada uno de sus estudiantes
   (incluyendo una fotografía reciente de los mismos y datos de contacto postal,
   telefónico y por Internet) en el que figurarán sus recomendaciones,
   comentarios y aquellos aspectos que considere de interés.''}

   \item \textit{``Los Asesores Académicos y los vicedecanos o
   subdirectores que ejerzan funciones de tutoría actuarán coordinadamente,
   siguiendo las directrices que al respecto apruebe el Director o Decano del
   Centro. Los Asesores se reunirán con la Dirección del Centro un mínimo de dos
   veces al año, para hacer un seguimiento de la labor realizada.``}

\end{itemize}