\section{Aplicaciones informáticas}

\paragraph{}Como aplicaciones informáticas que resuelvan, o traten de resolver,
el mismo problema que el planteado por el presente documento, se conoce
únicamente la aplicación \textit{Portal para la asesoría académica}
\cite{carmonaVaro}.

\paragraph{}Dicha aplicación, tiene su razón de ser como \textit{herramienta que
permita el almacenamiento y la gestión de las reuniones de asesoría, que sea de
gestionar la asignación de asesores y alumnos en la formación de grupos de
asesoría y que facilite la comunicación con estos,} (sic) \textit{permitiendo la
compartición de documentos de interés e información relevante para los alumnos y
facilitando información a los asesores sobre el seguimiento de los alumnos.
Además se persigue poder ofrecer tanto a alumnos como a asesores una serie de
encuestas para evaluar la actividad.}

\paragraph{}Para su consecución hace uso de un sistema de gestión de contenidos
(CMS, \textit{Content Management System}) denominado Joomla! \cite{joomla},
debido a que \textit{permiten descentralizar las labores del mantenimiento del
contenido de un portal, de forma que el personal no técnico de los distintos
departamentos de una empresa pueda añadir, editar y gestionar su propio
contenido en una web.}

\paragraph{}El principal problema de esta aplicación es que no resuelve el
problema de gestionar con eficacia la información relativa a las asesorías
académicas:

\begin{itemize}
 \item En ella no se tienen en cuenta ni los centros, ni las titulaciones, ni
       las asignaturas a las que pertenecen tanto asesores como alumnos, lo que
       provoca que la información generada resulte muy desorganizada, además de
       escasa. Por ejemplo, no es posible que un asesor conozca las asignaturas
       aprobadas o pendientes de un determinado alumno al que preste asesoría,
       algo esencial a la hora de poder orientarlo correctamente.
 \item Tampoco se tienen en cuenta diferentes cursos académicos, lo que provoca
       que la información siempre parezca actual, complicando un más que
       deseable estudio de la evolución temporal de los acontecimientos. Esto
       significa que un asesor, por ejemplo, pueda conocer cuántas asignaturas
       aprueba un alumno anualmente, para poder recomendarle mejor a la hora
       de efectuar la siguiente matriculación.
 \item Por otro lado, no es posible distinguir entre las preguntas oficiales
       proporcionadas por el Vicerrectorado para la asesoría académica de las
       posibles preguntas personales que pueda necesitar realizar un profesor.
       Por lo tanto, en dicha aplicación todas las preguntas son preguntas del
       sistema, por lo que los asesores no disponen de la libertad de generar
       sus propias preguntas personales para realizarlas a los alumnos a los que
       preste asesoría.
\end{itemize}