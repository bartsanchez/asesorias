\paragraph{}El Vicerrectorado de Planificación y Calidad de la Universidad de
Córdoba ha emitido una serie de fichas de información personal y de seguimiento,
individual y grupal, de los estudiantes asesorados con el objeto de que los
asesores tengan cierto control y organización durante la labor de asesoría.

\paragraph{}Esto supone que cada asesor es responsable directo de su propia
organización con respecto a la información relativa a los estudiantes
asesorados, pudiendo hacer uso de estas fichas, las cuales, sirven como guía.

\paragraph{}Si tenemos en cuenta que el Reglamento regulador de la figura del
Asesor Académico dispone que ``los asesores que ejerzan funciones de tutoría
actuarán coordinadamente'', se percibe como una mala idea que cada asesor se
organice a su manera.

\paragraph{}Para solucionar este problema, se hace necesario desarrollar un
sistema software que centralice toda la información que los asesores necesiten
disponer y que facilite la labor de asesoría con respecto a sus alumnos.
