\paragraph{}En este capítulo se van a proponer las posibles futuras mejoras
a las que puede ser sometida la aplicación desarrollada en el presente
proyecto fin de carrera. Las futuras mejoras son las siguientes:

\begin{itemize}
  \item Integrar el expediente académico de los alumnos en el sistema
  desarrollado. Para ello, será necesario realizar la conexión con la base de
  datos de Sigma \cite{sigma}, para lo cual se requeriría la autorización
  correspondiente de la Universidad de Córdoba.
  \item Mejorar la comunicación entre los distintos usuarios de la aplicación,
  sobre todo entre asesores y alumnos. Para ello, sería deseable que se pudieran
  dejar mensajes en el sistema para que, por ejemplo, se pudiera realizar
  convocatorias públicas a los alumnos.
  \item Organizar las preguntas de las plantillas por bloques. Esto facilitaría
  la organización y legibilidad de las entrevistas realizadas por los asesores
  a los alumnos. Se podrían agrupar por preguntas docentes, salidas
  profesionales, entorno social, etc.
  \item Mejorar la gestión de las preguntas de las plantillas, distinguiéndolas
  por el tipo de respuesta que tengan, por ejemplo:
  \begin{itemize}
    \item Numérica.
    \item Lógica (Sí o No).
    \item Descriptiva.
  \end{itemize}
  \item Permitir la inclusión en el sistema de plantillas de centro. De esta
  manera, sería más fácil y homogéneo la realización de entrevistas que afecten
  a un centro en particular.
\end{itemize}


