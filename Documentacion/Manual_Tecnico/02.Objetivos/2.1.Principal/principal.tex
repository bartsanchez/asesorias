\subsection{Principal}

\paragraph{}El objetivo principal de la realización de este proyecto será el
desarrollo e implantación de un sistema software que facilite a los asesores
académicos la gestión y mantenimiento de información relativa a los alumnos
asesorados. Para conseguirlo, se perseguirá la consecución de una serie de
características que dicho sistema pretende reunir, como son:

\begin{itemize}
   \item Permitir a los asesores gestionar a la información de sus alumnos de
         forma fácil e intuitiva.
   \item Permitir a los asesores realizar un seguimiento de cada alumno en
         particular.
   \item Ejecutar consultas, de forma que, a través de una serie de condiciones
         de entrada, se genere una salida que satisfaga dichos valores.
   \item La parametrización de todo documento emitido por la aplicación. Esta
         función del sistema va a ser de gran importancia, ya que permitirá que
         los documentos sean modificados o actualizados según sea conveniente.
   \item Generar informes de información personal de alumnos y su seguimiento,
         que puedan ser mostrados por pantalla o imprimidos por los asesores.
   \item Generar entrevistas, grupales o individuales, que puedan ser mostradas
         por pantalla e imprimidas por los asesores; las cuales, además, podrán
         ser recibidas por los alumnos a través del correo electrónico.
   \item Efectuar copias de seguridad, para salvaguardar cuando se desee la
         información existente en el sistema.
\end{itemize}

\paragraph{}Por otra parte, y como objetivos personales del autor del texto, se
encuentran:

\begin{itemize}
   \item Puesta en práctica del desarrollo de técnicas y habilidades aprendidas
   durante la formación del alumno a lo largo de su vida académica.

   \item Aprendizaje a la hora de resolver situaciones desconocidas
   anteriormente con respecto a posibles dificultades e imprevistos que puedan
   surgir a lo largo de la consecución del proyecto.
\end{itemize}