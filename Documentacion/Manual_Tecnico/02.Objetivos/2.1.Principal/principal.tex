\section{Objetivo principal}

\paragraph{}El objetivo principal de la realización de este proyecto será el
desarrollo e implantación de un sistema de software que facilite a los usuarios
implicados en las Asesorías Académicas (asesores y alumnos) la gestión y
mantenimiento de la información que les sea de utilidad para el correcto
desempeño de esta actividad.

\paragraph{}Este sistema debe permitir el acceso de manera personalizada a la
información que en él se almacena; por tanto, se hace necesario distinguir entre
los distintos tipos de usuarios que accederán al sistema en cuestión, que serán:

\begin{itemize}
 \item Usuario administrador principal.
 \item Usuario administrador de centro.
 \item Usuario asesor.
 \item Usuario alumno.
\end{itemize}

\paragraph{}Esto implica que, a la hora de acceder al sistema, se deba realizar
una identificación para conocer al usuario que desea acceder y poder ofrecerle
los servicios que, para su tipo de usuario y para ese usuario en concreto, han
sido establecidos.

\paragraph{}El usuario administrador principal será el encargado del buen
funcionamiento del sistema, gestionando al resto de usuarios del sistema y
estableciendo toda la estructura organizativa que presentará la aplicación
(creación de los distintos centros y sus respectivos administradores, definiendo
los planes de estudios, etc).

\paragraph{}La labor del usuario administrador de centro será muy similar a la
del usuario administrador principal excepto que no podrá extralimitarse más
allá del centro que le ha sido adjudicado para su gestión.

\paragraph{}Por otro lado, los usuarios asesor y alumno deberán poder acceder a
la información que les pertenece y realizar las acciones que sean de su
incumbencia.