\section{Objetivos secundarios}

\paragraph{}Se pretende que el sistema a desarrollar ofrezca una serie de
funcionalidades para cada tipo de usuario, con el objetivo de satisfacer las
necesidades de cada uno de ellos. A continuación se detallan las características
que se esperan alcanzar para cada uno de los usuarios del sistema.


\begin{itemize}
   \item Características comunes a todos los usuarios del sistema.
      \begin{itemize}
         \item Todos los usuarios podrán ejecutar consultas, de forma que, a
         través de una serie de condiciones de entrada, se genere una salida que
         satisfaga dichos valores.
         \item La parametrización de todo documento emitido por la aplicación.
         Esta función del sistema va a ser de gran importancia, ya que permitirá
         que los documentos sean modificados o actualizados según sea
         conveniente.
         \item Se podrán efectuar copias de seguridad, para salvaguardar cuando
         se desee la información existente en el sistema.
      \end{itemize}

   \item Características del usuario administrador.
      \begin{itemize}
         \item Posibilidad de añadir usuarios al sistema, especificando su tipo.
         \item Posibilidad de eliminar usuarios del sistema.
         \item Estará capacitado para modificar las contraseñas del resto de
         usuarios del sistema.
         \item \textit{Le estará permitido administrar las distintas plantillas
         que podrán ser usadas para generar entrevistas.}
         \item Podrá generar informes del estado de usuarios del sistema, que
         puedan ser mostrados por pantalla o imprimidos.
      \end{itemize}


   \item Características del usuario asesor.
      \begin{itemize}
         \item Tendrá la posibilidad de modificar su información personal.
         \item \textit{Será el encargado de añadir usuarios alumnos al sistema,
         los cuales quedarán asociados al asesor que los registre.}
         \item Se permitirá a cada asesor gestionar la información de sus
         alumnos de forma fácil e intuitiva.
         \item Deberá permitir realizar un seguimiento de cada alumno en
         particular.
         \item Podrá generar informes de información personal de alumnos y su
         seguimiento, que puedan ser mostrados por pantalla o imprimidos.
         \item Un asesor podrá generar entrevistas, grupales o individuales, que
         puedan ser mostradas por pantalla o imprimidas; las cuales, además,
         podrán ser recibidas por los alumnos a través del correo electrónico.
      \end{itemize}

   \item Características del usuario alumno.
      \begin{itemize}
         \item \textit{Podrá autoregistrarse en el sistema, debiendo solicitar
         a un asesor que lo acepte como alumno asesorado.}
         \item Tendrá la posibilidad de modificar su información personal.
      \end{itemize}
\end{itemize}

