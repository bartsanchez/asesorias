\section{Objetivos secundarios}

\paragraph{}Para alcanzar nuestro objetivo principal, se han de alcanzar unos
objetivos secundarios, que tienen su importancia para el desarrollo de la
aplicación. Estos objetivos son los siguientes:

\begin{itemize}
   \item Diseñar una base de datos que permita la gestión de toda la información
         de interés relativa a la Asesoría Académica para cada curso académico:
   \begin{itemize}
      \item Asesores.
      \item Alumnos: datos académicos y personales.
      \item Seguimiento de la asesoría: entrevistas, rendimiento académico, etc.
      \item Organización docente: centros, titulaciones, asignaturas.
      \item Etc.
   \end{itemize}

   \item Realizar un desarrollo de la aplicación modular que permita satisfacer
         las necesidades de cada uno de los tipos de usuarios:
   \begin{itemize}
      \item Administrador principal.
      \item Administrador de centro.
      \item Asesor académico.
      \item Alumno.
   \end{itemize}

   \item Crear un entorno interactivo robusto, consistente, amigable, intuitivo,
         de fácil manejo que fomente la interacción entre los usuarios y la
         interfaz, en la que se incluirán todas las opciones necesarias para
         cada uno de los tipos de usuario existente en la aplicación.

   \item Proporcionar una aplicación con unos mecanismos de control de acceso
         para garantizar la privacidad, seguridad e integridad de la
         información.

   \item Potenciar la explotación del sistema, facilitando las consultas sobre
         las asesorías académicas.
\end{itemize}


\paragraph{}Se pretende que el sistema a desarrollar ofrezca una serie de
funcionalidades para cada tipo de usuario, con el objetivo de satisfacer las
necesidades de cada uno de ellos. A continuación se detallan las características
que se esperan alcanzar para cada uno de los usuarios del sistema.


\begin{itemize}
   \item Características comunes a todos los usuarios del sistema.
      \begin{itemize}
         \item Todos los usuarios podrán ejecutar consultas, de forma que, a
         través de una serie de condiciones de entrada, se genere una salida que
         satisfaga dichos valores.
         \item La parametrización de todo documento emitido por la aplicación.
         Esta función del sistema va a ser de gran importancia, ya que permitirá
         que los documentos sean modificados o actualizados según sea
         conveniente.
      \end{itemize}

   \item Características del usuario administrador principal.
      \begin{itemize}
         \item Tendrá la posibilidad de gestionar al resto de usuarios del
               sistema.
         \item Establecerá los distintos centros que presente el sistema,
               designando su respectivo usuario administrador de centro.
         \item Podrá establecer y mantener los diferentes planes docentes que el
               sistema soportará.
         \item Le estará permitido administrar las distintas plantillas
         que podrán ser usadas para generar entrevistas.
         \item Podrá generar informes del estado de usuarios del sistema, que
         puedan ser mostrados por pantalla o imprimidos.
         \item Podrá efectuar copias de seguridad, para salvaguardar cuando
         desee la información existente en el sistema.
      \end{itemize}

   \item Características del usuario administrador de centro.
      \begin{itemize}
       \item Posibilidad de gestionar a los usuarios pertenecientes al centro
             que le concierne, tanto asesores como alumnos.
       \item Podrá establecer y mantener los diferentes planes docentes de los
             que su centro disponga.
       \item Por defecto en el sistema, será el encargado de relacionar los
             distintos usuarios asesores con sus correspondientes alumnos,
             asignando a cada usuario alumno el asesor que le corresponda. Esta
             opción, no obstante, podrá ser reemplazada por otras dos
             posibilidades, que serán comentadas con mayor profundidad más
             adelante:
             \begin{itemize}
               \item El usuario asesor se autoasigna sus usuarios alumnos.
               \item El usuario alumno elige a su usuario asesor.
             \end{itemize}
             Estas opciones no podrán activarse simultáneamente en el sistema,
             debiendo elegirse únicamente una de las tres posibilidades.
      \end{itemize}

   \item Características del usuario asesor.
      \begin{itemize}
         \item Tendrá la posibilidad de visualizar y modificar su información
         personal.
         \item Se permitirá a cada asesor gestionar la información de sus
         alumnos de forma fácil e intuitiva.
         \item Deberá permitir realizar un seguimiento de cada alumno en
         particular.
         \item Podrá generar informes de información personal de alumnos y su
         seguimiento, que podrán ser mostrados por pantalla o imprimidos.
         \item Un asesor podrá generar entrevistas, grupales o individuales, que
         podrán ser mostradas por pantalla o imprimidas, las cuales, además,
         podrán ser recibidas por los alumnos a través del correo electrónico.
         \item Dispondrá de dos funcionalidades, desactivadas por defecto,
         respecto a la forma en que se relaciona con el usuario alumno:
         \begin{itemize}
            \item Será el encargado de autoasignarse sus propios usuarios
            alumnos a los que ofrece asesoría.
            \item Podrá validar a los usuarios alumnos que les pidan asesoría.
         \end{itemize}
      \end{itemize}

   \item Características del usuario alumno.
      \begin{itemize}
         \item Tendrá la posibilidad de visualizar y modificar su información
         personal.
         \item Como opción desactivada por defecto, podrá visualizar los
         asesores disponibles para prestar asesoría así como solicitar sus
         servicios, en caso de no tener asignado uno previamente.
      \end{itemize}
\end{itemize}

