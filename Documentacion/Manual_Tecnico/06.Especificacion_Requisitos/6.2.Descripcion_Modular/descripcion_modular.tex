\section{Descripción Modular}

\paragraph{}

\begin{itemize}
   \item Datos Personales.
   \begin{itemize}
      \item \textit{¿Fotografía.?}
      \item D.N.I.
      \item Fecha de nacimiento.
      \item Nombre.
      \item Apellidos.
      \item Dirección en Córdoba.
      \item Teléfonos.
      \item Correo electrónico.
      \item Residencia durante el curso.
   \end{itemize}
   \item Datos Familiares.
   \begin{itemize}
      \item Dirección familiar.
      \item Localidad.
      \item Código postal.
      \item Provicia.
      \item Teléfonos.
   \end{itemize}
   \item Datos Académicos.
   \begin{itemize}
      \item Estudios realizados en año pasado.
      \item Estudios que está realizando.
      \item Curso.
      \item Año de ingreso en la Universidad.
      \item Otros estudios universitarios.
      \item Modalidad de acceso a la universidad.
      \item Calificación de: acceso/estudios previos.
   \end{itemize}
\end{itemize}


\paragraph{}Para llevar a cabo el seguimiento de los alumnos, el sistema deberá
poder gestionar la siguiente información:

\begin{itemize}
   \item Datos de la reunión.
   \begin{itemize}
      \item Asesor.
      \item Alumno (en caso de seguimiento individual).
      \item Asistentes (en caso de seguimiento grupal).
      \item Fecha.
      \item Hora.
      \item Duración.
   \end{itemize}
   \item Motivo.
   \item Orientaciones realizadas.
   \item Observaciones.
   \item Próxima reunión, temas a tratar.
   \begin{itemize}
      \item Fecha.
      \item Hora.
   \end{itemize}
\end{itemize}
