\section{Descripción Modular}

   \subsection{Módulo de asesores}

      \paragraph{}Una vez que ingresa en el sistema, un usuario asesor tendrá
      a su disposición un resumen con la información de todos y cada uno de los
      alumnos a los que presta asesoría. No obstante, el asesor podrá
      especificar un determinado alumno para obtener toda la información que
      se dispone de dicho individuo, como datos personales y académicos, o
      detalles de entrevistas realizadas.

      \paragraph{}La aplicación proporcionará a los asesores métodos y
      herramientas con los que poder realizar entrevistas a los alumnos
      asesorados.

      \paragraph{}\textit{También, el sistema dotará al usuario asesor de una
      zona especial, a través de la cual podrá transmitir información que estime
      oportuno a cada uno de sus alumnos asesorados; por ejemplo, horario
      de asesoría, próximas reuniones, etc.}

      \paragraph{}\textit{Este usuario será el encargado de dar de alta en el
      sistema a cada uno de los alumnos a los que preste servicio de asesoría.}

   \subsection{Módulo de alumnos}

      \paragraph{}\textit{¿Se autoregistrará en el sistema?}

      \paragraph{}Cada uno de los alumnos que acceda al sistema dispondrá
      de una serie de campos de información, opcionales u obligatorios, que
      deberá rellenar con el objetivo de facilitar al asesor que tenga asignado
      la labor de asesoría. Esta información se estructurará de la siguiente
      forma:

      \begin{itemize}
         \item Datos Personales.
         \begin{itemize}
            \item \textit{¿Fotografía.?}
            \item D.N.I.
            \item Fecha de nacimiento.
            \item Nombre.
            \item Apellidos.
            \item Dirección en Córdoba.
            \item Teléfonos.
            \item Correo electrónico.
            \item Residencia durante el curso.
         \end{itemize}
         \item Datos Familiares.
         \begin{itemize}
            \item Dirección familiar.
            \item Localidad.
            \item Código postal.
            \item Provicia.
            \item Teléfonos.
         \end{itemize}
         \item Datos Académicos.
         \begin{itemize}
            \item Estudios realizados en año pasado.
            \item Estudios que está realizando.
            \item Curso.
            \item Año de ingreso en la Universidad.
            \item Otros estudios universitarios.
            \item Modalidad de acceso a la universidad.
            \item Calificación de: acceso/estudios previos.
         \end{itemize}
      \end{itemize}

      \paragraph{}\textit{Además, el sistema proporcionará una zona especial
      a cada uno de los usuarios alumnos, en la podrán visualizar cierta
      información de sus asesores, como horario de asesoría disponible, próxima
      reunión, etc. Esta información depende por completo de lo que cada asesor
      estime oportuno.}


   \subsection{Módulo de administrador}

      \paragraph{}El administrador es el encargado y principal responsable
      del funcionamiento del sistema. Tiene permisos para gestionar el
      sistema completo, y es el único que puede administrar, en cuanto a
      creación, modificación y eliminación se refiere, al resto de usuarios
      del sistema.

      \paragraph{}Además, será el encargado de crear y mantener toda la
      estructura que componen las titulaciones, asignaturas, áreas de
      conocimiento, departamentos y cursos académicos.

   \subsection{Módulo de entrevistas}

      \paragraph{}Existen dos tipos de entrevistas contempladas: individuales y
      grupales. Como sus propios nombres indican, las entrevistas individuales
      se realizan a un alumno en particular, mientras que las entrevistas
      grupales se realizan a varios alumnos previamente seleccionados por el
      asesor, común a todos ellos.

      \paragraph{}La información que se gestionará entrevistas tiene la
      siguiente estructura:

      \begin{itemize}
         \item Datos de la reunión.
         \begin{itemize}
            \item Asesor.
            \item Alumno (en caso de seguimiento individual).
            \item Asistentes (en caso de seguimiento grupal).
            \item Fecha.
            \item Hora.
            \item Duración.
         \end{itemize}
         \item Motivo.
         \item Orientaciones realizadas.
         \item Observaciones.
         \item Próxima reunión, temas a tratar.
         \begin{itemize}
            \item Fecha.
            \item Hora.
         \end{itemize}
      \end{itemize}

   \subsection{Módulo de elaboración de informes}

      \paragraph{}RELLENAR.

   \subsection{Módulo de titulaciones}

      \paragraph{}En este módulo se establece la estructura que componen
      cada una de las titulaciones que contendrá el sistema. Por ejemplo,
      una titulación estará compuesta por asignaturas de un determinado
      curso académico. Esta estructura, será explicada con detalle en el
      capítulo \ref{modEntInt}, \textit{Modelo Entidad-Interrelación}.

      \paragraph{}Únicamente el administrador podrá realizar modificaciones de
      los elementos que este módulo aparecen, con la intención de evitar
      posibles malos usos de la aplicación por parte del resto de usuarios,
      asesores incluídos.

   \subsection{Módulo de copias de seguridad}

      \paragraph{}Este módulo está orientado a ofrecer mecanismos al usuario
      administrador con el objeto de realizar copias de seguridad de un
      estado determinado del sistema.

      \paragraph{}Estas copias de seguridad serán almacenadas en dispositivos
      de almacenamiento externo como discos magnéticos u ópticos. A su vez,
      transcurrido determinado tiempo, el sistema recordará al administrador
      la realización de este proceso.

      \paragraph{}La recuperación de la información almacenada es posible
      mediante un proceso de restauración que dispondrá la aplicación, en el
      cual se transferirá dicha información desde el soporte magnético u
      óptico auxiliar a la ubicación específica del sistema.
