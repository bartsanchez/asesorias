\section{Descripción Modular}\label{descMod}

   \subsection{Módulo de administrador}

      \subsubsection{Gestión del sistema}

      \paragraph{}El administrador es el encargado y principal responsable
      del funcionamiento del sistema. Tiene permisos para gestionar el
      sistema completo, y puede administrar, en cuanto a creación, modificación
      y eliminación se refiere, al resto de usuarios del sistema.

      \paragraph{}Además, será el encargado de crear y mantener toda la
      estructura que componen las titulaciones, asignaturas y cursos
      académicos.

      \subsubsection{Gestión de plantillas}

      \subsubsection{Gestión de entrevistas}

      \subsubsection{Gestión de informes}

      \paragraph{}Para realizar una mejor gestión del sistema, la aplicación
      proporcionará al usuario administrador de métodos o herramientas que
      permitan la generación de informes de estado del sistema. Estos informes
      podrán ser mostrados por pantalla o imprimidos.

      \paragraph{}Dichos informes, que podrán ser personalizados, pueden ser
      de dos tipos:

      \begin{itemize}
       \item Informes para visualizar los usuarios registrados en el sistema, y
             la vinculación existente entre ellos.
       \item Informes para visualizar la estructura que componen las
             titulaciones, asignaturas y cursos académicos.
      \end{itemize}

      \subsubsection{Gestión de copias de seguridad}

      \paragraph{}El sistema ofrecerá mecanismos al usuario administrador para
      realizar copias de seguridad de un determinado estado del sistema.

      \paragraph{}Estas copias de seguridad serán almacenadas en dispositivos
      de almacenamiento externo como discos magnéticos u ópticos. A su vez,
      transcurrido determinado tiempo, el sistema recordará al administrador
      la realización de este proceso.

      \paragraph{}La recuperación de la información almacenada es posible
      mediante un proceso de restauración que dispondrá la aplicación, en el
      cual se transferirá dicha información desde el soporte magnético u
      óptico auxiliar a la ubicación específica del sistema.

   \subsection{Módulo de asesores}

      \subsubsection{Explotación del sistema}

      \subsubsection{Gestión de alumnos}\label{gestAlumnos}

      \paragraph{}Este usuario se asignará a sí mismo a los alumnos a los que
      impartirá asesoría, teniendo para ello dos formas distintas de realizarlo:

      \begin{enumerate}
       \item El usuario asesor dará de alta en el sistema a un usuario alumno,
             el cual quedará vinculado en el sistema al usuario asesor.
       \item Un usuario alumno se autoregistrará en el sistema (ver sección
       \ref{modAlumnos}, \textit{Módulo de Alumnos}) y posteriormente solicitará
       a un usuario asesor que lo acepte como alumno asesorado.
      \end{enumerate}


      \paragraph{}Una vez que ingresa en el sistema, un usuario asesor tendrá
      a su disposición un resumen con la información de todos y cada uno de los
      alumnos a los que presta asesoría. No obstante, el asesor podrá
      especificar un determinado alumno para obtener toda la información que
      se dispone de dicho individuo, como datos personales y académicos, o
      detalles de entrevistas realizadas.

      \paragraph{}También, el sistema dotará al usuario asesor de una
      zona especial, a través de la cual podrá transmitir información que estime
      oportuno a cada uno de sus alumnos asesorados; por ejemplo, horario
      de asesoría, próximas reuniones, etc.

      \subsubsection{Gestión de entrevistas}

      \paragraph{}La aplicación proporcionará a los asesores métodos y
      herramientas con los que poder realizar entrevistas a los alumnos
      asesorados.

      \paragraph{}Existen dos tipos de entrevistas contempladas: individuales y
      grupales. Como sus propios nombres indican, las entrevistas individuales
      se realizan a un alumno en particular, mientras que las entrevistas
      grupales se realizan a varios alumnos previamente seleccionados por el
      asesor, común a todos ellos.

      \paragraph{}La información que se gestionará entrevistas tiene la
      siguiente estructura:

      \begin{itemize}
         \item Datos de la reunión.
         \begin{itemize}
            \item Asesor.
            \item Alumno (en caso de seguimiento individual).
            \item Asistentes (en caso de seguimiento grupal).
            \item Fecha.
            \item Hora.
            \item Duración.
         \end{itemize}
         \item Motivo.
         \item Orientaciones realizadas.
         \item Observaciones.
         \item Próxima reunión, temas a tratar.
         \begin{itemize}
            \item Fecha.
            \item Hora.
         \end{itemize}
      \end{itemize}

      \subsubsection{Gestión de informes}

      \paragraph{}La aplicación proporcionará al usuario asesor de métodos o
      herramientas que permitan la generación de informes que ayuden a los
      asesores a cumplir con su labor de asesoría. Estos informes podrán ser
      mostrados por pantalla o imprimidos.

      \paragraph{}Estos informes, que podrán ser personalizados, podrán ser
      de varios tipos:

      \begin{itemize}
       \item Informes generales de usuarios alumnos vinculados al usuario
             asesor.
       \item Informes particulares, con una determinada información relativa a
             un alumno en concreto.
       \item Informes de seguimiento, en los que se visualizará la trayectoria
             académica de un usuario alumno en particular.
       \item Informes de entrevistas, donde quedarán reflejadas las entrevistas
             que se han llevado a cabo, bajo unas determinadas circunstancias.
      \end{itemize}


   \subsection{Módulo de alumnos}\label{modAlumnos}

      \subsubsection{Gestión personal}

      \paragraph{}Como se indica en la sección \ref{gestAlumnos},
      \textit{Gestión de alumnos}, del módulo de asesores, existen dos formas
      distintas de que un usuario alumno pueda registrarse para acceder sistema:
      o bien un usuario asesor da de alta en el sistema a un usuario alumno,
      quedando vinculados mutuamente; o bien es el propio usuario alumno el que
      se registra en el sistema, y pide expresamente a su asesor que lo acepte
      para que el vínculo se haga efectivo. A continuación se profundiza en
      la segunda opción, más propia del módulo de alumnos que del módulo de
      asesores.

      \paragraph{}Siempre que un usuario, aún sin identificar, acceda al
      sistema, se le brindará la posibilidad de autoregistrarse para poder
      utilizar la aplicación. Esta opción solamente funcionará para dar de alta
      a usuarios alumnos, por motivos de seguridad.

      \paragraph{}Para que el registro se lleve a cabo, el usuario deberá
      rellenar una mínima información que le será requerida, además de
      seleccionar uno de los asesores disponibles para ejercer la asesoría. Una
      vez hecho esto, el sistema notificará al usuario asesor de que el usuario
      alumno solicita su vinculación en el sistema.

      \paragraph{}Si el usuario asesor acepta la solicitud, el usuario alumno
      quedará registrado en el sistema, vinculado al usuario asesor que aceptó.
      En caso contrario, si el usuario asesor rechaza la solicitud, el usuario
      alumno que realizó la petición no se registrará en el sistema.

      \subsubsection{Explotación del sistema}

      \paragraph{}Cada uno de los alumnos que acceda al sistema dispondrá
      de una serie de campos de información, opcionales u obligatorios, que
      deberá rellenar con el objetivo de facilitar al asesor que tenga asignado
      la labor de asesoría. Esta información se estructurará de la siguiente
      forma:

      \begin{itemize}
         \item Datos Personales.
         \begin{itemize}
            \item Fotografía.
            \item D.N.I.
            \item Fecha de nacimiento.
            \item Nombre.
            \item Apellidos.
            \item Dirección en Córdoba.
            \item Teléfonos.
            \item Correo electrónico.
            \item Residencia durante el curso.
         \end{itemize}
         \item Datos Familiares.
         \begin{itemize}
            \item Dirección familiar.
            \item Localidad.
            \item Código postal.
            \item Provicia.
            \item Teléfonos.
         \end{itemize}
         \item Datos Académicos.
         \begin{itemize}
            \item Estudios realizados en año pasado.
            \item Estudios que está realizando.
            \item Curso.
            \item Año de ingreso en la Universidad.
            \item Otros estudios universitarios.
            \item Modalidad de acceso a la universidad.
            \item Calificación de: acceso/estudios previos.
         \end{itemize}
      \end{itemize}

      \paragraph{}Además, el sistema proporcionará una zona especial
      a cada uno de los usuarios alumnos, en la podrán visualizar cierta
      información de sus asesores, como horario de asesoría disponible, próxima
      reunión, etc. Esta información depende por completo de lo que cada asesor
      estime oportuno.

      \subsubsection{Gestión de informes}

      \paragraph{}La aplicación proporcionará al usuario alumno de métodos o
      herramientas que permitan la generación de informes que les sirvan de
      orientación de cara a la planificación académica. Estos informes podrán
      ser mostrados por pantalla o imprimidos.