\section{Descripción Modular}\label{descMod}

   \subsection{Módulo de administrador principal}

      \paragraph{}El módulo de administrador principal está compuesto de
      los siguientes submódulos:

      \begin{itemize}
       \item Gestión del sistema.
       \item Gestión de plantillas.
       \item Gestión de entrevistas.
       \item Gestión de copias de seguridad.
       \item Explotación del sistema.
      \end{itemize}

      \subsubsection{Gestión del sistema}

      \paragraph{}El administrador principal es el encargado y responsable
      del funcionamiento del sistema. Tiene permisos para gestionar el
      sistema completo, y puede administrar, en cuanto a creación, modificación
      y eliminación se refiere, al resto de usuarios del sistema.

      \paragraph{}Este usuario es el encargado de establecer los distintos
      centros que presentará el sistema, además de sus respectivos usuarios
      administradores de centro.

      \paragraph{}Además, le estará permitido crear y mantener toda la
      estructura que componen las titulaciones, asignaturas y cursos
      académicos.

      \subsubsection{Gestión de plantillas}

      \paragraph{}El sistema ofrecerá la posibilidad de hacer uso de las
      plantillas oficiales que establezcan los distintos organismos competentes
      relacionados con la Asesoría Académica. El usuario administrador principal
      será el único responsable de su gestión.

      \paragraph{}Le estará permitido añadir, eliminar y editar las distintas
      plantillas oficiales, con objeto de adecuarse a lo establecido por los
      distintos organismos competentes.

      \subsubsection{Gestión de entrevistas}

      \paragraph{}Al igual que ocurre con la gestión de plantillas, el usuario
      administrador de centro es el único responsable de la gestión (adición,
      eliminación y edición) de las distintas entrevistas que ofrezcan los
      distintos organismos competentes relacionados con la Asesoría Académica.

      \subsubsection{Gestión de copias de seguridad}

      \paragraph{}El sistema ofrecerá mecanismos al usuario administrador
      principal para realizar copias de seguridad de un determinado estado del
      sistema.

      \paragraph{}Estas copias de seguridad serán almacenadas en dispositivos
      de almacenamiento externo como discos magnéticos u ópticos. A su vez,
      transcurrido determinado tiempo, el sistema recordará al administrador
      la realización de este proceso.

      \paragraph{}La recuperación de la información almacenada es posible
      mediante un proceso de restauración que dispondrá la aplicación, en el
      cual se transferirá dicha información desde el soporte magnético u
      óptico auxiliar a la ubicación específica del sistema.

      \subsubsection{Explotación del sistema}

      \paragraph{}Para realizar una mejor gestión del sistema, la aplicación
      proporcionará, al usuario administrador principal, métodos o herramientas
      que permitan la generación de informes de estado del sistema. Estos
      informes podrán ser mostrados por pantalla o imprimidos.

      \paragraph{}Dichos informes, que podrán ser personalizados, pueden ser
      de dos tipos:

      \begin{itemize}
       \item Informes para visualizar los usuarios registrados en el sistema, y
             la vinculación existente entre ellos.
       \item Informes para visualizar la estructura que componen las
             titulaciones, asignaturas y cursos académicos.
      \end{itemize}

   \subsection{Módulo de administrador de centro}

      \paragraph{}El módulo de administrador de centro está compuesto de
      los siguientes submódulos:

      \begin{itemize}
       \item Gestión de centro.
       \item Gestión de plantillas.
       \item Gestión de entrevistas.
       \item Explotación del sistema.
      \end{itemize}

      \subsubsection{Gestión de centro}

      \paragraph{}El usuario administrador de centro es el principal responsable
      del buen funcionamiento del centro que administra; y, por tanto, podrá
      administrar, en cuanto a creación, modificación y eliminación se refiere,
      al resto de usuarios del centro al cual pertenece, excepto a otros
      posibles administradores del mismo centro que compartan responsabilidad,
      los cuales solo podrán ser administrador por el administrador principal.

      \paragraph{}Además, le estará permitido crear y mantener toda la
      estructura que componen las titulaciones, asignaturas y cursos
      académicos del centro al que pertenece.

      \subsubsection{Gestión de plantillas}

      \paragraph{}El administrador de centro estará capacitado para gestionar
      (añadir, eliminar y editar) plantillas específicas de utilidad para el
      centro al que pertenezca (plantillas de centro), independientes del resto
      de centros que haya establecidos en el sistema.

      \subsubsection{Gestión de entrevistas}

      \paragraph{}Al igual que ocurre con las plantillas de centro, el usuario
      administrador de centro gestionará entrevistas específicas del centro
      al cual pertenece.

      \subsubsection{Explotación del sistema}

      \paragraph{}La aplicación proporcionará al usuario administrador de centro
      métodos o herramientas que permitan la generación de informes de estado
      del centro que administre. Estos informes podrán ser mostrados por
      pantalla o imprimidos.

      \paragraph{}Dichos informes, que podrán ser personalizados, pueden ser
      de dos tipos:

      \begin{itemize}
       \item Informes para visualizar los usuarios registrados en el centro
             al que pertenezca, y la vinculación existente entre ellos.
       \item Informes para visualizar la estructura que componen las
             titulaciones, asignaturas y cursos académicos del centro que
             administre.
      \end{itemize}

   \subsection{Módulo de asesores}

      \paragraph{}El módulo de asesores está compuesto de los siguientes
      submódulos:

      \begin{itemize}
       \item Gestión de alumnos.
       \item Gestión de plantillas.
       \item Gestión de entrevistas.
       \item Explotación del sistema.
      \end{itemize}

      \subsubsection{Gestión de alumnos}

      \paragraph{}Una vez que ingresa en el sistema, un usuario asesor tendrá
      a su disposición un resumen con la información de todos y cada uno de los
      alumnos a los que presta asesoría. No obstante, el asesor podrá
      especificar un determinado alumno para obtener toda la información que
      se dispone de dicho individuo, como datos personales y académicos, o
      detalles de entrevistas realizadas.

      \paragraph{}También, el sistema dotará al usuario asesor de una
      zona especial, a través de la cual podrá transmitir la información que
      estime oportuno a cada uno de sus alumnos asesorados; por ejemplo, horario
      de asesoría, próximas reuniones, etc.

      \subsubsection{Gestión de entrevistas}

      \paragraph{}La aplicación proporcionará a los asesores métodos y
      herramientas con los que poder realizar entrevistas a los alumnos
      asesorados.

      \paragraph{}Existen dos tipos de entrevistas contempladas: individuales y
      grupales. Como sus propios nombres indican, las entrevistas individuales
      se realizan a un alumno en particular, mientras que las entrevistas
      grupales se realizan a varios alumnos previamente seleccionados por el
      asesor, común a todos ellos.

      \paragraph{}La información que se gestionará entrevistas tiene la
      siguiente estructura:

      \begin{itemize}
         \item Datos de la reunión.
         \begin{itemize}
            \item Asesor.
            \item Alumno (en caso de seguimiento individual).
            \item Asistentes (en caso de seguimiento grupal).
            \item Fecha.
            \item Hora.
            \item Duración.
         \end{itemize}
         \item Motivo.
         \item Orientaciones realizadas.
         \item Observaciones.
         \item Próxima reunión, temas a tratar.
         \begin{itemize}
            \item Fecha.
            \item Hora.
         \end{itemize}
      \end{itemize}

      \subsubsection{Explotación del sistema}

      \paragraph{}La aplicación proporcionará métodos o herramientas que
      permitan la generación de informes que ayuden a los asesores a cumplir con
      su labor de asesoría. Estos informes podrán ser mostrados por pantalla o
      imprimidos.

      \paragraph{}Estos informes, que podrán ser personalizados, podrán ser
      de varios tipos:

      \begin{itemize}
       \item Informes generales de usuarios alumnos vinculados al usuario
             asesor.
       \item Informes particulares, con una determinada información relativa a
             un alumno en concreto.
       \item Informes de seguimiento, en los que se visualizará la trayectoria
             académica de un usuario alumno en particular.
       \item Informes de entrevistas, donde quedarán reflejadas las entrevistas
             que se han llevado a cabo, bajo unas determinadas circunstancias.
      \end{itemize}


   \subsection{Módulo de alumnos}\label{modAlumnos}

      \paragraph{}El módulo de alumnos está compuesto de los siguientes
      submódulos:

      \begin{itemize}
       \item Explotación del sistema.
      \end{itemize}

      \subsubsection{Explotación del sistema}

      \paragraph{}Cada uno de los alumnos que acceda al sistema dispondrá
      de una serie de campos de información, opcionales u obligatorios, que
      deberá rellenar con el objetivo de facilitar al asesor que tenga asignado
      la labor de asesoría. Esta información se estructurará de la siguiente
      forma:

      \begin{itemize}
         \item Datos Personales.
         \begin{itemize}
            \item Fotografía.
            \item D.N.I.
            \item Fecha de nacimiento.
            \item Nombre.
            \item Apellidos.
            \item Dirección en Córdoba.
            \item Teléfono.
            \item Correo electrónico.
            \item Residencia durante el curso.
         \end{itemize}
         \item Datos Familiares.
         \begin{itemize}
            \item Dirección familiar.
            \item Localidad.
            \item Provincia.
            \item Código postal.
            \item Teléfono.
         \end{itemize}
         \item Datos Académicos.
         \begin{itemize}
            \item Estudios realizados en año pasado.
            \item Estudios que está realizando.
            \item Curso.
            \item Año de ingreso en la Universidad.
            \item Otros estudios universitarios.
            \item Modalidad de acceso a la universidad.
            \item Calificación de: acceso/estudios previos.
         \end{itemize}
      \end{itemize}

      \paragraph{}Además, el sistema proporcionará una zona especial
      a cada uno de los usuarios alumnos, en la podrán visualizar cierta
      información de sus asesores, como horario de asesoría disponible, próxima
      reunión, etc. Esta información depende por completo de lo que cada asesor
      estime oportuno.

      \paragraph{}Por otra parte, la aplicación proporcionará al usuario alumno
      métodos o herramientas que permitan la generación de informes que les
      sirvan de orientación de cara a la planificación académica. Estos informes
      podrán ser mostrados por pantalla o imprimidos.