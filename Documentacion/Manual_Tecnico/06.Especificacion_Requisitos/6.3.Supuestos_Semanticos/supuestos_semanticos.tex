\section{Supuestos Semánticos}\label{supSem}

   \paragraph{}A continuación se detallan las entidades que participan en el
   diseño del sistema, con sus correspondientes supuestos semánticos.

   \begin{description}

      \item[Centro] Un centro representa una institución donde se imparte
      docencia de titulaciones universitarias.
      \begin{itemize}
         \item Solo se van a considerar centros de la universidad de Córdoba.
         \item Un centro será identificado mediante un código numérico o por su
         nombre.
      \end{itemize}

      \item[Titulación] Conjunto de materias cuya superación supone la obtención
      de un título académico.
      \begin{itemize}
         \item Una titulación se imparte en un único centro universitario, pero
         en un centro universitario se pueden impartir varias titulaciones.
         \item Cada titulación se puede identificar de dos formas:
         \begin{itemize}
            \item Usando el código del centro y el código de la titulación.
            \item Usando el código del centro, el nombre de la titulación y el
            año en que se aprobó su plan de estudios.
         \end{itemize}
      \end{itemize}

      \item[Asignatura] Materia que forma parte del plan de estudios de una
      titulación.
      \begin{itemize}
         \item Una asignatura pertenece a una única titulación, pero una
         titulación puede estar compuesta por varias asignaturas.
         \item Cada asignatura se puede identificar de dos formas:
         \begin{itemize}
            \item Usando el código del centro, el código de la titulación y el
            código de la asignatura.
            \item Usando el código del centro, el código de la titulación y el
            nombre de la asignatura.
         \end{itemize}
         \item Una misma asignatura se puede impartir en varios cursos
         académicos.
         \item Una asignatura tiene asignada una carga docente traducida en un
         número de créditos, teóricos y prácticos.
         \item Una asignatura puede ser de varios tipos: troncal, obligatoria,
         optativa o de libre configuración.
         \item En cada curso académico, una asignatura podrá tener matriculados
         varios alumnos, y un alumno podrá matricularse de varias asignaturas.
      \end{itemize}

      \item[Alumno] Estudiante de una titulación que recibe asesoría.
      \begin{itemize}
         \item Un alumno podrá ser identificado por su dni o su pasaporte.
      \end{itemize}

      \item[Asesor] Profesor que realiza un seguimiento permanente y orientado a
      la optimización del esfuerzo de estudio de un alumno.
      \begin{itemize}
         \item Un asesor podrá ser identificado mediante su dni o pasaporte.
         \item En cada curso académico, un asesor podrá tener asignados varios
         alumnos, pero un alumno solo podrá tener asignado un asesor.
      \end{itemize}

      \item[Administrador de centro] Persona encargada de la gestión
      administrativa de las asesorías en un centro universitario.
      \begin{itemize}
       \item Un administrador de centro podrá ser identificado mediante su dni o
       pasaporte.
       \item Un administrador de centro podrá gestionar varios centros, y un
       centro podrá ser gestionado por varios administradores.
      \end{itemize}
 \end{description}
