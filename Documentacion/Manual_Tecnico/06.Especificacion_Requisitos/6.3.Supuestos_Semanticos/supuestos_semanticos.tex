\section{Supuestos Semánticos}

   \paragraph{}A continuación se detallan las entidades que participan en el
   diseño del sistema, con sus correspondientes supuestos semánticos.

   \begin{description}

      \item[Titulación] Conjunto de materias cuya superación supone la obtención
      de un título académico.
      \begin{itemize}
         \item Cada titulación tiene un identificador que hace referencia a ella
          de forma unívoca.
          \item Una titulación está compuesta de al menos una asignatura y como
         máximo un número indeterminado.
      \end{itemize}

      \item[Asignatura] Materia que forma parte del plan de estudios de una
      titulación.
      \begin{itemize}
         \item Cada asignatura tiene un identificador que hace referencia a ella
         de forma unívoca.
         \item Una asignatura pertenece a una y solo a una titulación.
         \item \textit{Una asignatura pertenece a una o a varias áreas de
         conocimiento.}
         \item Una asignatura tiene asignada una carga docente traducida en
         un número de créditos.
         \item \textit{Los créditos de una asignatura pueden estar compuestos de
         créditos teóricos, prácticos y/o problemas.}
         \item Una misma asignatura se puede impartir en varios cursos
         académicos.
      \end{itemize}

      \item[Curso académico] Tiempo señalado en cada año en el que se imparte
      docencia.
      \begin{itemize}
         \item Cada curso académico tiene un identificador que hace referencia
         al mismo de forma unívoca.
         \item En un curso académico se imparte al menos una y como máximo
         un número indeterminado de asignaturas.
         \item Un número indeterminado de alumnos pertenece a un determinado
         curso académico.
      \end{itemize}

      \item[Alumno] Estudiante de una titulación que recibe asesoría.
      \begin{itemize}
         \item Cada alumno tiene un identificador que hace referencia al mismo
         de forma unívoca.
         \item Un alumno recibe asesoría de un, y solo un, asesor.
         \item Un determinado alumno puede estar matriculado en un número
         indeterminado de cursos académicos.
      \end{itemize}

      \item[Asesor] Tutor que realiza un seguimiento permanente y orientado a la
      optimización del esfuerzo de estudio de un alumno.
      \begin{itemize}
         \item Cada asesor tiene un identificador que hace referencia al mismo
         de forma unívoca.
         \item Un asesor asesora a un número indeterminado de alumnos.
      \end{itemize}

 \end{description}
