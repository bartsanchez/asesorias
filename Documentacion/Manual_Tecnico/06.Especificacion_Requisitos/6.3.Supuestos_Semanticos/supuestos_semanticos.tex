\section{Supuestos Semánticos}\label{supSem}

   \paragraph{}A continuación se detallan las entidades que participan en el
   diseño del sistema, con sus correspondientes supuestos semánticos.

   \begin{description}

      \item[Titulación] Conjunto de materias cuya superación supone la obtención
      de un título académico.
      \begin{itemize}
         \item Cada titulación tiene un identificador que hace referencia a ella
          de forma unívoca.
          \item Una titulación está compuesta de al menos una asignatura y como
         máximo un número indeterminado.
      \end{itemize}

      \item[Asignatura] Materia que forma parte del plan de estudios de una
      titulación.
      \begin{itemize}
         \item Cada asignatura tiene un identificador que hace referencia a ella
         de forma unívoca.
         \item Una misma asignatura se puede impartir en varios cursos
         académicos.
         \item Una asignatura pertenece a una y solo a una titulación.
         \item Una asignatura tiene asignada una carga docente traducida en
         un número de créditos.
         \item Una asignatura puede ser de varios tipos: troncal, obligatoria,
         optativa o de libre configuración.
      \end{itemize}

      \item[Alumno] Estudiante de una titulación que recibe asesoría.
      \begin{itemize}
         \item Cada alumno tiene un identificador que hace referencia al mismo
         de forma unívoca.
         \item Cada alumno recibe asesoría de un, y solo un asesor.
         \item Un determinado alumno puede estar matriculado en un número
         indeterminado de asignaturas.
      \end{itemize}

      \item[Asesor] Tutor que realiza un seguimiento permanente y orientado a la
      optimización del esfuerzo de estudio de un alumno.
      \begin{itemize}
         \item Cada asesor tiene un identificador que hace referencia al mismo
         de forma unívoca.
         \item Un asesor ofrece asesoría a un número indeterminado de alumnos.
      \end{itemize}

 \end{description}
