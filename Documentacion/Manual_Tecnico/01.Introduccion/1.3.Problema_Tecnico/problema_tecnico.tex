\subsection{Problema Técnico}

\paragraph{}En base a las características del problema real, pueden extraerse
una serie de condicionantes técnicos que influirán de forma decisiva en el
producto final. Para la obtención de éstos, se va a utilizar una técnica de
ingeniería denominada PDS (Product Design Specification). Los siguientes
elementos determinan las especificaciones técnicas que caracterizan al problema.

\begin{itemize}
 \item Funcionamiento

   \paragraph{}La aplicación permitirá la gestión y el mantenimiento de la
   información relativa a la Asesoría Académica; es decir, información personal
   de los alumnos asesorados así como el seguimiento de los mismos a lo largo
   de la labor de asesoría.

   \paragraph{}El sistema dispondrá de la información personal necesaria de
   alumnos con el objetivo de de facilitar el trabajo de la labor de asesoría a
   quienes la realizan. A su vez, se proporcionará a los asesores herramientas
   que faciliten un seguimiento exhaustivo de cada uno de los alumnos
   asesorados.

   \paragraph{}Además se dispondrá de un generador de informes donde se podrán
   realizar consultas y elaborar documentos personalizados a partir de
   determinados parámetros de entrada. Todo documento generado tendrá la
   capacidad de ser parametrizado; es decir, será relativamente manipulable
   para que pueda ser modificado o actualizado según sea preciso.

   \textit{\paragraph{}Por otra parte, la aplicación permitirá a los asesores
   crear entrevistas (individuales o grupales), personalizables o elegidas a
   través de una serie de plantillas, que serán contestadas por los alumnos.
   ¿¿Mediante la propia aplicación??.}

   \paragraph{}Finalmente, para evitar pérdidas de información existente
   en el sistema, y mantener una apropiada integridad de la misma, la aplicación
   permitirá hacer copias de seguridad.


 \item Entorno
 \item Vida esperada
 \item Ciclo de mantenimiento
 \item Competencia
 \item Aspecto externo
 \item Estandarización
 \item Calidad y fiabilidad
 \item Pruebas
 \item Seguridad
\end{itemize}
