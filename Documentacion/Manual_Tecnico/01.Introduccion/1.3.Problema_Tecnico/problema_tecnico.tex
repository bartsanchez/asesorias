\section{Problema Técnico}

\paragraph{}A partir de las características del problema real, pueden extraerse
una serie de condicionantes técnicos que influirán de forma decisiva en el
producto final. Para la obtención de éstos, se va a utilizar una técnica de
ingeniería denominada PDS (Product Design Specification). Los siguientes
elementos determinan las especificaciones técnicas que caracterizan al problema.

\begin{itemize}
 \item Funcionamiento

   \paragraph{}La aplicación permitirá la gestión y el mantenimiento de la
   información relativa a la Asesoría Académica; es decir, información personal
   de los alumnos asesorados así como el seguimiento de los mismos a lo largo
   de la labor de asesoría.

   \paragraph{}El sistema dispondrá de la información personal necesaria de
   alumnos con el objetivo de de facilitar el trabajo de la labor de asesoría a
   quienes la realizan. A su vez, se proporcionará a los asesores herramientas
   que faciliten un seguimiento exhaustivo de cada uno de los alumnos
   asesorados.

   \paragraph{}Además se dispondrá de un generador de informes donde se podrán
   realizar consultas y elaborar documentos personalizados a partir de
   determinados parámetros de entrada. Todo documento generado tendrá la
   capacidad de ser parametrizado; es decir, será relativamente manipulable
   para que pueda ser modificado o actualizado según sea preciso.

   \textit{\paragraph{}Por otra parte, la aplicación permitirá a los asesores
   crear entrevistas (individuales o grupales), personalizables o elegidas a
   través de una serie de plantillas, que serán contestadas por los alumnos.
   ¿¿Mediante la propia aplicación??.}

   \paragraph{}Finalmente, para evitar pérdidas de información existente
   en el sistema, y mantener una apropiada integridad de la misma, la aplicación
   permitirá hacer copias de seguridad.

   \paragraph{}En el capítulo \ref{espReq}, \textit{Especificación de
   requisitos}, se encuentran reflejadas con más detalle las funciones del
   sistema.

 \item Entorno

   \paragraph{}El sistema a desarrollar consistirá en una aplicación web alojada
   en un servidor, a la que los usuarios accederán a través de Internet o red
   local mediante el uso de un navegador.

   \paragraph{}La aplicación constará de una interfaz amigable e
   intuitiva basada en cuadros de diálogo, botones, menús desplegables, etc.
   Como periféricos de entrada, serán necesarios el teclado y el ratón. Como
   periféricos de salida, la impresora. Para la generación de determinados
   informes, como los correos electrónicos, será necesaria la conexión a
   Internet.

 \item Vida esperada

   \paragraph{}Este sistema, en principio, no tiene una vida esperada
   determinada. El tiempo de funcionamiento debería ser el máximo posible,
   siempre que su utilización suponga alguna ventaja sustancial con respecto a
   otros futuros sistemas alternativos.

   \paragraph{}Debido a los cambios administrativos que se pueden dar en los
   centros universitarios, como por ejemplo, la incorporación de nuevas
   titulaciones o el cambio de planes de estudios, este sistema tendrá que ser
   continuamente revisado para mantenerlo actualizado a los cambios producidos.

 \item Ciclo de mantenimiento

   \paragraph{}Debido a que el sistema informático va a ser desarrollado como
   objetivo de un proyecto de fin de carrera, se considera que el autor no debe
   ser el responsable de su mantenimiento. No obstante, en su realización, se
   utilizará una metodología de desarrollo modular que permite a otros
   programadores incorporar futuras mejoras con facilidad.

 \item Competencia

   \paragraph{}Debido a la reciente implantación del rol del asesor académico,
   no se conoce ningún sistema de software dentro de la Universidad de Córdoba
   ni en el resto de universidades que realice las funciones desempeñadas por
   éste.

 \item Aspecto externo

   \paragraph{}El sistema constará de una interfaz intuitiva y ergonómica que
   hará que su uso sea cómodo y fácil. Ésta constará de elementos bien conocidos
   y comúnmente utilizados como cuadros de diálogo, botones, menús desplegables,
   etc.

   \paragraph{}El soporte utilizado para albergar la aplicación será el CD-ROM,
   por ser ampliamente utilizado, compatible y de tamaño suficiente.

   \paragraph{}El aspecto externo se encuentra explicado con más detalle en el
   capítulo \ref{espReqInt}, \textit{Especificación de requisitos de la
   interfaz}.

 \item Estandarización

   \paragraph{}El sistema de software a desarrollar mantendrá un alto nivel de
   estandarización en cuanto a:

   \begin{itemize}
      \item Sistema de almacenamiento de información. El motor de la base de
      datos utilizado para albergar la información está ampliamente implantado
      en el mercado. Dicho sistema de almacenamiento se denomina MySQL.
      \item Interfaz de ventanas. Se hará uso de instrumentos de diseño de
      interfaz muy populares, los cuales se encuentran expuestos en el capítulo
      \ref{recursos}, Recursos.
      \item Soporte digital CD-ROM.
   \end{itemize}


 \item Calidad y fiabilidad

   \paragraph{}La aplicación comprobará los pasos dados por el usuario: si en
   algún momento éste realiza una acción no permitida, el sistema intentará
   guiarle mediante mensajes de ayuda.

 \item Pruebas

   \paragraph{}Las pruebas realizadas sobre el sistema son un elemento
   determinante a la hora de garantizar la calidad y fiabilidad del mismo.

   \paragraph{}Para comprobar que el sistema funciona correctamente, estas
   pruebas tendrán dos objetivos básicamente:

   \begin{itemize}
      \item Comprobar que la aplicación hace lo que debe hacer.
      \item Probar que la aplicación no hace lo que no debe hacer, es decir,
      comprobar que no provoca efectos secundarios adversos.
   \end{itemize}

   \paragraph{}Se realizarán pruebas sobre los siguientes subsistemas de la
   aplicación:

   \begin{itemize}
      \item Interfaz: comprobando el correcto funcionamiento de los diferentes
      componentes que la constituyen.
      \item Comunicación con la base de datos del sistema y la apropiada
      modificación de la información que alberga.
      \item Generación de documentos y parametrización de los mismos.
      \item Instalación y desinstalación.
   \end{itemize}

   Las pruebas realizadas sobre este nuevo sistema de software se
   encuentran detalladas en el capítulo \ref{pruebas}, \textit{Pruebas}.


 \item Seguridad

   \paragraph{}El sistema de gestión de bases de datos de la aplicación tiene la
   capacidad de proteger los datos contra su pérdida parcial o total debidos a
   fallos en el sistema. A su vez, garantiza la privacidad, es decir, sólo
   permite el acceso a la información a personas autorizadas.

   \paragraph{}Dependiendo de quién acceda o use la base de datos, se le
   mostrará una visión diferente de los datos que sea capaz de reconocer,
   interpretar y manejar. De esta manera tenemos varias visiones diferentes. La
   que poseerán los alumnos, la que poseerán los profesores, y finalmente la del
   administrador. Estas visiones   particulares son proporcionadas por los
   programas de aplicación que manejan sólo parte de la información contenida
   en la base de datos.


   \paragraph{}Finalmente, el mecanismo de copias de seguridad del sistema
   permitirá al usuario salvaguardar la información siempre que crea
   conveniente.

\end{itemize}
