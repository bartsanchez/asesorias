\section{Asesorías Académicas}

\paragraph{}La asesoría es una acción docente de orientación con la finalidad de
participar en la formación integral del alumno, potenciando su desarrollo
académico y personal, así como su proyección social y profesional.

\paragraph{}La forma de expresión clásica de asesoría docente en el ámbito
universitario se limitaba a las Tutorías Académicas. Este tipo de asesoría,
inherente al rol de profesor, la realiza cada docente en su asignatura con su
grupo de estudiantes. En ellas los profesores supervisan el trabajo del
estudiante, orientan, resuelven dudas, aconsejan bibliografía, revisan trabajos
y pruebas, etc., pero siempre dentro del ámbito de la propia asignatura. Por su
parte, como se ha indicado anteriormente, la asesoría académica tiene una
aplicación más amplia y genérica.

\paragraph{}Con el fin de aumentar la calidad en la enseñanza dentro del ámbito
universitario, aparece la figura del Asesor Académico. Su labor sería la
de orientar e informar al estudiante sobre cualquier duda en relación con los
aspectos académicos que se le puedan plantear durante su estancia en la
Universidad, a través de un seguimiento permanente, eficaz y orientado a la
optimización del esfuerzo de estudio por parte del alumnado.

\paragraph{}La Universidad de Córdoba, con el fin de garantizar y satisfacer el
ejercicio de un derecho reconocido a los estudiantes, como es la labor de
asesoría docente, y mejorar el rendimiento académico del alumnado, aprobó en
marzo de 2007 por el Consejo de Gobierno el Plan Propio de Calidad de la
Enseñanza, donde se contempla la creación de la figura del Asesor Académico.

\paragraph{}Para llevar a cabo su implantación, la Universidad establece un
reglamento regulador que define el papel del asesor académico; además, con la
intención de facilitar la organización y el seguimiento, se proporcionan una
serie de guías y fichas donde el docente gestionará su relación con cada uno de
los alumnos a los que asesore.

\paragraph{}Dada la complejidad que supone gestionar toda la información
relativa a la Asesoría Académica, más para el docente aunque algo también para
el alumno, y lo reciente de su implantación, hace ineficiente el normal
desarrollo de la actividad de asesoría. El proyecto fin de carrera que se desea
realizar pretende superar estas dificultades; proporcionando, a todos los
implicados en la actividad de asesoría, un sistema informático que facilite la
gestión y el mantenimiento de la información relativa a la Asesoría Académica.
