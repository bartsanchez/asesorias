\section{Definición del problema real}

\paragraph{}Debido a la reciente implantación del rol del asesor académico en el
entorno
universitario de la Universidad de Córdoba, resulta tediosa y poco eficiente la
tarea de organizar la información relativa a la relación alumno-asesor para
cada una de las dos partes.

\paragraph{}La Universidad ha proporcionado una serie de documentos (consejos
generales, fichas de seguimiento, encuestas, guías de reuniones, etcétera.) con
el objetivo de que los asesores puedan llevar cierto control sobre los alumnos
a los que imparten asesoría. Este método conlleva varias deficiencias:

\begin{itemize}
 \item Toda la gestión y control de la información se realiza sobre papel, lo
       que no resulta eficiente en el caso de disponer de una gran cantidad de
       alumnos a los que impartir asesoría; además, no es difícil que algún
       folio se extravíe.
 \item Imposibilidad de compartir información públicamente (siempre que ésta
       de verdad quiera ser compartida). Esto conlleva, por ejemplo, a que un
       alumno no puede, en principio, conocer qué profesores se ofrecen, en un
       momento determinado, a prestar servicio de asesoría.
 \item Resulta muy problemática la actualización o modificación de cualquier
       elemento de la información que sea necesario corregir. Como sabemos, los
       métodos de corrección de la escritura tradicional no son todo lo
       efectivos o limpios que nos gustaría.
\end{itemize}

\paragraph{}El sistema que se pretende desarrollar persigue resolver estos
problemas así
como facilitar la gestión y manipulación de toda la información relativa a
las Asesorías Académicas, además de aprovechar las ventajas que suponen hoy en
día las tecnologías de la información, como pueden ser:

\begin{itemize}
 \item Fácil acceso a la información. En la actualidad y cada día más, muchas
       personas tienen acceso a internet, a través de múltiples dispositivos y
       en diversos lugares.
 \item Almacenamiento de grandes cantidades de información, así como su posible
       centralización.
 \item Fácil manipulación de los datos almacenados.
 \item Interactividad. Se fomenta la relación alumno-asesor a través de
       múltiples formas ( correo electrónico, foros, etc. ).
\end{itemize}
