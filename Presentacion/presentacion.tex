\documentclass[10pt, hyperref={pdfpagelabels=false}]{beamer}

% Configuración
\usepackage[utf8]{inputenc}
\usepackage[spanish]{babel}
\usepackage{lmodern} % Arregla un warning
\usepackage{multicol}

% Tema beamer
\usetheme{Warsaw}

% Detalles
\title[Asesorías Académicas]{Aplicación web para la gestión de la Asesoría
                             Académica en la Universidad de Córdoba}
\author{Bartolomé Sánchez Salado}
\institute{
    \begin{tabular}{lp{3cm}r}
      \textbf{Director} && \\
      Nicolás Luis Fernández García && \scriptsize{Córdoba, enero 2011} \\
    \end{tabular}}
\date{}

\begin{document}

  %--- PORTADA ---%
  \begin{frame}[plain]
    \begin{tabular}{ccc}
      \\
      \includegraphics[height=0.18\textheight]{uco} &
      \begin{tabular}{c}
        {\bf UNIVERSIDAD DE CÓRDOBA} \\
        {\bf Escuela Politécnica Superior} \\\\
        Ingeniería Técnica en Informática de Gestión \\
        Proyecto de fin de carrera \\\\
        \end{tabular} &
      \includegraphics[height=0.18\textheight]{eps} \\
    \end{tabular}
    \titlepage
  \end{frame}

  \AtBeginSection[]
  {
    \begin{frame}<beamer>
      \frametitle{Sumario}
      \tableofcontents[currentsection,hideallsubsections]
    \end{frame}
  }

  \AtBeginSubsection[]
  {
    \begin{frame}<beamer>
      \frametitle{Contenido de la sección}
      \begin{multicols}{2}
        \tableofcontents[currentsubsection,subsectionstyle=show/shaded/hide]
      \end{multicols}
    \end{frame}
   }

   \begin{frame}{Contenido}
    \tableofcontents[hideallsubsections]
   \end{frame}

  \section{Presentación}
    \subsection{Introducción}
      \begin{frame}{Introducción}
        \begin{block}{Asesoría clásica}
          \begin{itemize}
          \item Se limitaba a tutorías.
          \item Siempre dentro del ámbito de una asignatura.
          \end{itemize}
        \end{block}
        \onslide<2->
        \begin{block}{Asesoría Académica}
          \begin{itemize}
          \item Tiene una aplicación más amplia y genérica.
          \item Aparece la figura del Asesor Académico.
          \end{itemize}
        \end{block}
      \end{frame}

      \begin{frame}
        \begin{block}{Asesor Académico}
          \begin{itemize}
           \item Orienta al estudiante durante su estancia en la Universidad.
           \item Realiza un seguimiento del alumno:
           \begin{itemize}
            \item Permanente.
            \item Eficaz.
            \item Orientado a la optimización del esfuerzo de estudio.
           \end{itemize}
          \end{itemize}
        \end{block}
      \end{frame}

      \begin{frame}
        \begin{block}{Asesoría Académica en la UCO}
          \begin{itemize}
           \item En marzo de 2007, a través del Plan Propio de Calidad de la
                 Enseñanza, se contempla la creación de la figura del Asesor
                 Académico.
          \end{itemize}
        \end{block}
        \onslide<2->
        \begin{block}{Asesoría Académica en la UCO}
          \begin{itemize}
           \item Se establece un reglamento regulador:
           \begin{itemize}
            \item Guías.
            \item Fichas.
           \end{itemize}
          \end{itemize}
        \end{block}
      \end{frame}


    \subsection{Definición del problema}
      \begin{frame}{Definición del problema}
        \begin{block}{Problema}
          \begin{itemize}
           \item Reciente implantación del Asesor Académico.
           \begin{itemize}
            \item Organización de la información poco eficiente.
           \end{itemize}
          \end{itemize}
         \end{block}
           \onslide<2->
         \begin{block}{Dificultades}
          \begin{itemize}
           \item Guías y fichas en papel:
           \begin{itemize}
            \item Gran cantidad de información.
            \item Acceso.
            \item Actualización/modificación.
           \end{itemize}
          \end{itemize}
        \end{block}
      \end{frame}

      \begin{frame}
        \begin{block}{Necesidad de un sistema de gestión}
          \begin{itemize}
           \item Fácil acceso a la información.
           \item Almacenamiento.
           \item Fácil manipulación de los datos.
           \item Interactividad.
          \end{itemize}
        \end{block}
      \end{frame}


    \subsection{Objetivos}
      \begin{frame}{Objetivos}

      \end{frame}

    \subsection{Antecedentes}
      \begin{frame}{Antecedentes}

      \end{frame}

    \subsection{Restricciones}
      \begin{frame}{Restricciones}

      \end{frame}

    \subsection{Recursos}
      \begin{frame}{Recursos}

      \end{frame}


  \section{Análisis}
    \subsection{Especificación de requisitos}
      \begin{frame}{Especificación de requisitos}

      \end{frame}

    \subsection{Modelización de la información}
      \begin{frame}{Modelización de la información}

      \end{frame}

    \subsection{Descripción funcional}
      \begin{frame}{Descripción funcional}

      \end{frame}


  \section{Diseño}
    \subsection{Diseño de datos}
      \begin{frame}{Diseño de datos}

      \end{frame}

    \subsection{Diseño arquitectónico}
      \begin{frame}{Diseño arquitectónico}

      \end{frame}

    \subsection{Diseño de la interfaz}
      \begin{frame}{Diseño de la interfaz}

      \end{frame}


  \section{Pruebas del software}
    \begin{frame}{Pruebas del software}

    \end{frame}


  \section{Demostración del sistema}
    \begin{frame}{Demostración del sistema}

    \end{frame}


  \section{Conclusiones}
    \subsection{Conclusiones}
      \begin{frame}{Conclusiones}

      \end{frame}

    \subsection{Futuras mejoras}
      \begin{frame}{Futuras mejoras}

      \end{frame}

\end{document}